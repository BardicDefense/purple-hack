\documentclass{article}
\usepackage{geometry, booktabs, tabularx, threeparttable, float, changepage}
\usepackage{graphicx} % Required for inserting images
\usepackage{hyperref}
\usepackage{longtable}
\usepackage{multicol}
\usepackage[export]{adjustbox}
\usepackage{wrapfig2}
\usepackage[raggedrightboxes]{ragged2e}

\providecommand{\tightlist}{%
\setlength{\itemsep}{0pt}\setlength{\parskip}{0pt}}
 
\setcounter{tocdepth}{2}



\title{Purple Hack}
\author{Maximilian Hart}
\date{Last Updated: December 2025}

\usepackage{titling}
\renewcommand\maketitlehooka{\null\mbox{}\vfill}
\renewcommand\maketitlehookd{\vfill\null}

\begin{document}

\begin{titlingpage}
\maketitle
\end{titlingpage}

\newpage

\tableofcontents

\vfill

The following are ``house rules'' and clarifications, a mish-mash of D\&D Basic/Expert a la \emph{Old-School Essentials}, \emph{Dolmenwood}, and a fantastic ruleset called \emph{\href{https://osrsimulacrum.blogspot.com/2021/06/simulacrum-beta-release.html}{Simulacrum}}. Rampant stealing has occurred below, and is intended only for use in my own home games unless it's covered by fair use in which case I'm happy to share. Please don't sue me.

Links: \href{https://docs.google.com/spreadsheets/d/1wOavveRJ8IEVoqvy33bxn12m20Y_7lqOkc9_KuXwJHo/copy}{Google Sheets Character Sheet} • \href{https://monogr.ph/694bf3d97da4a20a582eec2c}{Patch Notes}

\newpage



\section{Character Creation}\label{character-creation}

%\begin{multicols}{2}

\subsection*{Roll Your Stats}\label{roll-your-stats}

Roll 3d6 down the line for each of the skills below. You can swap one
set of stats. If your total ability score modifiers equal -3 or lower,
you may discard the character and start again.

\begin{itemize}
\tightlist
\item
  \textbf{Strength} measures physical might. It affects your lifting,
  carrying, and grappling. Any Str modifier is applied to your melee and
  thrown weapon attack damage.
\item
  \textbf{Dexterity} measures agility, reflexes, and hand-eye
  coordination. Any Dex mod is applied to your Armour Class, as long as
  you can move freely.
\item
  \textbf{Constitution} measures endurance and vitality. Any Con mod is
  applied to your Hit Point total at the start of the game and each time
  you go up a Level.
\item
  \textbf{Perception} measures your observational acuity. Any Per mod
  affects your ability to see through illusions, and to spot traps,
  secret doors, and ambushes.
\item
  \textbf{Willpower} measures discipline and mental endurance. Any Will
  mod affects your ability to resist enchantment, fear, mind control,
  possession, and other mental attacks.
\item
  \textbf{Arcana} measures your connection to magical forces. Any Arc
  mod is applied to chances to learn spells and to saving throws to
  resist spell damage.
\end{itemize}

\subsection*{Choose a Character Race \& Make Any
Adjustments}\label{choose-a-character-race-make-any-adjustments}

\begin{itemize}
\tightlist
\item
  \textbf{Humans}: The most flexible \& common. +1 to one stat of your
  choice.
\item
  \textbf{Dwarf}: +1 Con, -1 Dex. Add Con mod to saves vs poison. Cannot
  wield large weapons.
\item
  \textbf{Elf}: +1 Per, -1 Str. Add Arc mod to saves vs magic. 90\%
  resistance to sleep and charm spells. Immune to ghoul paralysis.
\item
  \textbf{Half-Elf}: 30\% resistance to sleep and charm spells.
\item
  \textbf{Halfling}: -1 Str, +1 Dex. +1 bonus to missile weapon attacks.
  +2 AC vs large enemies. Cannot wield large weapons.
\item
  \textbf{Breggle}: +1 AC if wearing no or light armour; natural horn
  attack that deals Level/2 damage, rounded down.
\end{itemize}

\subsection*{Note Ability Score Modifiers}\label{note-ability-score-modifiers}

\begin{multicols}{2}

\begin{table}[H]
\centering
\begin{tabular}{cc}
\toprule\
Ability Score & Modifier \\
\midrule
    2-3 & -3 \\
    4-5 & -2 \\
    6-8 & -1 \\
    9-12 & 0 \\
    13-15 & +1 \\
    16-17 & +2 \\
    18-19 & +3 \\
\bottomrule
\end{tabular}
\end{table}

\columnbreak
\begin{figure}[H]
	\centering
	\includegraphics[width=.7\columnwidth]{img/dice.jpg}
	\label{fig:dice}
\end{figure}

\end{multicols}

\newpage

\subsection*{Note Your Saving Throw
	Bonus}\label{note-your-saving-throw-bonus}

\textbf{Save Bonus:} Starts at 0. Characters gain a +1 save bonus for
every 2 full Hit Dice they have.

\begin{wrapfigure}[11]{l}{0.4\linewidth}
	\centering
	\includegraphics[width=1\linewidth]{img/dwarfarmour.jpg}
	\label{fig:dwarf}
\end{wrapfigure}

\paragraph{}

\subsection*{Choose Your Class \& Note Hit
Die}\label{choose-your-class-note-hit-die}

\subsubsection*{Warrior}\label{warrior}

1d8 Hit Die (8hp+Con mod at 1st Level). Receive attack bonus of +1 every
Level, starting at Level 1, stopping at +15 at Level 15. Can use any
weapon or armour, so long as they meet its strength minimum, if any (see
\hyperref[armour-weapons-and-equipment]{equipment}). Every Name Level,
warriors add an extra weapon die of damage to their armed combat attacks
and an additional feat. At 1st Level, warriors select one style,
reflecting their preferred manner of fighting:

\paragraph{}

\begin{itemize}
\tightlist
\item
  \textbf{Arcanist}: You can read Mithric and gain access to two schools
  of magic of your choice. You can cast spells in armour and if damaged
  while casting can make a Constitution 14 saving throw to avoid the
  spell being lost. You cannot create or cast from scrolls, though you
  can copy spells from them. You have half the spell slots of a mage of
  your same Level (rounded down, to a minimum of 1 slot). You start with
  a spellbook and spells the same way a mage does, but no schools or
  spells are automatically acquired through gaining Levels.
\item
  \textbf{Hordeslayer}: If you kill an opponent with an attack, you can
  immediately make a bonus attack of the same kind (melee or ranged).
  You can make a maximum number of melee bonus attacks per round equal
  to your Level, and a maximum number of missile bonus attacks equal to
  your number of name Levels plus 1.
\item
  \textbf{Smiter}: Once per combat encounter, you can declare a smite
  after you score a hit, melee or ranged, which doubles the number of
  damage dice rolled. If a smite was announced on a critical hit, the
  extra weapon dice do not automatically deal maximum damage.
\item
  \textbf{Knight}: You swear fealty to a noble house; your knighthood
  may be revoked if you displease or dishonour your liege (it may be
  possible to regain said favour and knighthood status by performing a
  special quest). You follow a code of chivalry (honour, service, glory,
  protecting the weak, and uphold the hierarchy). You gain a +1 Attack
  bonus when mounted and a +2 to Saving Throws against magic and effects
  that cause fear. You are a ``squire'' until 3rd Level, whereupon you
  are knighted. Once knighted, you and companions for whom you vouch
  earn rights of hospitality and aid from nobles and friendly knights
  (you are also expected to extend such hospitality in kind), and you
  gain a +2 bonus to Attack and Damage rolls against Large creatures.
\end{itemize}

\newpage

\begin{wrapfigure}[12]{r}{0.4\linewidth}
	\centering
	\includegraphics[width=1\linewidth]{img/elfmage.jpg}
	\label{fig:elfmage}
\end{wrapfigure}

\paragraph{}


\subsubsection*{Mage}\label{mage}

1d6 Hit Die (6hp+Con mod at 1st Level). Can cast one spell per round.
Can use any weapon or armour, so long as they meet its strength minimum,
if any, but they can't normally cast spells in armour. Receive attack
bonus of +1 every two Levels, starting at Level 2. All spells are
divided into eight schools (see \hyperref[magic-spells]{Magic \&
Spells}). By default, mages have access to four schools (one chosen, and
three rolled randomly). After rolling, a mage can trade away access to
one school (up to two maximum) in exchange for one of the following
special abilities (each can be chosen only once, and only at character
creation):

\begin{itemize}
\tightlist
\item
  \textbf{Battlemage}: You can cast spells while wearing light armour,
  and your class attack bonus improves to +1 every Level, starting at
  Level 1, stopping at +15.
\item
  \textbf{Focused}: Select one additional mage feat.
\item
  \textbf{Innate}: While you can still use them, you need not own a
  spellbook or consult one to prepare your spells.
\item
  \textbf{Specialist}: You have one additional spell slot at each spell
  Rank. Choose one school you can already access; when you gain a Level,
  you learn one extra random new spell from this school. Spells from
  this school are easier to bind.
\end{itemize}

\textbf{Starting Spells}: Mages start with one random 1st-Rank spell
from each school to which they have access. Specialists then select one
extra 1st-Rank spell of their choice from their specialty school. Mages
can prepare a limited number of spells each day, and gain further spell
slots as they gain Levels (see \hyperref[magic-spells]{Magic \&
Spells}).

\subsection*{Determine Background}\label{determine-background}

Using the \href{background.md}{character background generator},
determine your character's height, weight, family upbringing, and
progenitor, and formative institution.

\subsection*{Choose Languages}\label{choose-languages}

You start play knowing your native language(s). Additionally, roll 1d6.
On a 5, you gain one more language. On a 6, you gain two more languages.
Mages automatically know Mithric (the language of magic).

\subsection*{Choose Feat(s)}\label{choose-feats}

All characters receive one feat at 1st Level, and another feat at each
Name Level. A feat can only be taken once, unless noted otherwise. No
ability score can be raised above 19 through feats; a feat that would do
so may still be taken but will provide no ability score increase.

\subsubsection*{Warrior feats}\label{warrior-feats}

\begin{itemize}
\tightlist
\item
  \textbf{Battle Rage}: When in melee, the warrior can choose to enter a
  berserk rage that last until the end of combat. While in the rage, the
  warrior gains a +2 bonus to Attack and Damage rolls, but suffers a -4
  penalty to Armour Class and is unable to flee.
\item
  \textbf{Brawler}: If your melee opponent is no more than one size
  Level larger than you, and your attack roll against them is a natural
  18 or 19, then in addition to your regular damage you can either:

  \begin{enumerate}
  \def\labelenumi{\arabic{enumi}.}
  \tightlist
  \item
    Disarm them
  \item
    Trip them (they become prone; two-legged creatures only)
  \item
    Drive them directly back 5 feet, if the space is available to do so.
    You may follow up immediately with a free 5-foot move of your own,
    even if you have already made your full move this turn or are
    otherwise locked in combat.
  \end{enumerate}
  If none of those are possible or desired, then instead you bash them
  for an additional 1d4 damage (+2 per Name Level).
\item
  \textbf{Captain}: Some are born to command. Add a +1 attack bonus to
  all other party members and associated NPCs, raised to +2 at Level 10
  or higher (this does not benefit yourself). Apply +1 to friendly
  Morale checks. These bonuses apply only as long as your orders can be
  understood and the individuals benefiting are willing to be led by
  you. Multiple captains in a group do not stack these benefits.
\item
  \textbf{Defender}: If you decide that none will pass, then \emph{none
  will pass}. You ignore all magical commands to move aside, flee,
  surrender and the like, and are immune to all fear-based effects,
  magical or not. You gain a 4-point modifier in your favor when
  resisting any other effect that would result in you being
  involuntarily moved. Also, in combat, you can always choose to receive
  the effects of having taken the Guard action, even if taking another
  action. If you actually choose the Guard action, you impart a +4 bonus
  to AC instead of +2.
\item
  \textbf{Great-Weapon Fighter}: When attacking with a two-handed melee
  weapon, your critical hit range improves by 1, plus 1 per Name Level
  (e.g.~you score critical hits on a natural to-hit roll of 19-20 at
  Level 1, 18-20 at Level 5, etc.).
\item
  \textbf{Marksman}: You can fire into melee without penalty. Your
  ranged to-hit penalties are reduced by 2 points, plus 2 per Name
  Level. Your rate of fire with small thrown weapons increases from 1 to
  2.
\item
  \textbf{Read Scrolls}: You can read Mithric, as well as cast from
  scrolls containing spells from four schools of your choice; if an
  Arcanist, two of these schools must be the two schools you already
  know. This does not grant the ability to create scrolls.
\item
  \textbf{True Grit}: You may reroll failed death saves.
\item
  \textbf{Whirlwind}: +2 AC (and a further +1 AC per Name Level). Your
  melee attacks happen first before all other simultaneous attacks in
  the melee phase. You are not locked in melee combat unless in melee
  with at least three opponents. To gain these benefits, you must be
  able to move freely at at least Speed 30, not wearing medium or heavy
  armour, and not be fatigued.
\end{itemize}

\subsubsection*{Mage Feats}\label{mage-feats}

A mage can combine multiple feats on a spell. The spell adjustments from
these stack (e.g.~a \emph{Silent Undeniable Magic Missile} spell would
in all ways be treated as a 3rd-Rank spell). A feat cannot take a spell
over 6th Rank.

\begin{itemize}
\tightlist
\item
  \textbf{Arcane Antipathy}: You dedicate yourself to rooting out
  practitioners of dark magic. You gain a +2 bonus to Saving Throws
  against magic, and spellcasters suffer a -2 penalty to Saving Throws
  against spells cast by you.
\item
  \textbf{Detect Magic}: You gain a ``Detect Magic'' skill with a
  starting Skill Target of 6 that allows you to touch an object, place,
  or creature for 1 Turn to see if it is magical. On a success, you know
  if the object, place, or creature is magical--i.e., enchanted,
  consecrated, affected by a spell, or possessed of innate magic of some
  kind. You can repeat the attempt as many times as you like, each time
  taking 1 Turn. The Skill Target lowers by 1 every Name Level.
\item
  \textbf{Dextrous}: You can cast 1st-Rank spells using only one hand
  (i.e.~with a shield or weapon or lantern in the other hand). For each
  Name Level you have, the Rank of spells that can be cast in this way
  increase by one.
\item
  \textbf{Familiar}: You acquire a magical connection to a Tiny or Small
  mundane creature appropriate to the area that obeys your commands. You
  can see through its eyes up to a mile away, and gain a small power
  appropriate to the creature while doing so (e.g.~+2 visual Perception
  for a bird). Regardless of its normal statblock, the creature has 2
  Hit Dice. Its death applies one level of fatigue to you for the next
  week, after which you may take a new familiar.
\item
  \textbf{Metamagic}: You can craft your spells at the time of
  memorization to require either no vocal component (``Silent'') or no
  somatic component (``Stilled''). Memorizing a Metamagic spell uses up
  a spell slot one Rank higher than the spell's normal Rank. This feat
  may be taken three times. The second time removes either vocal or
  somatic components from your all spells permanently (must be chosen at
  the time of choosing the feat), with no Rank adjustment to them
  required. A spell may not be both Silent and Stilled until the feat is
  taken a third time, which allows this combination.
\item
  \textbf{Quickcast}: Casting times of your spells are reduced by 1 (see
  \hyperref[combat-phases]{magic phase}). This can take them to 0 or
  below. This feat may be taken up to twice.
\item
  \textbf{Undeniable}: Once an Undeniable spell is declared, loss of
  concentration does not disrupt it (unless you are killed or otherwise
  rendered incapable of casting). All other restrictions apply.
  Preparing an Undeniable spell uses up a spell slot one Rank higher
  than the spell's normal Rank.
\end{itemize}

\subsubsection*{Unrestricted Feats}\label{unrestricted-feats}

\begin{itemize}
\tightlist
\item
  \textbf{Animal Companion}: You may attempt to forge a bond with a
  single wild or domestic ordinary animal. If the animal companion dies
  or is dismissed, you may attempt to forge a connection with a new
  animal. An animal companion may not exceed your Level and does not
  level in any way. To establish a connection, you must approach the
  animal and interact in a peaceful manner for 1 Turn. On a successful
  Animal Handling check with a default Skill Target of 4, the animal
  becomes your companion. Once a connection is established, the
  companion follows you everywhere, understands basic commands, and
  fights to the death to defend you.
\item
  \textbf{Anointed}: \emph{Note: A Warrior can only take this if they
  are an Arcanist.} You are zealous for your God and a saint whose
  particular patronage you choose. Spells or effects that would target
  supernatural creatures aligned with your patron also affect you.

  \begin{itemize}
  \tightlist
  \item
    You lose access to the usual arcane spell list and instead gain
    access to the holy spell list. You do not need a spellbook, and
    instead pray daily for spells, choosing from all the spells
    available to you each day. You must carry a holy symbol and hold it
    to cast your spells.
  \item
    Gain access to the Turn Undead spell, but as a once-per-Turn ability
    that does not consume any spell slots.
  \item
    Gain the ``Laying on Hands'' ability, a healing touch once per day
    that heals HP equal to your Level.
  \end{itemize}
\item
  \textbf{Conditioning}: Gain +2 to a chosen ability score. This feat
  can be taken only once for a given ability score.
\item
  \textbf{Fieldcraft}: Gain +1 Constitution. Pick two broad terrain
  types (forest, hills, plains, desert, swamps, jungle, tundra,
  mountains, etc.). In these terrain types, you gain a +1 on tasks such
  as stealth, perception, tracking, and concealing tracks, and you:

  \begin{itemize}
  \tightlist
  \item
    Heal at least 2 HP per night if you get a good night's sleep, and
    still heal even if marching in terrain that normally prevents such.
  \item
    Lower your Survival and Tracking Skill Targets by 1.
  \item
    You're less likely to be surprised in your chosen terrain
  \end{itemize}
\item
  \textbf{Lockpicking}: Gain +1 Dexterity. You have the tools and
  expertise to pick locks; add -2 to the Lockpicking Skill Target (or
  start at 4 if the Skill is not already acquired), and add a +1 to your
  Perception checks to notice trapped locks.
\item
  \textbf{Tough}: Your hit die is one die higher (i.e.~a warrior will
  use a d10, and a mage will use a d8). If you take this feat later than
  1st Level, reroll your current hit dice with the new hit die size, and
  take whichever maximum hit point total is higher.
\item
  \textbf{Undead Slayer}: Gain a +1 Attack bonus and +1 weapon die of
  damage against undead monsters. Your attacks harm undead monsters that
  can normally only be harmed by magical or silver weapons, even when
  not wielding a weapon of the appropriate type.
\end{itemize}

\subsection*{Determine Starting Age \& Choose
Skills}\label{determine-starting-age-choose-skills}

Using the \href{background.md\#starting-age}{character background
generator}, determine your character's starting age. Then, select
appropriate skills.

Skills represent specialized knowledge of or training in a particular
field. All characters select one to three skills at 1st Level, depending
on starting age, and two more at each Name Level.

All characters start with innumerable general adventuring skills (like
Survival, Climbing, or Stealth) at a Skill Target of 6. In general, each
time a character selects a skill, the Skill Target of that skill is
reduced by 1. A Skill Target cannot be reduced below 2. (See
\hyperref[checks-saves]{checks \& saves}.)

Other skills are more specialized or non-intuitive (like reading lips or
lockpicking), and reflect special training, deep knowledge, or intense
\& focused practice in that particular narrow area. These skills must be
selected in order to even be attempted by a character. Which skills are
general skills and which are specialized which is up to the referee.

\begin{quote}
Note: Broad interaction (e.g.~Conversation), social skills that achieve
a result otherwise possible only through roleplay (e.g.~Deception or
Intimidation), skills that boil down a non-linear non-standard task down
to a simple roll (e.g.~Dungeoneering or Investigation) or skills that
allow for ready identification of magical items cannot be taken.
\textbf{Skills are largely meant to measure a proficiency over and above
what one could reasonably be expected to have, NOT to define what is
possible!}
\end{quote}

Concretely, skills might, depending on the circumstances, allow you to:

\begin{itemize}
\tightlist
\item
  Avoid what might otherwise involve a roll.
\item
  Lessen the consequences of failure.
\item
  Gain information not obvious to the average observer.
\end{itemize}

Sample general skills include: Acrobatics, Alertness, Climbing,
Disguise, Etiquette, First Aid, Gambling, Herbalism, Jumping, Language
(\emph{choose one}), Lore (\emph{specific subject}), Navigation
(Nautical), Performance, Riding, Running, Shadowing, Ship Piloting,
Stealth, Survival, Swimming, Tracking, Wrestling.

Sample specialized skills include: Alertness, Blindfighting, Decipher
Document, Herbalism, Lockpicking, Pick Pockets, Read Lips.

Examples of some skills that impart concrete mechanical benefits other
than a simple -1 to a Skill Target:

\begin{itemize}
\tightlist
\item
  \textbf{Alertness}: Make an Alertness check to act normally if
  surprised.
\item
  \textbf{Acrobatics}: Move at full speed through allies' spaces.
\item
  \textbf{Blindfighting}: Removes penalty for fighting while blinded.
\item
  \textbf{Herbalism}: A single dose of a medicinal herb is sufficient
  for two subjects.
\item
  \textbf{Tracking}: Find tracks left by creatures in a 30' square area.
  Takes 1 Turn to search, with only one attempt allowed. Some modifiers
  may apply, and it can only be done in the wild.
\item
  \textbf{Wrestling}: Grants a +2 on grappling attack rolls.
\end{itemize}



\subsection*{Buy Equipment}\label{buy-equipment}

Roll 6d6 and multiply the number by 10. That's your starting silver
pieces (d.). See the \hyperref[armour-weapons-and-equipment]{equipment}
list for things to buy (don't forget armour!) or choose a
\hyperref[quick-packs]{quick pack}.

If you plan to cast Turn Undead, be sure to buy a holy symbol! If you're
a thief, you'll want lockpicks. A magic-user needs a book to use as a
spellbook.

\subsection*{Note a Motivation or
Motto}\label{note-a-motivation-or-motto}

\begin{wrapfigure}[10]{l}{0.3\linewidth}
	\centering
	\includegraphics[width=1\linewidth]{img/portalthief.jpg}
	\label{fig:portalthief}
\end{wrapfigure}

Why is your character risking their life for adventure? This can
be one word, or a short sentence. Examples:

\begin{itemize}
\tightlist
\item
  Rejected from his clan, Gorend is on a mission to prove himself.
\item
  Percival always has to be the hero.
\item
  Cedric can't resist a good story.
\item
  Spread the faith
\item
  Earn glory
\item
  Amass wealth
\item
  Take revenge
\item
  Master a skill
\item
  Obey duty
\item
  Discover truth
\item
  Do good
\item
  Help others
\end{itemize}
%\end{multicols}

\newpage

\section{Armour, Weapons, and
Equipment}\label{armour-weapons-and-equipment}


\subsection*{Item Slots}\label{item-slots}

We're using container-based ``item slot'' encumbrance, measuring both
weight and awkwardness.

You can carry 10 + Str mod slots of equipped items (anything you're
wearing, holding, actively using, ready to use at short notice). You can
carry 16 + Str mod slots of stowed items. Your encumbrance--which
determines your speed in combat and exploration, and your travel points
available--is based on how many slots you have occupied:

\begin{table}[ht]
\centering
\begin{adjustbox}{width=1\textwidth,center=\textwidth}
\begin{tabular}{cccccc}
\toprule
    Equipped Slots & Stowed Slots &
    Combat Speed & Exploration Speed &
    Normal Travel & Forced March \\
    \midrule
    0-3 & 0-10 & 40' & 120' & 8 pts & 12 pts \\
    4-5 & 11-12 & 30' & 90' & 6 pts & 9 pts \\
    6-7 & 13-14 & 20' & 60' & 4 pts & 6 pts \\
    8-10 + Str mod & 15-16 + Str mod & 10' & 30' & 2 pts & 3 pts \\
\bottomrule
\end{tabular}
\end{adjustbox}
\end{table}

For ease of communication, each of these tiers of encumbrance are
referred to by their associated combat speed. For example, moving at
``Speed 20'' means you have 4 travel points, or 6 if you force march.

A backpack can only carry 12 stowed slots of items; more than that, and
you must use a sack. 13-14 slots of items means you're carrying the sack
with one hand, and 15 or more slots requires two hands to carry.

\begin{itemize}
\tightlist
\item
  \textbf{Small} (S) items (like chalk or potions) fit four to a slot.
\item
  \textbf{Medium} (M) items (most things) are 1 slot.
\item
  \textbf{Large} (L) items like 2h weapons and medium armour are 2
  slots.
\item
  \textbf{Heavy armour} takes 3 slots (as do all items listed as L+) and
  adds a level of encumbrance.
\end{itemize}

Coins \& gems stack 500 to a slot. A typical body (willing or
unconscious) fills 9 slots, before gear, and adds a level of
encumbrance.



\subsection*{Starting Gear}\label{starting-gear}

\begin{wrapfigure}[9]{r}{0.3\textwidth}
	\centering
	\includegraphics[width=1\linewidth]{img/backpack.jpg}
	\label{fig:backpack}
\end{wrapfigure}

\paragraph{}

The following base gear is automatically added to your character for
free:

\begin{itemize}
\tightlist
\item
  One weapon of your choice (+ 20 arrows/bolts if missile weapon) with a
  scabbard/sheathe/quiver if needed, (number of slots depends on size)
\item
  Backpack (10 + Str mod slots) (worn clothing \& in-use containers do
  not count toward your slots), waterskin (M), tinderbox (S) (for
  lighting torches or small fires)
\item
  Miscellaneous tiny items (within reason); worn items/clothing
\end{itemize}

\newpage

After this, you can have whatever basic gear you wish (within reason, as
determined by the referee). Some common basic adventuring items include
but are not limited to (size listed after each):

\begin{itemize}
\tightlist
\item
  \textbf{Bedroll (M)}: A blanket that can double as a sleeping bag.
  Helps protect against the cold when
  \hyperref[camping-in-the-wilds]{camping in the wilds}.
\item
  \textbf{Caltrops, bag (M)}: Small metal spikes sufficient to cover a
  5' × 5' area. Creatures moving through have a 2-in-6 chance of
  treading on a spike for a 50\% penalty to movement rate for 24 hours
  (or until magically healed). Intelligent creatures can move cautiously
  through areas with known caltrops, which requires their entire
  Movement Phase to travel 5', but eliminates any risk of impalement.
\item
  \textbf{Candle (S)}: Dimly illuminates a 5' radius, and burns for 4
  hours.
\item
  \textbf{Chalk (S)}: Useful for marking trails in dungeons.
\item
  \textbf{Crowbar (M)}: Makes it easier to force open doors, and doubles
  as an improvised medium weapon.
\item
  \textbf{Food, fresh, poor} (M): Poor-quality fresh food, 1 day's
  worth. Won't win awards, but it'll keep you fed.
\item
  \textbf{Food, trail} (M): Preserved and packaged food, 1 day's worth.
  Won't go bad.
\item
  \textbf{Grappling Hook (S)}: Steel, 1'x8'', four prongs
\item
  \textbf{Hammer (S)}: Doubles as an improvised small weapon.
\item
  \textbf{Sack, canvas (M)}: In case your backpack overflows with
  treasure.
\item
  \textbf{Spike, iron x4 (S)}: One of these can be hammered in to block
  one typical door.
\item
  \textbf{Lantern (M)}: A lantern can be closed to hide its light, burns
  one flask of lamp oil every 4 hours, and illuminates a 20' radius.
\item
  \textbf{Oil flask (M)}: Fuels lantern 4 hours. Poured on ground and
  lit burns for 1 turn. Thrown on monster (roll to hit) \& set on fire
  does 1d6 damage each round for two rounds.
\item
  \textbf{Pick, mining (M)}: This excavating tool doubles as an
  improvised medium weapon.
\item
  \textbf{Pole, 10' (L)}: Often used to prod potential dangers at a safe
  distance. When wielded during cautious exploration, has a 2-in-6
  chance of setting off most traps.
\item
  \textbf{Rope, hemp, 3/8'' (L)}: Holds 250lbs, 50' length.
\item
  \textbf{Scroll case, leather (M)}: Each holds up to 10 spell Ranks in
  spell scrolls.
\item
  \textbf{Shovel (L)}: Excavating tool doubles as an improvised medium
  weapon.
\item
  \textbf{Sledgehammer (L)}: Excavating tool doubles as an improvised
  medium weapon.
\item
  \textbf{Torch, wax (S)}: Burns for 1 hour, illuminating a 40' radius.
  Will remain lit if dropped to the ground or thrown. Can be used as a
  small weapon without dousing it; some targets take further damage due
  to the fire. Also useful to burn away rot grubs, green slime, and
  webs.
\item
  \textbf{Twine, hemp, 1/8'' (S)}: Ball, holds 10lbs, 100' length.
\item
  \textbf{Vial, glass (S)}: 4oz, 1.4'' dia, 5'' high, cork stopper.
\end{itemize}

Finally, you can add ``costed'' items, which do cost money at character
creation.

\begin{quote}
\emph{\textbf{Note}: After character creation, all items you buy in the
world are ``costed items,'' even if they appear on the free list above;
their prices may differ from the prices below.}
\end{quote}

\newpage

\subsubsection*{Costed Items}\label{costed-items}
\begin{multicols}{2}
\vspace{5mm}
\begin{quote}
    \emph{40 copper farthings (f.) = 10 silver pence (d.) = 1 gold noble
(g.)}
\end{quote}
\begin{table}[H]
\begin{adjustbox}{width=0.9\columnwidth,center=\columnwidth}
\begin{tabular}{ll}
\toprule
Item & Cost \\
\midrule
Acid, vial (S) & 50d. \\
Arrow or light bolt, silver, single & 5d. \\
Book, leatherbound, 24pgs (M) & 30d. \\
Chain, iron, 1'' links, 10' (L) & 40d. \\
Holy symbol, iron & 5d. \\
Holy symbol, silver & 50d. \\
Holy symbol, wooden & 1d. \\
Holy water vial (S) & 25d. \\
Mapping kit (M) & 10d. \\
Mirror, small, bronze (S) & 2d. \\
Rope, silk, 3/8'' (M) & 75d. \\
Tent, single (L) & 20d. \\
Thieves' picks \& tools (M) & 12d. \\
Wolfsbane sprig (S) & 1d. \\
\bottomrule
\end{tabular}
\end{adjustbox}
\end{table}

\begin{figure}[H]
    \centering
    \includegraphics[width=0.5\linewidth]{img/helmsword.jpg}
    \label{fig:helmsword}
\end{figure}

\columnbreak

\textbf{Acid, vial}: Can splash contents on a target within 5' or hurl
it as a small thrown weapon; it shatters on impact. A hit deals 1d6+1
acid damage. A vial can also be used to open most mundane locks in 1
turn.

\textbf{Holy symbols}: Wooden symbols incur -1 penalty to Turn Undead
2d6 dice roll; silver symbols receive +1 bonus.

\textbf{Holy water}: Can splash contents onto an undead creature within
5' or hurl it as a small thrown weapon; it shatters on impact. Causes
2d4 damage when thrown on most undead.

\textbf{Mapping kit}: A cased roll of parchment plus quills and vials of
ink, sufficient to map all but the largest areas.

\textbf{Rope, silk}: Lighter and stronger than hemp, can hold 500lbs.
50' length.

\textbf{Tent}: Protects against adverse weather when adventuring in the
wilds. See \hyperref[camping-in-the-wilds]{camping in the wilds}.

\textbf{Wolfsbane}: A werewolf must make a successful saving throw
against a target of 11 each round in order to make a melee attack
against someone decorated with sprigs of wolfsbane.
\end{multicols}

\subsection*{Quick Packs}\label{quick-packs}

\begin{itemize}
\tightlist
\item
  \textbf{The Gygax} (10 slots, 25d. cost): Hempen rope 50' (L), 10'
  pole (L), Iron spikes x4 (S) x2, Hammer (S), Lantern (M), Oil flask
  (M) x2, Holy water vial (S, 25sp), Trail food (M) x2.
\item
  \textbf{The Generalist} (6 slots, 10d. cost): Candle (S), Chalk (S),
  Hammer (S), Iron spike x4 (S), Crowbar (M), Trail food (M), Oil flask
  (M), Lantern (M), Mapping kit (M, 10d.).
\item
  \textbf{The Cautious} (7 slots): Candle, pillar (S) x4, Chalk (S),
  Hammer (S), Iron spikes x4 (S), Wolfsbane (S), Caltrops (M), Trail
  food (M) x2, 10' pole (L).
\item
  \textbf{The Delver} (6 slots): Candle, pillar (S) x3, Hammer (S), Beef
  jerky (S) x2, Cheese (S), Dried fruit (S), Pickaxe (L) or Shovel (L)
  or Sledgehammer (L), Hempen rope (L).
\item
  \textbf{The Scholar} (6 slots, 10d. cost): Chalk (S), Vial (S) x3,
  Trail food (M), Oil flask (M), Lantern (M), Mapping kit (M, 10d.),
  Scrollcase (M).
\item
  \textbf{The Torchbearer} (6 slots): Trail food (M) x2, Oil flask (M)
  x2, Torches (S) x8
\end{itemize}

\subsection*{Armour}\label{armour}

An unarmoured character has a base Armour Class of 10.

\begin{table}[ht]
\centering
    \begin{tabular}{lccc}
    \toprule
    Armour & AC & Cost & Time to Don/Doff \\
    \midrule
    Small Shield (M) & +1 & 25d. & 1 action \\
    Large Shield (L) & +2 & 60d. & 1 action \\
    Leather or Furs (light armour) (M) & 12 & 80d. & 1 min \\
    Ring (med. armour) (L) & 13 & 130d. & 5 min \\
    Scale/Lamellar (med. armour) (L) & 14 & 180d. & 5 min \\
    Chainmail (med. armour) (L) & 15 & 280d. & 5 min \\
    Splint (heavy armour) (L+) & 16 & 580d. & 10 min \\
    Plate (heavy armour) (L+) & 17 & 850d. & 10 min \\
    \bottomrule
    \end{tabular}
\end{table}

Heavy armour (L+ = 3 slot) and Large Shields require minimum 9 Str to equip. Heavy armour adds a level of encumbrance. A Shield also provides a save bonus vs non-gaseous breath weapons equal to its AC bonus.

\subsection*{Weapons}\label{weapons}

\subsubsection*{Melee Weapons}\label{melee-weapons}


Weapons are divided into three basic damage categories: small, medium,
and large.

\begin{itemize}
\tightlist
\item
  \textbf{Small} weapons deal 1d4 damage (plus the user's Str modifier),
  are one-handed, and include knives, daggers, and small handaxes. Can
  be dual-wielded for +1 to hit (doesn't grant additional attacks) and
  can often be thrown as missile weapons. Average cost: 12d.
\item
  \textbf{Medium} weapons deal 1d6+1 damage (plus the user's Str
  modifier), are one-handed, and include most swords, spears, maces,
  flails, and battle axes. Can be wielded two-handed, which adds +1
  damage. Average cost: 20d.
\item
  \textbf{Large} weapons deal 1d8+3 damage (plus the user's Str
  modifier), are two-handed, and include polearms and pikes, as well as
  large swords, spears, axes, and maces. A minimum Str of 9 is required
  to wield large weapons. Average cost: 75d.
\end{itemize}

\paragraph{Weapons of Special Metals}\label{weapons-of-special-metals}

Adventurers sometimes seek out weapons made of cold iron or silver, as
these are harmful to fey or certain undead monsters. Cold iron and
silver weapons must be crafted to special order, incurring a delay of
2d6 days.

\textbf{Cold Iron Weapons} are made of pure iron, forged in flames of
ancient yew wood, and doused in water infused with mandrake root. They
cost double the weapon's normal price. They gain a +1 damage bonus
against fey and demi-fey, but suffer a -1 damage penalty against other
creatures.

\textbf{Silver Weapons} are made of pure silver. They inflict +1 damage
against undead, and normal damage against other foes, and certain foes
can \emph{only} be harmed by silver (or magic) weapons. They don't hold
an edge well, and are prone to wear and tear, making all item
degradation checks with disadvantage (rolling twice and taking the lower
die). For cost, see \hyperref[upgrading-weapons]{Upgrading Weapons}.

\subsubsection*{Missile Weapons}\label{missile-weapons}

Missile weapons have a rate of fire of 1 shot per attack (except for
small thrown weapons with the marksman feat). Loading does not require
an action if ammunition is readily at hand (except for crossbows, which
require an action to load). All missile weapons have three range
brackets (Short, Medium, Far). Attacks made against targets at a given
range apply the attack penalty listed.

\begin{table}[ht]
\centering
    \begin{threeparttable}[b]
        \begin{tabular}{lcccccc}
            \toprule\
                Weapon &  Size \& Damage & Cost & Hands
                & S:0 & M:-4 & F:-8 \\
            \midrule
                Arrows/bolts x20 & M & 20d. & - & - & - & - \\
                Bow, short\tnote{*} & M & 35d. & 2 & 45' & 90' & 180' \\
                Bow, long\tnote{*} (req 9 Str) & M & 60d. & 2 & 90' & 180' & 360' \\
                Crossbow, light\tnote{**} & M & 50d. & 2 & 45' & 90' & 180' \\
                Crossbow, heavy\tnote{**} & L & 75d. & 2 & 90' & 180' & 360' \\
                Holy Water & *** & 25d. & 1 & 10' & 20' & 30' \\
                Javelin/Spear & M\tnote{†} & 16d. & 1 & 20' & 40' & 60' \\
                Sling & S & 1d. & 1 & 45' & 90' & 180' \\
                Small thrown weapon & S\tnote{†} & 12d. & 1 & 10' & 20' & 30' \\
            \bottomrule
        \end{tabular}
        \begin{tablenotes}
           \item [*] \footnotesize {If fired indoors, halve all ranges.}
           \item [**] \footnotesize {+2 attack bonus, reload requires action.}
           \item [***] \footnotesize {See costed equipment list above for details.}
           \item [†] \footnotesize {Plus user's Str modifier.}
        \end{tablenotes}
    \end{threeparttable}
\end{table}

\subsection*{Animals \& Mounts}\label{animals-mounts}

\begin{table}[ht]
    \begin{minipage}{.5\linewidth}
        \centering
        \begin{tabular}{lcccc}
            \toprule
                Animal & Cost & Size & Combat Speed & Item Slot Limit \\
            \midrule
                Donkey/Pony & 70d. & L & 50 & 20 \\
                Dog, hunting & 35d. & M & 50 & - \\
                Dog, war & 65d. & M & 50 & - \\
                Hawk & 40d. & S & 80 & - \\
                Horse, draft & 150d. & L & 50 & 30 \\
                Horse, riding & 100d. & L & 80 & 20 \\
                Horse, war & 300d. & L & 65 & 25 \\
                Mule & 90d. & L & 65 & 25 \\
                Ox & 120d. & L & 50 & 35 \\
            \bottomrule
        \end{tabular}
    \end{minipage}%
    \begin{minipage}{.5\linewidth}
        \flushright
        \begin{tabular}{lc}
            \toprule
                Item & Cost \\
            \midrule
                Cart & 60d. \\
                Feed, per day & 1d. \\
                Wagon & 800d. \\
            \bottomrule
        \end{tabular}
    \end{minipage}
\end{table}

\begin{multicols}{2}

\footnotesize{
	\textbf{Cart}: Open, road-bound, two-wheeled vehicle. Pulled by 1-2
	beasts of burden. Capacity: 2x of the animals drawing it.
	
	\textbf{Encumbrance}: A rider counts against a mount's item slot limit
	at a rate of 3 slots per the rider's size level, starting at Tiny (so 9
	slots for a Medium rider), plus the rider's own carried slots, if any. A
	mount can carry no more than two typical riders. An animal, cart, or
	wagon 1 point over its item limit gains one encumbrance level. Every 3
	slots past that adds another (or every 8 points past for a cart or
	wagon).
	
	\textbf{Wagon}: Open, four-wheeled, road-bound vehicle for heavy loads.
	Pulled by 4-6 beasts of burden. Capacity: 2x of the animals drawing it.
	
	\textbf{War}: An animal not trained for combat may panic in battle. If
	it's wounded, its owner must make a saving throw against a target of 14
	to keep it from fleeing or tossing its rider.
	
	\emph{For water vehicles, see \hyperref[ships]{ships}.}
}

\end{multicols}

\newpage

\section{Checks \& Saves}\label{checks-saves}

\textbf{Checks} are a single roll against a target number, used to
resolve situations with interesting stakes that would either be too
tedious or difficult to describe, or involve a strong element of chance.
Even if it comes down to a die roll, players will be rewarded for
thinking the action through, and sometimes penalized for not. For
example, if a player thinks to first apply some lamp oil to the chain
and gearing responsible for raising a stuck gate, this would reduce the
difficulty. Some checks are made in secret by the referee in cases where
success or failure is not obvious to the character (like Listen, Search,
or Stealth).

\bigskip

For \textbf{Ability Checks}, the player rolls 1d6 and adds the modifier
of the ability being tested, along with any situational modifiers. If
the result equals or exceeds a target number of 4, the check succeeds. A
roll of a natural 1 always fails, and a roll of a natural 6 always
succeeds, irrespective of modifiers.

\bigskip

For \textbf{Skill Checks}, the player rolls 1d6 and adds any situational
modifiers. If the result equals or exceeds the character's Skill Target
for the skill being tested, the check succeeds. Skills default to a
target of 6, unless the character has a Skill improvement that indicates
a lower target. A roll of natural 1 always fails, and a roll of a
natural 6 always succeeds, irrespective of modifiers.

\bigskip

\textbf{Saving throws} or \textbf{saves} represent an attempt to resist
a notable threat, such as magic, poison, or disease. A creature can
always choose to fail a saving throw. The player rolls 1d20 and adds the
modifier of any relevant ability score or situational modifiers. On a 14
or higher, the save succeeds; the exception is saves for instant-death
effects (known as ``death saves''); their default save is 11. A natural
20 always saves, and a natural 1 always fails. See also
\hyperref[item-saving-throws]{item saving throws}.

\bigskip

\textbf{Save Bonuses}: Characters gain a +1 save bonus for every 2 full
Hit Dice they have.

\vfill

\begin{figure}[b]
	\centering
	\includegraphics[width=1\linewidth]{img/adventurers.jpg}
	\label{fig:adventurers}
\end{figure}

\newpage

\section{Combat}\label{combat}


\subsection*{Surprise}\label{surprise}

Each party unaware of the other's presence rolls to see if they're
surprised. One player rolls for the party as a whole, and the referee
rolls for each other side.

The default chance of surprise is 4 in 12 (i.e., on a die roll of 4 or
lower). Each roll applies to all the individuals on each side of a
conflict; individuals may, however, adjust the die roll in their favor.

If one side is hidden or has set up an ambush, the other side has a 6 in
12 chance of being surprised.

A character with a Perception modifier subtracts that modifier from the
surprise threshold (i.e., a character with a +1 Per mod is surprised on
a 3 or lower; a character with a -1 Per mod is surprised on a 5 or
lower, etc.). A character with the Fieldcraft feat has a default chance
of being surprised of 2 in 12 while in their chosen terrain. The minimum
chance of being surprised is 1 in 12.

For example, the player party's die roll is a ``3,'' indicating the
party is surprised. However, the surprise occurs in a forest, a terrain
for which one character has a Fieldcraft feat. That character is not
surprised; while the rest of the party would be unable to act in the
first round of combat, that character can act normally.

\subsubsection*{Effects of Surprise}\label{effects-of-surprise}

Surprise lasts one round. In that round, those that are surprised cannot
take any actions. Attacks against surprised targets from behind also
ignore any shield modifiers and raise the range for a critical hit by 4.
Killing surprised foes might force a Morale check.

\subsection*{Reactions}\label{reactions}

Some encounters are essentially pre-determined due to the nature of the
creatures or the encounter itself, and will end up in combat no matter
what (e.g., intelligent undead and sentinels like golems will almost
always attack).

However, when any creatures are encountered, if their nature or the
circumstances don't automatically dictate their behavior, the referee
\emph{always} rolls to see how they react to the players \emph{before}
actions are taken.

If the players do not immediately attack, the referee rolls on the
Reaction Table, applying -2 to the roll if the creatures encountered are
Evil, and rerolling Hostile results if the creatures are Good:

\begin{table}[H]
\centering
\begin{tabular}{cc}
    \toprule
        2d6 & Behavior \\
    \midrule
        2 & Hostile \\
        3-6 & Unfriendly (unintelligent: Hostile) \\
        7-8 & Neutral/uncertain \\
        9-11 & Unthreatening \\
        12 & Actively helpful (unintelligent: Unthreatening) \\
    \bottomrule
\end{tabular}
\end{table}

\subsection*{Attacking}\label{attacking}

To attack, the attacker rolls 1d20 and adds their attack bonus and all
applicable modifiers. If the result equals or exceeds the target's
Armour Class (AC), the attack hits. The most common attack modifiers
are:

\begin{itemize}
\tightlist
\item
  Missile:

  \begin{itemize}
  \tightlist
  \item
    Dexterity modifier (does not apply to damage)
  \item
    Range: Short: no modifier, Medium: -1, Long: -2. Beyond Long range:
    impossible.
  \item
    Target is prone: -2
  \item
    Low visibility (gloom, smoke, fog, etc.): -2
  \item
    Target has cover: -2 (half cover) or -4 (heavy cover)
  \item
    Firing from a moving or unsteady position: -4
  \end{itemize}
\item
  Melee

  \begin{itemize}
  \tightlist
  \item
    Strength modifier (also applies to damage)
  \item
    Dual wielding: +1
  \item
    Target is prone: +2
  \end{itemize}
\item
  Attacker is striking from the rear: +2 and no shield bonus to AC
\item
  Target is surprised or attacker is invisible: +4 (replaces above)
\item
  Attacker is on a mount, target is upright: +1
\item
  Attacker cannot see target: -4
\item
  Attacker is fatigued: -2 (light) or -4 (heavy)
\item
  Attacker is using improvised weapon: -2
\end{itemize}

For PCs, a natural 20 is always a hit. The attack is also a critical
hit, unless the attacker could only hit by rolling a natural 20. Melee
and short-range missile attacks against sleeping, paralyzed, willing,
and similar targets always hit and deal maximum damage. Such targets
include PCs at 0hp.

\paragraph{Damage}\label{damage}

An attacker's Strength modifier is applied to melee and thrown weapon
attack damage. Modifiers cannot drop a successful attack below 1 point
of damage. A successful charge or set vs.~charge adds one weapon die to
the attack. Attacks against helpless targets automatically deal maximum
damage.

A character who takes more than 11 hit points of damage in a single
attack is dealt a serious \textbf{wound}, and will bleed an additional
1hp of damage per round at the end of each melee phase. This scales
linearly: 22 damage taken at once will bleed 2hp per round, etc. A bleed
can be stopped with magical healing or using an action to bandage the
wound (which requires bandages).

An \textbf{unarmed strike} deals 1d2 damage (a critical hit does 4
damage), plus the attacker's Strength modifier. Weapon dice do not
apply.

If a PC rolls a natural 20 on an attack, a \textbf{critical hit} is
scored: the attack's weapon dice are treated as having rolled their
maximum possible damage. However, you cannot score a critical hit if you
can only hit by rolling a 20. The critical hit range is increased by 4
for all attacks from behind against a surprised target.

\subsection*{Combat Phases}\label{combat-phases}

Each combat round is 10 seconds and has eight steps, taken in the
following order:

\begin{itemize}
\tightlist
\item
  \textbf{Declarations:} Dodging/parrying \& guarding. Declaring spells,
  charging \& fleeing.
\item
  \textbf{Missile:} Combatants with equipped missile weapons can attack.
\item
  \textbf{Initiative:} 1d6 is rolled by the referee: On a 1-3 the
  monsters' side moves first. On a 4-6 the players' side moves first.
  Only governs the movement phase; other phases happen simultaneously.
\item
  \textbf{Movement:} Characters not locked in melee can move.
\item
  \textbf{Melee:} Melee attacks and other actions
\item
  \textbf{Defensive Movement:} Characters locked in melee who haven't
  yet moved can move.
\item
  \textbf{Magic:} Spells go off in order of spell Rank.
\item
  \textbf{Morale:} 2d6 for NPCs and monsters
\end{itemize}

A character can move and take one action. Any action can be delayed
until the end of the combat round. One Medium or Large weapon can be
freely drawn or stowed each round; a character can draw two Small
weapons simultaneously.

\subsubsection*{Declaration Phase}\label{declaration-phase}

A character charging or fleeing must declare their intent to do so at
this time, and a character casting a spell must pick the exact spell at
this time. A fleeing character may take no actions.

After declarations, certain actions can be taken this phase:

\begin{itemize}
\tightlist
\item
  \textbf{Parry/Dodge:} Gain +1 AC or +Str mod AC or +Dex mod AC,
  whichever is higher.
\item
  \textbf{Guard:} Defend an adjacent ally, granting them +2 AC. A
  character cannot benefit from multiple Guards. Ends if either
  character moves away from the other, though both characters can move
  together.
\item
  \textbf{Brace vs.~Charge:} Only possible with a pike, spiked polearm,
  or spear, and if not prone or engaged in melee. A set combatant
  attacks first against the first charge made against them and adds one
  weapon die if the attack hits.
\end{itemize}

\subsubsection*{Missile Phase}\label{missile-phase}

Missile attacks can be made if you are not locked in melee.

Actions that can be taken this phase:

\begin{itemize}
\tightlist
\item
  \textbf{Attack:} Attack with a missile weapon. A second attack this
  round can only be made if both attacks are made with small weapons,
  and it incurs a -4 penalty to hit. A Marksman character can ignore
  this penalty.
\item
  \textbf{Reload:} Reload a crossbow.
\end{itemize}

Other considerations:

\begin{itemize}
\tightlist
\item
  \textbf{Crossbows} can be loosed while kneeling or prone.
\item
  \textbf{Firing into Melee}: The attack roll is penalized by -2 for
  each combatant in melee with the intended target (to a maximum penalty
  of -6). A character with the Marksman feat ignores this penalty. On a
  natural 1 attack roll, the missile hits a randomly determined
  combatant in melee with the intended target, inflicting damage.
\item
  \textbf{Poor Conditions}: If in very windy conditions, or if the
  target is above the attacker, long-range missile attacks cannot be
  made. If both, medium-range missile attacks also cannot be made.
\item
  A \textbf{splash weapon} may be thrown on a hard surface instead of at
  a creature; this is considered a missile attack vs AC 10. On a
  success, all creatures within 5' of the targeted surface are splashed
  with the liquid and, if it's harmful to them, suffer 1/4 normal damage
  (rounded up).
\item
  When a \textbf{splash weapon misses}, the container smashes 10' from
  the intended target in a random direction. Creatures within 5' of this
  point are splashed with the liquid and, if it is harmful to them,
  suffer 1/4 normal damage (rounded up).
\end{itemize}

\subsubsection*{Initiative Phase}\label{initiative-phase}

1d6 is rolled by the referee: On a 1-3 the monsters' side moves first.
On a 4-6 the players' side moves first.

\subsubsection*{Movement Phase}\label{movement-phase}

\begin{wrapfigure}[18]{r}{0.33\textwidth}
	\flushright
	\includegraphics[width=1\linewidth]{img/assassin.jpg}
	\label{fig:assassin}
\end{wrapfigure}

Each combatant not locked in melee can move up to its combat speed,
starting with the side that won initiative.

Actions that can be taken this phase:

\begin{itemize}
\tightlist
\item
  \textbf{Charge:} Move at least 20' in a straight line and attack,
  gaining +1 to hit and one additional damage die on a hit, but a -1 to
  AC for the round. This uses both your action and movement for the
  round.
\item
  \textbf{Drop Prone/Stand Up}
\item
  \textbf{Take Cover:} Hunker down behind cover for ``half cover'' (-2
  penalty on incoming missile attacks). If this action is taken again
  and it makes sense, ``full cover'' is granted (-4 penalty on incoming
  missile attacks).
\item
  \textbf{Hide:} Use cover or concealment to become hidden. Requires a
  successful Stealth check with a target of 6. Feats and abilities may
  lower this target.
\item
  \textbf{Something cool:} Take a movement-related action of some sort
  not on the list.
\end{itemize}

Other considerations:

\begin{itemize}
\tightlist
\item
  A character can \textbf{move through creatures} at that allow it and
  aren't in a tight formation. This costs 10' of movement per 5' of
  creatures moved through.
\item
  \textbf{Difficult Terrain} applies a x2 movement cost penalty.
\item
  \textbf{Holding Movement}: A combatant on the side that won initiative
  can choose to move after the enemy in the round.
\item
  A combatant can move half speed while dragging a \textbf{grappled}
  opponent or while \textbf{sneaking} to remain hidden.
\item
  A combatant can move 5' per round \textbf{crawling} while prone.
\item
  A character can \textbf{jump} up to 5' wide with a 10' run-up. For
  longer jumps of up to 10', a Daunting (11+) Strength check is
  required. This ends the combatant's movement for the round.
\end{itemize}

\paragraph{Locked in Melee}\label{locked-in-melee}

A creature within 5 feet of one or more enemies (three or more if it has
the Whirlwind feat) is \textbf{locked} in melee with those enemies.
Combatants locked in melee cannot use missile weapons or leave their
location until the defensive movement phase.

\begin{itemize}
\tightlist
\item
  \textbf{Blind} creatures can only lock opponents in melee if
  surrounding them.
\item
  \textbf{Flight and Teleportation} allow one to leave melee without
  penalty, even if surrounded.
\item
  \textbf{Invisible} creatures are not locked unless their melee
  opponents can see them or otherwise fully ignore invisibility, or are
  surrounding them.
\item
  \textbf{Prone} creatures cannot lock opponents in melee.
\item
  \textbf{Size}: A creature more than two sizes large than any of their
  melee opponents is not locked, even if surrounded.
\end{itemize}

\subsubsection*{Melee Phase}\label{melee-phase}

Actions that can be taken this phase:

\begin{itemize}
\tightlist
\item
  \textbf{Attack:} Strike with a weapon, making an attack roll to hit.
\item
  \textbf{Trip/Grapple/Push/Escape:} The attacker makes a melee attack
  roll with a -4 penalty. If successful, the defender must succeed on a
  saving throw. If the save fails, the defender is knocked prone, held
  by the attacker (who must have a free hand to do so), or pushed 5',
  respectively. Escape as an action allows for a saving throw to escape
  being grappled or immobilized. Special attacks like these happen
  before regular weapon attacks.
\item
  \textbf{Use Item:} Use a magic item, drink a potion, etc.
\item
  \textbf{First Aid:} Attempt to bind a wound or stabilize a dying
  companion.
\item
  \textbf{Something cool:} Take an action of some sort not on the list.
\end{itemize}

\begin{wrapfigure}[12]{l}{0.30\textwidth}
	\flushleft
	\includegraphics[width=1\linewidth]{img/warriors.jpg}
	\label{fig:warriors}
\end{wrapfigure}

\paragraph{}

Other considerations:

\begin{itemize}
\tightlist
\item
  \textbf{Polearms} (spears, pikes, etc.) in the second rank of a battle
  formation can melee attack by reaching through the first rank.
\item
  \textbf{Spacing \& the Second Rank:} Only daggers, shortswords,
  spears, and polearms can be used three-abreast in a 10' area. All
  other one-handed weapons require five feet of room (two-abreast in a
  10' area), and non-thrusting two-handed weapons require a full 10'
  space to wield.
\item
  \textbf{Attacks vs Fleeing Enemies} gain a +2 bonus to hit and ignore
  the defender's shield bonus to AC.
\end{itemize}



\subsubsection*{Defensive Movement}\label{defensive-movement}

Each combatant that is locked in melee and has not yet moved this round
can move up to half its movement speed, starting with the side that won
initiative. A character that declared its intent to flee must do so now,
moving its full movement speed.

\newpage

\subsubsection*{Magic Phase}\label{magic-phase}

If a caster wants to cancel a spell they're casting that hasn't been
disrupted by having been attacked or otherwise interrupted, they must do
so before any spells are resolved. The spell is not lost.

After spells are canceled, spells still being cast are revealed and cast
in the order of their \textbf{casting time}, which is equal to their
spell Rank, unless stated otherwise. Spells with the same casting time
are cast simultaneously, unless stated otherwise. For each instance of
the Quickcast feat that they have, a mage reduces their spells' casting
times by 1. Other feats raise a spell's Rank, raising its casting time
to match. Spells cast from scrolls add 2 to their casting time.

\subsubsection*{Morale Phase}\label{morale-phase}

If a battle is going against them, combatants may decide to retreat,
flee, or surrender. Players always make this decision for their
characters, but the referee may roll to determine if monsters or other
NPCs break morale.

\textbf{Morale Checks:} Roll 2d6 for a given side. If the result exceeds
the combatant's Morale, their morale breaks and they try to flee or
surrender in the next round. Otherwise, they keep fighting.

\textbf{When to check:}

\begin{itemize}
\tightlist
\item
  The first time a combatant on that side has been killed, and when half
  the side has been killed.
\item
  For a creature encountered alone, when first harmed and when at 1/4 or
  less of its full Hit Point total.
\end{itemize}

If a side makes two successful Morale checks in an encounter, they will
fight to the death, with no further checks necessary.

\textbf{Modifiers:} Light fatigue gives -1 to Morale checks; heavy
fatigue gives -2. Other situational modifiers may apply.

\subsection*{Escaping an Encounter}\label{escaping-an-encounter}

If a side has none of its members locked in melee after the movement of
that side in either movement phase has been resolved (whether or not the
other side has moved yet), that side can choose to try to escape the
encounter.

If the enemy follows, \textbf{pursuit checks} are made. The referee
rolls 1d12 for each group of enemies with a different combat speed,
while the players roll 1d12 for their entire party. Apply any relevant
modifiers. If only some of the PCs have modifiers, only those PCs apply
them, giving them a result different from their base party result.




\textbf{Faster Side} assumes that side can use their full speed, which
isn't a given. \textbf{Hindered Senses} usually involves difficulties
using one's primary tracking sense. Hunting dogs compensate for some of
these instances.

Each individual in a pursuit has one \textbf{pursuit action} each
pursuit round. However, neither mapping nor spellcasting can occur.


\begin{itemize}
\tightlist
\item
  \textbf{Dropping Items}: Any pursued as their action can drop items to
  apply a bonus to their side's pursuit check. Dropping desirable items
  like food or treasure can be effective, but this depends on how the
  pursuers view what is dropped, how much is dropped, and how many
  pursuers there are compared to what is dropped; the referee must
  arbitrate this. If enough food is dropped \& desired, durable food
  (+2) is always less effective than fresh food (+4).
\item
  \textbf{Missile Attacks}: After each pursuit check, if the pursuit is
  not over, then missile attacks can be made by those who have not used
  their action. Assume a range of 20'.
\end{itemize}

\begin{table}[ht]
	\centering
	\begin{tabular}{lc}
		\toprule
		Fleeing Side & Modifier \\
		\midrule
		Is faster & +1 per 5' of combat speed faster \\
		Is invisible & Auto escape \\
		Drops caltrops or flaming oil & +4 \\
		Drops desirable items (food/treasure) & +2 or +4 \\
		Is lightly fatigued & -2 \\
		\bottomrule
	\end{tabular}
\end{table}

\begin{table}[ht]
	\centering
	\begin{tabular}{lc}
		\toprule
		Pursuing Side & Modifier \\
		\midrule
		Is faster & +1 per 5' of combat speed faster \\
		Has senses hindered & -4 \\
		Is lightly fatigued & -2 \\	
		\bottomrule
	\end{tabular}
\end{table}



\subsubsection*{Ending Pursuit}\label{ending-pursuit}

A side automatically wins if they are either six points of modifiers or
more ahead of all their opponents, or score a higher result than the
other side twice in a row.

If the pursuers catch the fleeing side, a new combat round begins in the
Melee Phase, with the two sides locked in melee, positioning and so on
to be determined by the referee.



\subsection*{Death}\label{death}

\begin{wrapfigure}[17]{r}{0.39\textwidth}
	\flushright
	\includegraphics[width=1\linewidth]{img/deadwarrior.jpg}
	\label{fig:deadwarrior}
\end{wrapfigure}

When a player character's hit points reach 0, the character is
unconscious and must make a Constitution 11 saving throw. On a failure,
the character dies. On a success, the character will die in 1d4 rounds,
rolled in secret by the referee, unless magically healed or aided by
another player character. Each PC may attempt to aid once, requiring a
successful First Aid skill check (default Skill Target of 6); if the
aiding PC uses proper bandages, the check's target is lowered by 1.
Other modifiers may apply.

Even after returning to 1 or more hp, the character will remain in a
coma for 1d6 turns and must rest for a minimum of one week before
resuming any sort of strenuous activity, mental or physical. Magical
healing that returns characters to consciousness does not incur these
penalties.

Characters who are slain may be raised from the dead if a Mage of
sufficient Level is available to perform the casting. Each time a
character is brought back from the dead, their Constitution score is
reduced by one point.



\newpage

\section{General Adventuring}\label{general-adventuring}

\subsection*{Climbing}\label{climbing}


No roll is required for simple climbs, like a basic rope or tree. A more
difficult climb might require a Dexterity ability check. If the check
fails, the character falls at the halfway point (see
\hyperref[falling]{falling}).

\begin{wrapfigure}[5]{r}{0.2\textwidth}
	\includegraphics[width=1\linewidth]{img/grapplinghook.jpg}
	\label{fig:grapplinghook}
\end{wrapfigure}

Base climbing speed is 5' per round. Add 5' if the climber has the
Climbing skill or is using rope (not cumulative).

The following each add a -1 penalty to the ability check: high winds,
extreme cold, smooth surfaces, slippery surfaces, each encumbrance
level. The aid of appropriate equipment--a rope and grapnel, or pitons
and a hammer--will add a +1 bonus to the ability check.

\subsection*{Doors}\label{doors}

\subsubsection*{Listening at Doors}\label{listening-at-doors}

Most doors do not block obvious, loud noises, but characters may attempt
to detect the presence of monsters waiting quietly beyond a closed door
by pressing an ear against it to listen. The referee then rolls a secret
Perception check.

Each listening attempt takes 1 Turn. Up to two characters can
simultaneously listen at a typical door, and a check is made for each.
Characters can listen at the same door as often as they wish, each
attempt requiring an additional Turn.

\subsubsection*{Spotting Secret Doors}\label{spotting-secret-doors}

Requires a Perception ability check, with the difficulty based on how
well-concealed the door is. During cautious exploration, this is a group
check made automatically by the referee in secret with a -2 penalty for
each secret door within 10'.

A character can also actively search a 10' square area for secret doors.
This takes one turn; more than one person cannot search the same area at
the same time.

Finding a secret door does not necessarily reveal how it opens.
Describing narratively how one is searching might bypass the Perception
check if the searching technique would logically reveal the door.

\subsubsection*{Stuck Doors}\label{stuck-doors}

Upon discovering that a door is stuck, a character may attempt a
Strength check to pull the door open as swiftly and quietly as possible.
Only the strongest character in the party may attempt (representing the
party's best efforts). On a failure, the party may decide to work with
crowbars or other tools to wrench the door open anyway. This takes a
Turn and will trigger a wandering monster check.

Locked doors need keys, thieves' tools and someone trained in
lockpicking, or a battering ram (or axes). Any failed attempt to open a
stuck or locked door will alert any creature(s) on the other side of the
door to the party's presence.

You should also know: Doors tend to close on their own. Iron spikes are
invaluable for keeping doors open or closed (it takes a round and some
noise to hammer a spike in).

\newpage

\subsection*{Dungeon Random Encounters}\label{dungeon-random-encounters}

Random encounters occur in almost all dungeon areas. These are typically
monsters inclined to hostility (you're in their home) and with no
treasure (it's back in their bedroom). To make a check, the referee
rolls 1d6 at the end of the appropriate turn: on a 1, an encounter
occurs.

If an area doesn't have a preset encounter rate, how often the referee
checks to see if a random encounter occurs varies, based on the nature
of the area being explored:

\begin{itemize}
\tightlist
\item
  Organized defenders, alert sounded: every turn
\item
  Organized defenders, no alert sounded: every 2 turns
\item
  No organized defenders: every 3 turns
\end{itemize}

Even if players hide somewhere (e.g.~a disused storeroom), the encounter
roll is still made. Only if the hiding place is exceptional are
encounter checks avoided.

\textbf{Noisy}: If the party makes an unusual amount of noise while
exploring (yelling, spiking a door; combats do not count), their next
encounter check is made at -4, and the party cannot surprise creatures
so encountered except via unusual means.

\subsection*{Equipment Wear \& Tear}\label{equipment-wear-tear}

Equipment can be worn down! Every object has five levels of decay: new,
used, worn, shabby, and ruined. (You can indicate this simply by adding
a (n) next to new items when you first acquire them, erasing and
replacing as things change.) Used, worn, and shabby items still function
just as well as their new counterparts; they're just more prone to
breaking at the most inopportune moments.

Objects are subject to a decay roll after:

\begin{itemize}
\tightlist
\item
  One month of normal use, in normal conditions
\item
  Three days of wilderness use
\item
  One day of dungeon use
\item
  Armour: Receiving a natural-20 hit
\item
  Weapons: Rolling a natural 1 to hit
\item
  Food: Every day, fresh food decays one level automatically, and then
  roll as usual.
\end{itemize}

Objects that are expressly protected (e.g., a scroll in a scrollcase, or
a spellbook in a metal box, etc.) are exempt. Thoughtful storage can
reduce wear \& tear!

Objects degrade as follows:

\begin{itemize}
\tightlist
\item
  A new object becomes used if a 1 in 4 is rolled. Under stress, a new
  object breaks on a 1 in 100.
\item
  A used object becomes worn on a 1 in 20. Under stress, it breaks on a
  1 in 50.
\item
  A worn object becomes shabby on a 1 in 12. Under stress, it breaks on
  a 1 in 20.
\item
  A shabby object becomes ruined on a 1 in 8. Under stress, it braks on
  a 1 in 6.
\end{itemize}

Magical items and ``sturdy'' items roll two of the appropriate dice,
only degrading if two 1s are rolled.

You can purchase objects that aren't necessarily new for less money. New
items cost 100\% of list price, used cost 80\% of list price, worn cost
60\%, and shabby cost 40\%. Likewise, you can sell such objects to
merchants for half that price: 50\% new, 40\% used, 30\% worn, and 20\%
shabby.

\subsubsection*{Repairing Items}\label{repairing-items}

Weapons and armour can be repaired by an appropriate craftsman, costing
10\% of the item's original base price to bring it from worn to used,
and 20\% of its original base price to bring it from shabby to worn
(i.e., 30\% to go from shabby to used), plus 1d6 days per step of
repair.

\subsubsection*{Alternative System}\label{alternative-system}

Simply track when something breaks, without tracking the intermediate
steps. At the above intervals, instead of making a regular decay roll,
roll 2d6. If two 1s are rolled, the item is ruined or broken, and must
be replaced. Magic items and ``sturdy'' items roll 1d100 + 1d6, and
break if a 1 is showing on both dice (i.e. a 00, 1, and 1). Using this
alternative system is simpler, but tracking more minutely has the
benefit of knowing when things are getting close to breaking, plus a
slight mathematical edge (tracking intervals gives you on average 44
checks until breakage, vs 36 simplified, or 624 vs 600 checks on a magic
or sturdy item).

\subsection*{Experience \& Leveling}\label{experience-leveling}

You gain experience in three ways:

\begin{enumerate}
\def\labelenumi{\arabic{enumi}.}
\tightlist
\item
  You gain experience primarily through recovering treasure: 1f.
  recovered = 1XP (so 4XP per 1d. recovered).
\item
  You gain experience for killing monsters.
\item
  You gain experience by performing Feats of Exploration.
\end{enumerate}

Experience gained from treasure and killing monsters is divided equally
among all characters of the party, including retainers (who then apply a
50\% penalty before adding it to their totals). Experience gained from
Feats of Exploration is divided equally among player characters only.

\subsubsection*{Feats of Exploration}\label{feats-of-exploration}

Each feat performed grants 2\% of the Total XP Needed by the party to
advance from the start of their current Level to the start of their next
Level. For example, the Total XP Needed of a party of four 2nd-Level
characters is 8,000 (4,000 for 3rd Level minus 2,000 for 2nd Level,
times four); each Feat performed will grant 160 XP to the party.

Feats include:

\begin{itemize}
\tightlist
\item
  Complete a quest
\item
  Enter and rest in a new settlement
\item
  Enter an unexplored hex for the first time
\item
  Explore and map five rooms of a dungeon
\item
  Discover and interact with all features of a single hex (grants double
  reward)
\end{itemize}

\newpage

\subsubsection*{XP Advancement Table}\label{xp-advancement-table}

The table below indicates what a character's XP total needs to reach to
advance to each Level:

\begin{longtable}[]{@{}cccc@{}}
\toprule\noalign{}
Level & Total XP & Warrior Hit Dice & Mage Hit Dice \\
\midrule\noalign{}
\endhead
\bottomrule\noalign{}
\endlastfoot
1 & 0 & 1d8 & 1d6 \\
2 & 2,000 & 2d8 & 2d6 \\
3 & 4,000 & 3d8 & 3d6 \\
4 & 8,000 & 4d8 & 4d6 \\
5 & 16,000 & 5d8 & 5d6 \\
6 & 32,000 & 6d8 & 6d6 \\
7 & 64,000 & 7d8 & 7d6 \\
8 & 128,000 & 8d8 & 8d6 \\
9 & 260,000 & 9d8 & 9d6 \\
10 & 380,000 & 9d8+2 & 9d6+2 \\
\end{longtable}

Each Level beyond 10 requires an additional 120,000 XP and adds only a
flat 2HP per Level, with no Con modifier being applied.

\subsection*{Falling}\label{falling}

Damage from falling is determined as follows: Falls of less than 5 ft do
no damage in game terms, falls of up to 10 ft cause 1d6 damage, falls of
up to 20 ft cause 3d6 damage, falls of up to 30 ft cause 6d6, 40 ft is
10d6, 50 ft is 15d6, and falls of over 50 ft cause 20d6 points of
damage.

\subsection*{Fatigue}\label{fatigue}

Fatigue represents a serious depletion of body, mind, or spirit.


\begin{table}[h]
	\centering
	\begin{tabular}{cc}
		\toprule
			Fatigue Level & Effect \\
		\midrule
			Light & -1 to attacks, checks, and Morale checks \\
			Heavy & -2 to attacks, checks, and Morale checks, -10 Speed \\
			Exhaustion & -4 to attacks, checks, and Morale checks, halve HP, -20
			Speed \\
			\bottomrule
	\end{tabular}
\end{table}

Possession four or more levels of fatigue results in death. Fatigue
levels from different sources stack.

A character exposed to any of the causes of fatigue below gains one
fatigue level at the increments listed next to each cause (e.g.~every
full day without water applies one level).

\begin{longtable}[]{@{}cc@{}}
\toprule\noalign{}
Cause of Fatigue & Fatigue Level Increments \\
\midrule\noalign{}
\endhead
\bottomrule\noalign{}
\endlastfoot
Lack of water & Days 1/2/3/4 \\
Lack of sleep & Days 1/4/8/--* \\
Lack of food & Days 2/10/20/30 \\
\end{longtable}

*A lack of sleep can only apply up to three fatigue levels, i.e.~it
cannot kill you alone.

Per positive Con modifier point, a character can ignore one day of a
lack of water, sleep, and/or food.

If total exposure to a cause of fatigue is avoided (e.g.~a bit of water
drank, a fitful nap here or there), the current fatigue level increment
is doubled.

A character can remove one level of fatigue for every good night's rest
obtained, assuming all other causes of fatigue are also resolved.

\subsection*{Healing}\label{healing}

A character will recover 1 HP by resting overnight in a safe and
comfortable location, assuming a meal of poor quality or trail food has
been eaten. A character who dines on common food will recover 2 HP
overnight, and a character who dines on fancy food will recover 3 HP
overnight. A character will heal HP equal to their Level per day of
uninterrupted rest. Even while adventuring, a character might get a
comfortable night's rest (see \hyperref[rest-checks]{Wilderness
Exploration \textgreater{} Rest Checks}) and naturally heal HP. 30 days
of rest will return any character to full HP.

\subsection*{Item Saving Throws}\label{item-saving-throws}

Generally, only creatures make saving throws. However, some items are
especially fragile, and some rare effects (like area-effect attacks or
puddings/oozes) specifically target objects. A failed item save results
in the item's destruction, while a successful save results in no damage
or effect. If a creature must make a saving throw and it passes, no item
carried by that creature needs to make a save unless the effect
specifies otherwise.

\textbf{Fragile items} can be almost anything (not including weapons,
armour, or typical adventuring gear), but most notably include potions,
scrolls, and wands that are equipped (kept close at hand, so that
they're usable in combat without requiring an action to draw). The price
of having such useful items at hand for immediate use is that they're
vulnerable to destruction. For example, a potion in a backpack is not
available for drinking in combat without spending a round to rummage
around for it, but at the same time, if its owner is hit by a fireball,
the potion is safe. Fragile items are the exception rather than the
rule, and tend to be consumable, creating a risk to their use.

The following are examples of times when an item saving throw may be
required. The items use the character's Save Target to survive unless
specified otherwise:

\begin{itemize}
\tightlist
\item
  \textbf{Area-effect Attacks}: Affect any item that's both fragile and
  equipped.
\item
  \textbf{Bashing Containers}: Potions in a container have a Save Target
  of 14, while scrolls, wands, and gemstones have a Save Target of 11.
\item
  \textbf{Disintegration}: Affects all equipped items, fragile or not.
  Save Target of 17, or 14 if the item is magical.
\item
  \textbf{Falls}: Affect carried potions (equipped or not), requires a
  fall of at least 20' onto a hard surface. -1 penalty to the roll for
  each additional full 10' fallen over 20'.
\item
  \textbf{Magic Items}: Any that provide magic attack, save, or AC
  bonuses apply this modifier to any save such items are forced to make.
\item
  \textbf{Shields}: If used to provide a save bonus against a breath
  weapon attack, and the shield's wielder fails their save, the shield
  must save as well.
\item
  \textbf{Water}: Affects paper and parchment. Save Target 20. Standard
  spellbooks (with vellum pages and magical inks) and scrolls in scroll
  cases always pass their save.
\end{itemize}

\subsection*{Jumping}\label{jumping}

\textbf{Long jumps}: Characters can jump across a stream, chasm, or pit
of up to 5' wide with a 10' run-up--no roll is required. For longer
jumps of up to 10', a Strength check is required. In combat, a jump ends
your move.

\textbf{High jumps}: Characters can jump over obstacles of up to 3' high
with a 10' run-up--no roll is required. For higher jumps of up to 5', a
Strength check is required.

\textbf{Modifiers}: +1 if the character has the Jumping skill, -1 if
wearing Medium armour, -2 if wearing Heavy armour.

\subsection*{Light \& Vision}\label{light-vision}

Wise characters carry illumination--magical or mundane--when exploring
at night or underground. Typical light sources enable normal vision
within a particular radius. A party needs one light source for
approximately every three members of the party.

Note that light sources can be seen from much further away than the
illumination they shed for those holding them. Approaching light will
warn intelligent creatures of the approach of surface-dwellers, perhaps
giving them a chance to prepare; creatures around a corner can see a
light source whose radius projects around that corner, while two corners
between prevent its detection.

\textbf{Low light}: In low light conditions (e.g.~at night without a
light source, or if a party has insufficient light sources), characters
suffer a -2 penalty to attack rolls and move at half speed.

\begin{wrapfigure}[11]{r}{0.3\textwidth}
	\includegraphics[width=1\linewidth]{img/pickinglock.jpg}
	\label{fig:pickinglock}
\end{wrapfigure}

\textbf{Pitch darkness or blindness}: Characters in pitch darkness or
blindness suffer a -4 penalty to attack rolls, armor class, and saving
throws, and have Speed 10. A character with the Blindfighting skill
would not suffer the penalties while in combat.

\subsection*{Lockpicking}\label{lockpicking}

Characters with the proper tools and either the Lockpicking Skill or
Feat can attempt to pick locks.

A lockpicking attempt requires 1 Turn. If a character fails, they can
try again, but after two failed attempts, the character cannot try that
lock again until they gain a Level.

\subsubsection*{Trapped Locks}\label{trapped-locks}

Noticing a trap on a lock requires a successful Perception ability check
(having the Lockpicking Feat adds +1 to the check). Only an active
search for traps can reveal such traps without triggering them. In the
case of multiple traps on a single lock, the referee will roll a
separate Perception check for each trap.

A successful lockpicking attempt disarms all detected traps in addition
to opening the lock, but undetected traps will be set off automatically
before any lockpicking roll is made, unless specified otherwise.

\newpage



\subsection*{Moving Silently}\label{moving-silently}

Anyone can attempt to sneak. A roll is called for when attempting
movement that normally attracts attention, such as slipping past a
guard, or maneuvering behind a target for a surprise attack.

When attempting to sneak up on or past a creature, a Surprise Roll is
made each round, adding the sneaking party's Dexterity modifier. If the
roll indicates Surprise, the sneaking party may move for 1 Round without
being detected. Otherwise, the sneaking party is spotted. If the
Surprise Roll indicates that the sneaking character has been detected,
that character may make a Stealth Skill Check to remain undetected.

\begin{itemize}
\tightlist
\item
  \textbf{Alert Enemies}: The attention of most intelligent undead
  (e.g., skeletons, zombies) and constructs (e.g.~golems) never wavers.
  As such, sneaking checks against them incur a -1 penalty on both the
  Surprise and the Stealth rolls.
\item
  \textbf{Armour}: Wearing non-magical medium or heavy armour incurs a
  -1 penalty on both the Surprise and the Stealth rolls.
\item
  \textbf{Surprise}: Moving silently can be used to set up surprise
  attacks.
\end{itemize}

\subsection*{Perception Checks}\label{perception-checks}

Perception checks are Ability Checks always made in secret by the
referee. There's no requirement for players to constantly state that
they're doing basic investigative tasks that are repetitive and/or
obvious, like ``I'm looking at the floor'' or ``I'm listening for
noises.'' The slow pace of cautious exploration accounts for these.

However, sometimes a scenario involves something unusually subtle or a
creature has the abilities or has taken the effort to evade typical
scrutiny, and calling out a specific action might be required (e.g.,
searching a lock for poison needles before attempting to pick it, or
searching a section of wall for a secret door).

Overall, checks aren't made just to use one's eyes or ears or otherwise
notice the obvious. A Perception check only occurs when the rules call
for one (e.g. looking for traps or secret doors), or if the referee
decides one is needed.

For a group Perception check, the referee applies the group's average
Per modifier, if any, rounding normally.

\subsection*{Player Roles}\label{player-roles}

To ensure the game runs smoothly, it can be helpful to assign certain
important roles to individual players. Player roles may be assigned on a
permanent basis, if players wish, or may be rotated between sessions.

\subsubsection*{Caller}\label{caller}

The Caller serves as group spokesperson, responsible for informing the
referee about the actions and movements of the party as a whole.
Delegating this role to one player--rather than having each player
informing the referee about their PC's individual actions--can speed up
play, especially with large groups.

\textbf{Party leader}: The caller's character usually takes on the role
of party leader.

\textbf{Switching caller}: The designation of caller can change during
player. For example, the character addressing monsters in an encounter
may become the \emph{de facto} caller for that encounter.

\subsubsection*{Chronicler}\label{chronicler}

The Chronicler makes notes on the party's adventures, including monsters
and NPCs encountered, battles fought, clues discovered, and mysteries to
be unravelled. The chronicle is an invaluable tool for recalling
previous events, especially when some time has passed between sessions.
In longer campaigns, chroniclers record the collective memory of the
party's epic adventures.
\begin{wrapfigure}[14]{l}{0.3\textwidth}
	\includegraphics[width=1\linewidth]{img/cartography.jpg}
	\label{fig:cartography}
\end{wrapfigure}

\subsubsection*{Mapper}\label{mapper}

The party Mapper creates maps of the areas explored based on the
referee's descriptions. Details such as monsters or traps encountered,
clues to puzzles, or interesting unexplored areas to which the party
wishes to return may be noted on the map as it is drawn.

Maps are usually best made simply: boxes and lines are usually
sufficient to keep you from getting lost.

\textbf{Lost maps}: Maps are treated as in-game items, created by and in
the possession of a specific character (who must have a quill, ink, and
paper). Maps must be treated with great care, in order to prevent loss
or damage in case harm should come to the character carrying them; a map
actively being made is a fragile item (see
\hyperref[item-saving-throws]{item saving throws}).

\subsubsection*{Quartermaster}\label{quartermaster}

The Quartermaster keeps track of the party's accounts, tracking shared
provisions, light sources, retainers, animals, and treasure.

\subsection*{Retainers}\label{retainers}

\begin{multicols}{2}

A party can hire a number of retainers equal to 4 + (the highest-Level
party member's Level divided by 2). For example, a party with a
3rd-Level character could hire 5 retainers, but a party with a 4th-Level
character could hire 6 retainers.

Potential retainers may be located by frequenting inns and pubs or by
paying to post notices of help wanted.

\textbf{Frequenting inns and pubs:} Spending a night buying rounds and
greasing palms in drinking establishments costs 50d; . Per attempt,
there is a 3-in-6 chance of successfully locating applicants.

\textbf{Posting notices of help wanted:} Spending a day posting notices
in public places costs 25d, this cost can be doubled to request a
certain class or abilities. Per attempt, there is a 2-in-6 chance of
successfully locating applicants, who apply at the specified location
1d4 days later.

\textbf{Specific Requests}: Requesting a certain class or ability
increases the cost of recruitment in both time and money. Each
additional class, ability, known spell, or skill requested doubles the
recruitment cost and time taken to find applicants. For example, to post
a help wanted notice for an adventurer who can cast Cure Light Wounds
and pick locks would cost 100d., and it would take 4d4 days before an
applicant shows up who fits the requirements.

When a general search succeeds, roll on the following table:

\paragraph{}

\begin{table}[H]
	\centering
	\begin{tabular}[]{@{}lcc@{}}
		\toprule\noalign{}
			Settlement Size & Townsfolk & Adventurers \\
		\midrule\noalign{}
			Hamlet & 1d2 & -- \\
			Village (VI) & 1d4 & 1 \\
			Small Town (IV-V) & 1d6 & 1d3 \\
			Large Town (III) & 2d4 & 1d4 \\
			City (I-II) & 2d6 & 1d6 \\
		\bottomrule\noalign{}
	\end{tabular}
\end{table}

Townsfolk are everyday folk without a Class willing to join an
adventuring party. Usually used as torchbearers or porters, they come
with no equipment and have 1d4 HP and a Loyalty of 6.

Adventurers are individual adventurers of a specific Class willing to
join an adventuring party. They will be Level 1d3, but cannot be higher
than the hiring PC's Level. They start at a Loyalty of 7 and come with
the same basic adventuring gear as PCs, but no starting silver.
Equipment purchased by the hiring PC is kept by the PC. No skills are
included unless paid for as part of the recruitment cost (but can be
chosen if a player ends up playing the retainer as a replacement for a
lost PC, for example).

Once an applicant has been located, the hiring PC must explain what the
job entails and offer a certain wage. Townsfolk expect a daily rate of
pay, depending on the job being done (see
\hyperref[hirelings]{hirelings}) while adventurers ask for a share of
any treasure recovered, usually a half share.

The applicant's reaction to the offer is determined by rolling on the
Offer Reaction Table below--applying modifiers based on party
reputation, the mission description, or lavish or miserly rates of pay
if desired:

\begin{figure}[H]
	\centering
	\includegraphics[width=.5\columnwidth]{img/knifefighter.jpg}
	\label{fig:knifefighter}
\end{figure}


\newcolumn

\begin{table}[H]
	\centering
		\begin{tabular}{cc}
			\toprule
				2d6 & Reaction to Offer \\
			\midrule
				2 & Hostilely declines* \\
				3-5 & Declines \\
				6-11 & Accepts \\
				12 & Eagerly accepts** \\
			\bottomrule
		\end{tabular}
\end{table}

\footnotesize{* The reaction is so bad that the applicant spreads negative rumors about the PC and/or party, resulting in a -2 on further hiring rolls on this table if the PC and/or party attempt further recruitment in this
area.

** Permanent +1 bonus to the retainer's Loyalty.}

\paragraph{}

\normalsize Loyalty is checked with a Morale roll after each adventure, if the
retainer is reduced to 1/4 or less of its full HP total, or if their
loyalty is severely tested. If a success is rolled, the retainer's
Morale increases by one to a maximum of 10 (though a retainer who
eagerly accepted employment can go up to Loyalty 11). On a failure, the
retainer departs. If a retainer dies while in your service, your
character is expected to pay the retainer's family a weregild of 100d.
or its equivalent.

Starting armour is based on class. Mages start with no armour;
battlemages start with Leather armour; Warriors roll 1d6:

\begin{itemize}
\tightlist
\item
  1-2=Leather (AC 12)
\item
  3-4=Ring (AC 13)
\item
  5=Scale (AC 14)
\item
  6=Chainmail (AC 15)
\end{itemize}

Roll 1d4 to determine the retainer's starting weapon:

\begin{enumerate}
\def\labelenumi{\arabic{enumi}.}
\tightlist
\item
  One medium melee weapon \& a shield
\item
  One large weapon
\item
  Two small one-handed weapons (+1 to hit)
\item
  One medium melee weapon \& one ranged weapon (plus ammo)
\end{enumerate}

Roll 1d6 on the list of \hyperref[quick-packs]{quick packs} to determine
the retainer's starting equipment.

\end{multicols}

\newpage

\subsection*{Sailing}\label{sailing}

Sea travel usually involves an individually owned vessel or a charter.
Scheduled services are rarely available. Passengers must make inquiries
at harborside, and pay for passage on a ship heading to their intended
destination, or perhaps to some port along the way.

For game purposes, most trade occurs between important ports, separated
by one week of sailing time. Ships typically spend one week in port
off-loading and on-loading cargo, finding passengers, and perhaps
engaging in a little recreation. Sailing to an intended destination is
called a \emph{voyage}, and each trip to a port on the way is called a
\emph{leg}. A voyage may have only one leg, or it may have several.

\textbf{Passengers}: While cargo is another matter entirely, there are
two types of passengers: deck passengers and cabin passengers.

\emph{Deck passengers} pay a 20d. fee per leg to sleep on the deck with
the rest of the crew. They provide their own food, or buy it from the
crew. The wealthy can travel as \emph{cabin passengers} for 200d. per
leg, and take a cabin. Most ships have one or perhaps two cabins, for
the captain and the windcaster. Larger ships may have additional cabins
for the owner and their family to use if aboard, and cabin passengers
usually use these extra cabins for their accommodation. They also must
provide their own food. \emph{Working passage} can be secured if the
captain has a crew shortage; instead of paying wages, the captain
provides passage. This usually does not continue for more than three
legs of a voyage, or the individual is considered to be hired for
standard salary.

\textbf{Ship Crew}: A merchant ship will have both an owner and a
captain (who may or may not be the same person), a helmsman, navigator,
merchants on board, sailors, and rowers. Many positions can be taken by
the same person.

\textbf{Navigation}: Compasses and astrolabes help sailors find their
way, but advanced tools like sextants do not exist. Navigators try to
keep the coast in sight at all times, looking out for landmarks along
the route. At night, the ship pulls up on a beach or into a bay where it
can shelter for the night, the crew camping on the beach. When storms
threaten, ships likewise head for the safety of land. Some of the larger
ports have lighthouses, guiding ships toward them rather than warning of
dangerous reefs. Crossing the deep sea is a dangerous thing--here be
monsters, and the tales are all true.

Once a ship has left port, it will be at sea for usually a week. The
navigator makes a roll to successfully chart a safe course for the
vessel, creating a dice result that the helmsman may have to then
overcome with his pilot skill.

Navigator's Result: 1d6 + navigator's Navigation (Nautical) Skill Target

Next, the referee rolls 1d6 for that leg of the journey to determine any
potential hazard. If a hazard occurs, the helmsman must avoid the hazard
by rolling equal to or over the navigator's result on 2d6. He may add (6
- his Ship Piloting Skill Target) as a positive modifier.

\begin{longtable}[]{@{}cl@{}}
\toprule\noalign{}
Die & Type of Hazard \\
\midrule\noalign{}
\endhead
\bottomrule\noalign{}
\endlastfoot
1 & \textbf{Reef}. Roll for damage +1. \\
2 & \textbf{Prone to Storm}. Roll for damage. Immediate landfall. \\
3 & \textbf{Sandbank}. Roll for damage. \\
4 & \textbf{Prone to Squall}. Roll for damage -1. \\
5-6 & No hazard. \\
\end{longtable}

\textbf{Encounters at Sea}: During the ship's time at sea, it may
encounter another ship or perhaps wreckage. The referee rolls at least
once once per leg, with the rolled encounter occurring during a point in
the journey decided by the referee. If a dash is shown, then no
encounter occurs. The encounter may be routine, or it may involve
interaction, or even combat.

\begin{multicols}{2}

\begin{table}[H]
	\centering
	\begin{tabular}[]{@{}cl@{}}
		\toprule\noalign{}
			Dice & Coastal Encounters \\
		\midrule\noalign{}
			2 & Vessel in trouble \\
			3 & Pirate/marauder ship \\
			4 & Wreckage \\
			5 & Vessel with secret \\
			6 & Merchant ship \\
			7-8 & -- \\
			9 & Military patrol \\
			10 & Royal courier ship \\
			11 & Familiar vessel \\
			12 & Monster \\
		\bottomrule\noalign{}
	\end{tabular}
\end{table}

\begin{table}[H]
	\centering
	\begin{tabular}[]{@{}cl@{}}
		\toprule\noalign{}
			Dice & Deep Sea Encounters \\
		\midrule\noalign{}
			2 & Vessel in trouble \\
			3 & Merchant ship \\
			4 & Wreckage \\
			5 & Pirate/marauder ship \\
			6 & Monster \\
			7-8 & -- \\
			9 & Monster \\
			10 & Pirate/marauder ship \\
			11 & Merchant ship \\
			12 & Military patrol \\
		\bottomrule\noalign{}
	\end{tabular}
\end{table}

\end{multicols}



\subsubsection*{Maritime Trade}\label{maritime-trade}

Usually ¼ of the profits of a given voyage go to the captain, from which
he pays the crew and any outstanding expenses. The rest goes to the
owner and any partners he may have. Many merchant captains own their own
ship.

\subsubsection*{Ship Expenses}\label{ship-expenses}

\begin{multicols}{2}

Some basic running costs:

\begin{enumerate}
\def\labelenumi{\arabic{enumi}.}
\tightlist
\item
  \textbf{Supplies}: Food and water for each crewman, costing 4d./week.
  Passengers bring their own supplies.
\item
  \textbf{Wages}: Crew wages are paid every four weeks, typically at a
  port.
\item
  \textbf{Cargo Handling}: Ports charge every ship a fee for taking up
  space on the quayside, and for unloading cargos. The cost is usually
  4d./week. Merchant ships who shift their own cargo pay half this fee.
\item
  \textbf{Repairs}: Each ship strength point repaired takes one week and
  costs 20d. in materials.
\item
  \textbf{Taxes}: Vary, but usually 2\% of the flat cost of cargo or
  trade goods. A captain does not pay the tax on a contracted cargo, the
  recipient awaiting delivery does that.
\end{enumerate}

\begin{table}[H]
	\centering
	\begin{tabular}[]{@{}lc@{}}
		\toprule\noalign{}
			Position & Monthly Pay \\
		\midrule\noalign{}
			Captain & 250d. \\
			Navigator & 175d. \\
			Helmsman & 150d. \\
			Windcaster & 100d. \\
			Senior Crew & 60d. \\
			Crew & 40d. \\
		\bottomrule\noalign{}
	\end{tabular}
\end{table}

\begin{figure}[H]
	\centering
	\includegraphics[width=.6\columnwidth]{img/kraken.jpg}
	\label{fig:kraken}
\end{figure}


\end{multicols}


\newpage

\paragraph{Revenue}\label{revenue}

Ships generate revenue transporting cargo and passengers from one port
to another. The captain must inquire at the port for a week to determine
availability. Roll once for each column a maximum of once per week. The
cargo/passengers will be heading for the ship's next destination (which
must be stated beforehand).

\begin{table}[h]
	\centering
	\begin{tabular}{lccc}
		\toprule
		Type of Port & Deck Passengers 
		& Cabin Passengers & Contracted Cargoes \\
		\midrule
			Minor (V and below) & 1d6-1 & 1d6-5 & 1d6-3 \\
			Medium (III-IV) & 2d6-1d6 & 1d6-3 & 1d6-1 \\
			Major (I-II) & 3d6-1d6 & 1d6-1 & 1d6+2 \\
		\bottomrule
	\end{tabular}
\end{table}

\textbf{Cargo}: The captain chooses either a small (1d6 cwt), medium
(1d6x5 cwt) or large (2d6x10 cwt) cargo, as fits the size of his ship.
The payment for shipping this cargo is received at the destination, at
the rate of 50d. per cwt. Any losses must be made up by the
captain/owner, and the recipient at the destination unloads the cargo
and pays the tax. Most cargos will be low cost high bulk, such as grain
or wool, oil or timber. Contractors usually prefer to use well-known and
reliable vessels to transport more costly items.

\textbf{Passengers}: Cabin passengers pay well (200d. per leg) but
require a cabin for their journey. They are the captain's guests and eat
with him. Deck passengers camp out on deck and bring along or buy their
own food, paying 20d. per leg.

\textbf{Private Messages:} Ship crews are routinely approached by
civilians to carry private messages, sometimes verbally, but more often
in the form of a letter. These messages will need delivering to an
actual address or location at the destination port. On a 9 or higher on
a 2d6 roll, a crew member will be approached in this way. Small payment
may be offered, from 5d. to 25d. Rendering such service is a good way to
make friends and contacts, and perhaps find patrons.

\textbf{Speculative Trade Goods}: A designated crew member, usually a
merchant, can buy a cargo in hope of selling it for profit somewhere
else. It requires some money up front, and of course success (and
therefore profit) is not guaranteed.

\subsubsection*{Ships}\label{ships}

\textbf{Sailing Ships} (\emph{knarr}): Tubby merchantmen (50'-70' long,
10'-15' wide) who stow their cargo directly on the keel and ship's
frame. There's a deck, used by the crew for working and cooking and
sleeping. A large hatch allows cargo to be lowered down below deck.
There are a pair of steering oars at the stern, and larger vessels have
one or more cabins below the stern deck as a refuge for the captain,
owner, windcaster, and perhaps a passenger. Each has a mast, maybe two.

\textbf{Merchant Galleys} (\emph{dromond}): Long (90'-130') and narrow
(20'-30'), fast and maneuverable. Sometimes used as warships. Helmsmen
of merchant galleys receive a +1 bonus to their skill rolls. The galley
is rowed, with benches down either side and a gangway for movement.
There is no `below decks'; cargo is stacked up in the bow, at the stern,
and tied against the single mast. The advantage of the galley is its
ability to row through calm weather. Galleys have rams and can attempt
to sink opposing ships.

\textbf{Coastal Trader} (\emph{karve}): With low sides and a flat
bottom, this ship (30'-50' long, 10'-15' wide) is ideal for coastal work
and travel up rivers. It has a single mast forward, with a square sail
and a hatch for cargo. Additional cargo can be tied down to the deck.

\textbf{War Galleys} (\emph{longship}): Long (70'-100') and narrow
(10'-15' wide) war galleys are used for royal courier duties,
patrolling, transport of important officials, and for battle. They have
rams at the bow, and raised platforms fore and aft from which marines
can shoot at opposing ships. Artillery pieces are fitted as standard,
and the crews are trained to fight. Most include cabins.

\begin{table}[h]
	\begin{adjustbox}{width=1\textwidth,center=\textwidth}
	\begin{threeparttable}
	\begin{tabular}{lccccc}
		\toprule\noalign{}
			Ship Type & 
			Passengers (cabin/deck) & 
			Capacity (cwt*) & Crew & 
			Strength & Cost \\
		\midrule\noalign{}
			Sailing ship, small & 0/2 & 70 & 8 sailors & 4 & 8,000d. \\
			Sailing ship, large & 1/5 & 250 & 12 sailors & 5 & 20,000d. \\
			Sailing ship, great & 3/10 & 500 & 24 sailors & 6 & 30,000d. \\
			Coastal trader, small & 0/2 & 10 & 6 sailors & 3 & 4,000d. \\
			Coastal trader, large & 0/4 & 20 & 10 sailors & 4 & 6,000d. \\
			Merchant galley, small & 0/4 & 100 & 4 sailors, 20 rowers & 3 &
			10,000d. \\
			Merchant galley, large & 1/8 & 300 & 4 sailors, 50 rowers & 4 &
			30,000d. \\
			War galley, large & 2/15** & 4 & 10 sailors, 144 rowers, 15 marines & 4
			& 40,000d. \\
			War galley, great & 4/40** & 6 & 20 sailors, 270 rowers, 40 marines & 5
			& 60,000d. \\
			War galley, colossal & 6/60** & 10 & 30 sailors, 572 rowers, 60 marines
			& 6 & 100,000d. \\
		\bottomrule\noalign{}
	\end{tabular}
	\begin{tablenotes}
		\item [*] A hundredweight (cwt) for shipping purposes is equal to 10 stone (i.e.~10 slots)
		\item [**] usually the shipboard marines
	\end{tablenotes}
	\end{threeparttable}
	\end{adjustbox}
\end{table}

\subsubsection*{Naval Combat}\label{naval-combat}

When ships encounter one another, they may end up in a confrontation
depending on the situation. This ruleset assumes two sides, with each
comprised of one or more vessels.

\subsubsection*{Damage and Repair}\label{damage-and-repair}

All ships have a hull strength rating. When a vessel is in danger of
suffering damage, either from a reef, sandbank, squall, storm, shipboard
artillery, or monster of the deep, roll 1d6. If the result is equal to
or greater than the ship's hull strength, lower ship strength by 1. When
a ship only has a strength of 2, it is taking on water or has damaged
rigging or sails, and now travels at ½ normal movement. When a ship's
strength is reduced to 0, it founders at sea and is lost, taking 2d6
turns to sink. If half the crew works to save the vessel, it might (50\%
chance) avoid this fate. This takes 1d6 turns and cannot be done if
under attack. All crew and passengers able to jump clear must roll a
Strength saving throw and hit or beat their AC to find debris which will
help them make it to shore. On a failure, usual
\hyperref[swimming]{swimming rules} apply.

Hull strength can be repaired, though the ship must be brought ashore.
It costs 100d. of spare parts, plus timbers felled from shore, to patch
the hull and bring its strength to full. The ship's carpenter must make
a Carpentry skill check with a target of 4 to finish the work in a week;
failure indicates that another week and another skill check will be
required to complete the work. The crew cannot carry out any more
repairs until those spare parts are replaced at a port.

\textbf{Fire}: When a vessel is struck by a flaming missile (or suffers
an incendiary mishap), the referee rolls 2d6. On a 5 or higher, the ship
catches fire with no immediate effect. Every turn after, he rolls again,
and each roll of 5+ results in lowering the vessel's hull strength by 1.
The crew can try to put out the fire, but it gets more difficult each
turn: on a 2d6, roll 7+ before the first fire damage roll, 8+ before the
second fire damage roll, etc. until the ship sinks or the fire is
extinguished. The difficulty is modified by -1 if all sailors are trying
to put it out, -2 if it's a galley and has stopped to allow large
numbers of rowers to help, and -1 if marines are trying to put it out.

\textbf{Pursuit \& Evasion}: To track the distance between ships (or
forces if each side has allies), a range band is used with counters
representing the two ships moving forwards and backwards along the range
band. Range band correspond to certain real-life ranges measured in
yards:

\begin{itemize}
\tightlist
\item
  Short: 0 bands -- 1-5 yards
\item
  Medium: 1-2 bands -- 6-50 yards
\item
  Long: 3-5 bands -- 51-250 yards
\item
  Very Long: 6-10 bands -- 251-500 yards
\item
  Distant: 11+ bands -- 500+ yards
\end{itemize}

\textbf{Encounter Distance}: Under usual circumstances, combatants see
each other at Distant range (1d6+10 range bands). If the ships are ever
separated by 20 range bands or more, then contact between them is lost.

\textbf{Time Unit}: Each turn of pursuit, evasion, or combat takes 1
minute. This is a ship turn.

\textbf{Favorable Winds}: One or other of the ships will have the winds
in their favor. A ship lying in ambush around a headland will always
begin combat with favorable winds; otherwise, each ship rolls a
contested 1d6 at the start of each turn to determine who has the
favorable wind.

\textbf{Movement}: A ship will move a variable number of range bands
depending on whether the wind is favorable to it, or unfavorable. On a
favorable wind, a ship will move 3 bands; on an unfavorable wind, a ship
will move 1 range band. In addition, a successful Navigation (Nautical)
Skill Check increases speed by 1 range band that turn. Galleys can
ignore an unfavorable wind by lowering the sails and rowing; oars will
move galleys at 2 range bands per turn.

\subsubsection*{Ship Combat}\label{ship-combat}

\textbf{Personal weaponry}: At Long range or shorter, missile combat can
begin between the crews of the two ships. Long range applies a -8
modifier to hit, Medium range applies a -4 modifier.

\textbf{Artillery}: Artillery weapons are ship-mounted weapons designed
to destroy sails, masts, and rigging, and to smash hull planking to sink
the enemy ship. Large ships may carry 1 artillery piece, great ships may
carry 2, and colossal ships may carry 4; their weight counts against the
ship's capacity.

\begin{itemize}
\tightlist
\item
  Catapults (60 cwt., 4 crew, 600d., 1 attack per turn) have an arm
  under tension, flinging its stone or incendiary missile in a high arc.
  Roll 2d6 and hit 8+, -4 at Very Long range, -2 at Medium range. Cannot
  fire at Short range.
\item
  Ballista (20 cwt., 3 crew, 1000d., 2 attacks per turn) is a
  high-tension metal-framed crossbow throwing spherical stones or
  grappling hooks (up to Medium range). Roll 2d6 and hit 8+, -4 at Very
  Long range, -2 at Long range. Cannot fire at Short range.
\end{itemize}

\textbf{Ramming}: Getting into Short range is not enough, the captain
must also roll 8+ on 2d6. Success indicates a ram attack, with a 1d6+2
test vs.~the opponent's hull strength. If the test is successful, the
target's hull strength lowers immediately to 0, and the two ships are
locked together. Failure indicates a mild collision, with the boats
pushing past one another. To `unhook,' a ramming galley can back-oar
with a roll of 7+, and pull away from a target vessel, allowing it to
sink.

\textbf{Oar-Shearing}: Rather than ram their ship into another, some
captains prefer to shear oars, smashing through the oars of opposing
ships to immobilize them. Once in Short range, the ship's pilot must
make a 9+ roll on 2d6. A success disables any oar-powered craft until it
can raise its mast (typically taking all available sailors 1d6+6 ship
turns).

\textbf{Grappling}: If an aggressor does not ram, it can choose to
grapple with ropes and hooks, in an attempt to pull it closer for
boarding. This requires a successful roll of 6+ on 2d6. If the roll
fails, the defending ship may attempt to open range and flee if desired,
moving one range band. The defending ship can also cut the ropes, and
will be successful on an 8+ roll on 2d6.

\textbf{Boarding}: A ship that has rammed another (successfully or
unsuccessfully) or that has pulled its target closer with grappling
hooks can send across a boarding party. Likewise, a ship that has been
rammed or grappled can send out fighting men to defend itself. Anyone
attacking from a ship that has rammed gets a single combat round of
surprise. When hand-to-hand combat is initiated, the ship turn is
dropped in favor of usual combat rounds.

\textbf{Using Fire}: Incendiaries can be shot from catapults, and each
catapult is capable of firing incendiaries up to three times in a single
battle.

\subsection*{Speculative Trade}\label{speculative-trade}

Characters can engage in speculative trade from one settlement to
another. Profits are far from guaranteed, however, and getting cargo
safely to its destination can be its own source of adventure.

Each major settlement will have categories of goods in surplus that it
exports (and which thus are bought and sold for a price lower than
average) and goods in demand that it imports (and which are thus bought
and sold for a price higher than average).

Local prices are determined by a 2d10 roll, and comparing the result to
the following table, adding 10\% to the resulting modifier if the goods
are in demand, and subtracting 10\% if the goods are in surplus, before
applying the final modifier to the goods' base price. The rolled price
modifier represents the characters' best attempts to negotiate a deal,
and are good for one week before being re-rolled.

\begin{multicols}{2}

\begin{table}[H]
	\centering
	\begin{tabular}{cc}
		\toprule
			2d10 & Price Modifier \\
		\midrule
			2 & -9\% \\
			3 & -8\% \\
			4 & -7\% \\
			5 & -6\% \\
			6 & -5\% \\
			7 & -4\% \\
			8 & -3\% \\
			9 & -2\% \\
			10 & -1\% \\
			11 & 0\% \\
			12 & +1\% \\
			13 & +2\% \\
			14 & +3\% \\
			15 & +4\% \\
			16 & +5\% \\
			17 & +6\% \\
			18 & +7\% \\
			19 & +8\% \\
			20 & +9\% \\
		\bottomrule
	\end{tabular}
\end{table}

Player characters may purchase bulk goods by hundredweight (cwt), where
1 hundredweight is 10 slots, up to their carrying capacity (or the
carrying capacity of any pack animals, carts or wagons, or ships they
may own).

\subsubsection*{Bulk Trade Loads
Available}\label{bulk-trade-loads-available}

\begin{table}[H]
	\centering
	\begin{tabular}{lc}
		\toprule
			Market Class & Loads Available \\
		\midrule
			Class I-II & 1d6+2 \\
			Class III-IV & 1d6-1 \\
			Class V & 1d6-2 \\
			Class VI-below & 1d6-3 \\
		\bottomrule
	\end{tabular}
\end{table}

\begin{figure}[H]
	\centering
	\includegraphics[width=.6\columnwidth]{img/cart.jpg}
	\label{fig:cart}
\end{figure}

\end{multicols}

\subsection*{Strongholds \& Domains}\label{strongholds-domains}

Strongholds and domains are as in OSE Advanced, with clarifications and
detailed procedures as in
\href{https://alexmooney.github.io/ACKS_SRD/Chapter07.html\#strongholds-and-domains}{ACKS}.

\subsection*{Suffocation}\label{suffocation}

A character can survive for up to 1 round (10 seconds) per point of
Constitution before suffocating to death. For example, a character with
Constitution 12 can survive without breath for at most 2 minutes. After
this time limit, the character is knocked unconscious and must make a
saving throw as per the rules about \hyperref[death]{Death}.

\subsection*{Swimming}\label{swimming}

All characters can swim, barring an unusual background. Assuming no
current, a land-dweller swims at half their combat speed, and can do so
for hours equal to 1/4 their Con score (3/4 speed and 1/2 Con with the
Swimming skill); round down.

Attempting to swim while wearing encumbered or wearing armour is
extremely dangerous. Characters must make a Strength check to avoid
going under, modified as follows:

\begin{itemize}
\tightlist
\item
  \textbf{Light armour} or \textbf{Speed 30 due to encumbrance}: No
  modifier.
\item
  \textbf{Medium armour} or \textbf{Speed 20 due to encumbrance}: -2
  penalty
\item
  \textbf{Heavy armour} or \textbf{Speed 10 due to encumbrance}: -4
  penalty
\item
  \textbf{Rough waters}: -1 or -2 penalty, as judged by the referee.
\end{itemize}

This check is repeated every hour.

\subsection*{Trap Detection}\label{trap-detection}

Spotting a trap requires a successful Perception check, usually rolled
in secret by the referee. A character can actively search a 10' square
for traps. This takes 1 turn; only one person can search a given square
at any one time. Subtle but ultimately visible traps and triggers (such
as tripwires) should be automatically detected by this; a failed search
usually does not trigger traps.

\begin{wrapfigure}[15]{l}{0.5\linewidth}
	\centering
	\includegraphics[width=1\linewidth]{img/pittrap.jpg}
	\label{fig:pittrap}
\end{wrapfigure}

How well a trap is concealed determines the difficulty. The average
concealed trap has a -2 penalty to the Perception check during cautious
exploration, or no modifier if actively being searched for.


\textbf{10-foot Poles}: Prodding ahead with one or more of these and
using cautious exploration gives a 2-in-6 chance of triggering area
traps (safely, unless the trap specifies otherwise).

\textbf{Area Traps}: Traps that are triggered by moving into a map
square (e.g.~pit traps, but not trapped doors) are area traps. During
cautious exploration, a separate group check is automatically made for
each area trap trigger within 10 feet.

\newpage

\section*{Magic \& Spells}\label{magic-spells}

See full \href{/spells.md}{arcane spell list here}, and the full
\href{/holyspells.md}{holy spell list here}.

\subsection*{Arcane Spell Slot Progression
	Table}\label{arcane-spell-slot-progression-table}

This shows the number of spells of a given spell Rank that a caster can
\hyperref[casting-preparing]{prepare} per day (also known as their
spell slots). Mages are the first number, Arcanist Warriors are the
number in parentheses. If the mage is a specialist, they add one to each
Rank's spell slot total if they can cast spells of that Rank. If a spell
is Rank-adjusted through feats, that spell is treated in all ways as a
spell of the Rank to which it has been modified.

\begin{table}[h]
	\centering
	\begin{tabular}{ccccccc}
		\toprule
			Character Level & 1 & 2 & 3 & 4 & 5 & 6 \\
		\midrule
			1 & 1(1) & & & & & \\
			2 & 2(1) & & & & & \\
			3 & 2(1) & 1(1) & & & & \\
			4 & 2(1) & 2(1) & & & & \\
			5 & 3(1) & 2(1) & 1(1) & & & \\
			6 & 3(1) & 2(1) & 2(1) & & & \\
			7 & 3(1) & 2(1) & 2(1) & 1(1) & & \\
			8 & 3(1) & 3(1) & 2(1) & 2(1) & & \\
			9 & 4(2) & 3(1) & 2(1) & 2(1) & 1(1) & \\
			10 & 4(2) & 3(1) & 3(1) & 2(1) & 2(1) & \\
			11 & 4(2) & 3(1) & 3(1) & 2(1) & 2(1) & 1(1) \\
			12 & 4(2) & 4(2) & 3(1) & 3(1) & 2(1) & 2(1) \\
			13 & 5(2) & 4(2) & 3(1) & 3(1) & 2(1) & 2(1) \\
			14 & 5(2) & 4(2) & 4(2) & 3(1) & 3(1) & 2(1) \\
			15 & 5(2) & 4(2) & 4(2) & 3(1) & 3(1) & 2(1) \\
			16 & 5(2) & 5(2) & 4(2) & 4(2) & 3(1) & 3(1) \\
			17 & 6(3) & 5(2) & 4(2) & 4(2) & 3(1) & 3(1) \\
			18 & 6(3) & 5(2) & 5(2) & 4(2) & 4(2) & 3(1) \\
			19 & 6(3) & 5(2) & 5(2) & 4(2) & 4(2) & 3(1) \\
			20 & 6(3) & 6(3) & 5(2) & 5(2) & 4(2) & 4(2) \\
		\bottomrule
	\end{tabular}
\end{table}

\begin{figure}[H]
	\centering
	\includegraphics[width=.3\linewidth]{img/magicmissile.jpg}
	\label{fig:magicbook}
\end{figure}

\newpage

\subsection*{Holy Spell Slot Progression
	Table}\label{holy-spell-slot-progression-table}

This shows the number of holy spells of a given spell Rank that a caster
with the Anointed can prepare per day (also known as their spell slots).
Anointed Mages are the first number, Anointed Arcanist Warriors are the
number in parentheses. If the mage is a specialist, they add one to each
Rank's spell slot total if they can cast spells of that Rank.


\begin{longtable}[]{@{}cccccc@{}}
	\toprule\noalign{}
	Character Level & 1 & 2 & 3 & 4 & 5 \\
	\midrule\noalign{}
	\endhead
	\bottomrule\noalign{}
	\endlastfoot
	1 & 1(0) & & & & \\
	2 & 2(1) & & & & \\
	3 & 2(1) & 1(0) & & & \\
	4 & 2(1) & 2(1) & & & \\
	5 & 3(1) & 2(1) & 1(0) & & \\
	6 & 3(1) & 2(1) & 2(1) & & \\
	7 & 3(1) & 3(1) & 2(1) & 1(0) & \\
	8 & 3(1) & 3(1) & 2(1) & 2(1) & \\
	9 & 4(2) & 3(1) & 3(1) & 2(1) & 1(0) \\
	10 & 4(2) & 4(2) & 3(1) & 2(1) & 2(1) \\
	11 & 4(2) & 4(2) & 3(1) & 3(1) & 2(1) \\
	12 & 5(2) & 4(2) & 4(2) & 3(1) & 2(1) \\
	13 & 5(2) & 5(2) & 4(2) & 3(1) & 3(1) \\
	14 & 5(2) & 5(2) & 4(2) & 4(2) & 3(1) \\
	15 & 6(3) & 5(2) & 5(2) & 4(2) & 3(1) \\
	16 & 6(3) & 6(3) & 5(2) & 4(2) & 4(2) \\
	17 & 6(3) & 6(3) & 5(2) & 5(2) & 4(2) \\
	18 & 6(3) & 6(3) & 6(3) & 5(2) & 4(2) \\
	19 & 6(3) & 6(3) & 6(3) & 5(2) & 5(2) \\
	20 & 6(3) & 6(3) & 6(3) & 6(3) & 5(2) \\
\end{longtable}

\begin{figure}[H]
	\centering
	\includegraphics[width=.4\linewidth]{img/cleric.jpg}
	\label{fig:cleric}
\end{figure}

\subsection*{Casting \& Preparing}\label{casting-preparing}

\textbf{Preparing Spells:} Once a spell is mentally in place, it remains
until either intentionally dismantled or the caster uses the spell. If a
spell is unused, it remains in the caster's mind, even though sleep has
occurred.

To reconstruct spells (called ``memorizing'' or ``preparing'' them), a
caster must first have a night's rest (at least 4 hours of uninterrupted
sleep) and no fatigue levels from exhaustion or lack of sleep, to regain
the clarity and focus needed. The time required to prepare a spell is 10
minutes per spell Rank of uninterrupted concentration. For example, a
3rd-Rank spell would require 30 minutes to prepare. A caster can prepare
no more than two uses of the same spell.

Note that \hyperref[holy-magic]{holy magic} operates under slightly
different rules than arcane magic for spell preparation.

\textbf{Casting Spells}: Spells take two hands to cast, and include a
vocal component; a caster who is bound or gagged or otherwise restrained
or silenced cannot cast spells.

\textbf{Spell Disruption}: If, between declaring a spell and casting it,
the caster is hit by an attack or the like (even if no damage is dealt),
or fails a saving throw, the spell is disrupted: it fails and its spell
slot is emptied with no other effect. Note that only spells being cast
can be disrupted. Spell effects from a rod, staff, wand, etc. and
spell-like innate creature abilities are immune to disruption.

\subsection*{Gaining New Spells}\label{gaining-new-spells}

Mages learn one random spell each time they gain a Level from a random
school they can access (plus another spell if a specialist). If the
caster has just gained access to a new spell Rank, the spell(s) are from
that Rank. Mages also gain access to a new school of their choice at
each Name Level. When this happens, the mage learns random spells of
that school, one at each spell Rank the mage can cast.

They can also gain access to new spells via the following methods. These
methods are the only way in which an Arcanist Warrior can gain new
spells.

\begin{itemize}
	\tightlist
	\item
	\textbf{Binding}: Anyone that can read Mithric can read a spellbook or
	scroll to see what it is, but you cannot prepare and thus cast a spell
	until you have bound it to you. Only spells from schools you can
	access can be bound. Binding a spell takes 1 day + 100d. per Rank of
	the spell, and a successful Arcana ability check, applying the ability
	modifier twice, rolled at the end of the binding period; apply +1 if a
	specialist is binding a spell from their specialist school. Failure
	means the spell is permanently erased from the source scroll or
	spellbook (for this reason, it is the rare mage that will allow
	another to copy from their spellbook). Success means that the spell
	has been copied to your spellbook, and that a permanent bond between
	the caster and spell has been created: you can always write the spell
	into a spellbook or scroll, even if you don't have another written
	copy on hand to reference or have the spell prepared for casting.
	\item
	\textbf{Research}: Casters can research and create brand new spells.
	Only spells from schools the caster can access can be researched, and
	any given spell can only be attempted once per Level. Researching a
	new spell requires, on average and per Rank of the spell, 100d. in
	materials and 1 week of time.
\end{itemize}

Note that \hyperref[holy-magic]{holy magic} operates under different
rules than arcane magic for spell acquisition.


\subsection*{Holy Magic}\label{holy-magic}




Holy spells take the form of blessings bestowed upon characters by the
grace of their patrons, usually saints of the Pluritine Church.

A holy spellcaster may pray to the saints once per day, receiving their
blessings in the form of spells. Once bestowed, a blessing remains with
the character until the spell is cast. A character can only pray for
spells following a night's rest, and it takes one hour of quiet prayer.
Characters capable of casting more than one spell of a given Rank may
pray for up to two copies of the same spell.

A bestowed prayer may be cast by reciting a prayer in the Mithric
tongue. When a spell is cast, the blessing expires until it is bestowed
again. The character must be able to speak. A holy spellcaster cannot
cast spells if gagged or in an area of magical silence.

\paragraph{Praying at Shrines}\label{praying-at-shrines}

Shrines and churches dedicated to specific saints or aspects of divinity
are scattered throughout the world. A holy spellcaster who prays for one
hour at such a shrine earns the blessing of the saint in the form of an
additional specific spell. Once bestowed, this blessing remains with the
character until the spell is cast. A character may only have the
blessing of a single saint at any one time. Praying at a shrine
dedicated to a different saint causes the spell bestowed by the new
saint to replace that bestowed by the previous saint. The spell may be
of higher Rank than what the character is usually able to cast, but
subsequent blessings from the same shrine are bestowed only if the spell
is of a Rank that the character can normally memorize and if at least
one day has passed.

\begin{wrapfigure}[16]{r}{0.5\linewidth}
	\centering
	\includegraphics[width=1\linewidth]{img/runestone.jpg}
	\label{fig:runestone}
\end{wrapfigure}

Characters who can not normally cast holy spells can also pray at
shrines and receive blessings. The character must be of 3rd Level or
higher, have performed worthy deeds in service of the saint or the
church associated with the blessing, and can receive the blessing of
each saint only one time ever.

\subsection*{Lost Spellbooks}\label{lost-spellbooks}

In the event of disaster, a spellcaster who uses a spellbook and has
lost it can rewrite spells they have learned into a blank spellbook.
This requires 1 day and 100d. per Rank of spell to be written.

\subsection*{Magic Item Crafting}\label{magic-item-crafting}

Potions are created by alchemists, with the more powerful potions
usually requiring the help of a Magic-User. A Magic-User seeking to
create potions must employ an alchemist. A Magic-User must be 9th Level
to create potions on their own, and 11th Level to create other magic
items.

\newpage

\subsection*{NPC Spellcasting}\label{npc-spellcasting}

Non-player characters may be hired to cast spells or perform other
services. As a general guideline, spells cost \emph{roughly} the
following, and will be subject to a host of in-the-fiction
considerations:

\begin{longtable}[]{@{}cc@{}}
	\toprule\noalign{}
	Spell Rank & Cost per Casting \\
	\midrule\noalign{}
	\endhead
	\bottomrule\noalign{}
	\endlastfoot
	1st & 50d. \\
	2nd & 100d. \\
	3rd & 250d. \\
	4th & 500d. \\
	5th & 1,250d. \\
	6th & 2,500d.+ \\
\end{longtable}

Sages or alchemists not employed by a PC will often charge around 50d.
to identify a potion (which takes an hour or so), or 100d.+ to identify
a magical item (which can take upwards of a week).


\subsection*{Potion Creation}\label{potion-creation}

\begin{wrapfigure}[17]{l}{0.5\linewidth}
	\centering
	\includegraphics[width=1\linewidth]{img/potions.jpg}
	\label{fig:potions}
\end{wrapfigure}


Potions can be crafted by any mage of 5th Level or above, with the aid
of an alchemist, and only one potion may be made at any one time.

Potion brewing requires a stocked laboratory of at least 500d. in value.
A mage can only brew potions they have drank or own the recipe for, and
each potion costs additional silver pence per dose to brew; the referee
will have these costs.

\subsection*{The Schools of Magic}\label{the-schools-of-magic}

All spells belong to one of eight schools of magic. A caster can only
learn spells from schools to which they have access. A caster gains
access to a new school of their choice at every Name Level, learning
random spells belonging to that school, one at each Rank they can cast.
The schools are:

\begin{itemize}
	\tightlist
	\item
	\textbf{Abjuration}: Spells protective in nature. These frequently
	ward against damage or hostile effects, like gas, poison, possession,
	etc.
	\item
	\textbf{Conjuration}: Spells that summon creatures or objects, from a
	simple fog to extraplanar entities that serve your every whim.
	\item
	\textbf{Divination}: Spells that reveal information, from hidden traps
	and chambers to items and fell secrets.
	\item
	\textbf{Enchantment}: Spells that affect the minds of others and bend
	life to your will--men, monsters, and even plants--making them angry
	or ambivalent, docile or dependent.
	\item
	\textbf{Evocation}: Spells that shape raw magic itself. Most purely
	offensive spells (like \emph{Ioun Stone}, \emph{Fireball}, etc.)
	belong to this school, making it nearly mandatory for any aspiring
	battle wizard.
	\item
	\textbf{Illusion}: Spells that deceive; imaginary sounds, smells,
	objects, creatures, and even entire environments are possible.
	\item
	\textbf{Necromancy}: Spells manipulating the energies of life and
	death, covering both healing the living and interactions with the dead
	(and undead).
	\item
	\textbf{Transmutation}: Spells that alter the properties of a
	creature, object, or environment. One can fly, breathe underwater, or
	gain great strength, alter the size of a creature or object, or
	transmute one type of material into another. Sometimes called
	``Alteration.''
\end{itemize}

\subsection*{Scrolls}\label{scrolls}

Spells can be bound to scrolls; each holds one spell. Holding it with
both hands and reading from it aloud casts its spell, disintegrating the
scroll.

A spell on a scroll is not a prepared spell and can be cast even if the
caster does not own the spell or isn't able to cast spells at that
spell's Rank. The caster must still have access to the school to which
the spell belongs, however (or be a holy caster to cast holy spells from
scrolls). Spells on scrolls can't be modified through feats, though a
caster can scribe a modified spell, and spells cannot be prepared from
scrolls.

For the purposes of range and so on, a scroll spell is treated as if the
reader is casting it normally or is the minimum Level required to cast
it normally, whichever is higher.

Scrolls may be scribed by anyone who has the ability to both read
scrolls and to cast the spell being scribed. It costs 250d. and two days
per Rank of the spell, which can be broken up into multiple sessions.
This requires a successful Arcana ability check, applying the ability
modifier twice, rolled at the end of the scribing time.

\begin{figure}[hb]
	\centering
	\includegraphics[width=.4\linewidth]{img/scroll.jpg}
	\label{fig:scroll}
\end{figure}

\newpage

\section*{Upgrading Armour and
Weapons}\label{upgrading-armour-and-weapons}

\subsection*{Upgrading Basics}\label{upgrading-basics}

The process of upgrading a piece of equipment always has a number of
basic requirements regardless of its type. The following general rules
apply:

\begin{itemize}
\tightlist
\item
  Upgrades take the form of tags. Tags add new properties to equipment.
\item
  Each tag is unique and can only be applied to a piece of equipment
  once.
\item
  Upgrading a piece of equipment with a new tag can only be performed by
  a dedicated nonplayer craftsperson of the appropriate discipline in a
  workshop environment.
\item
  Upgrading a piece of equipment with a new tag is a skilled endeavor,
  taking anywhere from an hour to more than a week depending on the
  task.
\item
  On your adventures, you may come across equipment already possessing
  tags; either for sale, offered as rewards, or available through other
  means.
\end{itemize}

Due to their extraordinary artisanship and magical reinforcement, magic
items can't be modified using normal artisinal methods. As such, the
upgrades presented here can't be applied to magic items. Furthermore,
none of the upgrades presented in this supplement confer ``magic item''
status to a piece of equipment, though certain upgrades may result in a
piece of equipment being considered magical under certain circumstances.



\subsection*{Upgrading Armour}\label{upgrading-armour}

There are a number of options available to a character to upgrade their
armour, from high-cost armour proofing, to useful additions like
insulation suitable for cold weather environments. A complete list of
these options, as well as cost and any additional requirements, are
shown in the Armour Upgrades table below. In addition to the basic rules
for upgrading, the following rules apply when upgrading a suit of armour
or a shield with a new tag:

\begin{itemize}
\tightlist
\item
  Any prerequisite conditions must be satisfied to apply a new tag, as
  shown in the Armour Upgrades table.
\item
  Typically, it takes an artisan a full day of work (minimum 8 hours) to
  upgrade a suit of armour or a shield with a new tag.
\item
  The armour proofing process normally requires a full workweek (5 days)
  and can only be undertaken by a master artisan. Once added, armour
  proofing tags can't be removed from a suit of armour.
\end{itemize}

\subsubsection*{Armour Proofing}\label{armour-proofing}

The primary method of upgrading armour, proofing is the process of
testing and improving a suit of armour's ability to withstand blows of
various types below a certain threshold, being certified as proof
against swords, arrows, and warhammers. While all types of armour can be
proofed against slashing damage, only medium and heavy armours can be
proofed against piercing damage, and only heavy armour can be proofed
against bludgeoning attacks.

\subsubsection*{Other Upgrades}\label{other-upgrades}

In addition to proofing, various other upgrades are available to armours
meeting the prerequisites for each, as shown in the Armour Upgrades
table. Certain upgrades are incompatible, such as the breathable and
insulated tags, while others have specific requirements. Certain types
of armour upgrade tags may be removable or temporary; when in doubt,
consult with your referee.

\subsection*{Upgrading Weapons}\label{upgrading-weapons}

Upgrading weapons follows a branching path system, with available
options split into multiple tiers. At first, a weapon can only be
upgraded with a limited selection of 1st tier tags depending on its
type, each of which ``unlocks'' one or more options from the next tier.
The following additional rules apply when upgrading a weapon with a new
tag:

\begin{itemize}
	\tightlist
	\item
	All prerequisite conditions must be satisfied to apply a new tag, as
	shown in the Weapon Upgrades table.
	\item
	2nd tier upgrades can only be applied by a trained craftsman, with 3rd
	tier upgrades requiring the skills of a master artisan.
	\item
	Typically, it takes an artisan a full day of work (minimum 8 hours) to
	upgrade a weapon with a new tag.
	\item
	Once added, a tag can't be removed from a weapon.
\end{itemize}

\subsubsection*{Weapon Upgrade Cost
	Structure}\label{weapon-upgrade-cost-structure}
\begin{wrapfigure}[19]{r}{0.5\linewidth}
	\centering
	\includegraphics[width=1\linewidth]{img/armourmerchant.jpg}
	\label{fig:armourmerchant}
\end{wrapfigure}

Upgrading a weapon has a base cost associated with each tier, with
subsequent upgrades of that tier costing twice the previous amount for
that tier. For example, Erik decides to pay a visit to a local
blacksmith in order to have a customized pommel attached to his
dagger---adding the balanced tag---for the tier 1 base cost of 100d..
Later, he returns to have the blade honed---adding the sharpened
tag---this time costing 200d., with the next 1st tier upgrade costing
400d., and so on.



Upgrades are grouped by tier for the purposes of this cost scaling:
continuing with the above example, if Erik then wishes to have the blade
of his dagger partially serrated in order to improve its effectiveness
when slicing and cutting---adding the superior tag---it would cost
1,000d. (the 2nd tier base cost) provided that the dagger doesn't
already possess any other 2nd tier tags.



\newpage

\subsection*{Armour Upgrades}\label{armour-upgrades}

	\begin{tabular}{p{1.5in} c p{1.3in} p{2.2in} }
		\toprule
			Tag & Cost (d.) & Prerequisite & Properties \\
		\midrule
			\textbf{Armour proofing: 1st tier} & 1000d. & Light, medium, or heavy
			armour & While wearing this armour, if you would take 3 or less
			nonmagical slashing damage before resistance is applied, you take none
			of that damage instead. \\
			\textbf{Armour proofing: 2nd tier} & 2000d. & Medium or heavy armour
			with the \emph{1st tier armour proofing} tag & While wearing this
			armour, if you would take 4 or less nonmagical slashing or piercing
			damage before resistance is applied, you take none of that damage
			instead. \\
			\textbf{Armour proofing: 3rd tier} & 3000d. & Heavy armour with the
			\emph{2nd tier armour proofing} tag & While wearing this armour, if you
			would take 5 or less non- magical slashing, piercing, or bludgeoning
			damage before resistance is applied, you take none of that damage
			instead. \\
			\textbf{Breathable} & 100d. & Light or medium armour. Incompatible with
			\emph{insulated}. & While wearing this armour, you have a +2 bonus on
			saving throws against fatigue due to extreme heat. \\
			\textbf{Climbing harness} & 100d. & Light, medium, or heavy armour &
			This armour has been modified with a climbing harness around the
			midriff, comprising leather straps and quick- draws. While wearing it,
			you gain a +1 bonus to checks to scale vertical surfaces when using a
			rope or similar aid. \\
			\textbf{Decorated} & 50d. & Medium or heavy armour and shields & This
			armour or shield is adorned with a holy symbol and allows its bearer to
			cast \emph{Turn Undead} without holding a holy symbol. \\
			\textbf{Insulated} & 100d. & Incompatible with \emph{breathable}. & This
			armour counts as cold weather gear in conditions of extreme cold. \\
			\textbf{Quick-release clasps} & 200d. & Light, medium, or heavy armour &
			You can doff this light armour as an action, medium armour in 1 minute,
			and heavy armour in 5 minutes. \\
			\textbf{Spiked} & 250d. & Medium or heavy armour & This armour has been
			modified with spikes, barbs, or another similar feature, and deals 1d4
			piercing damage to creatures who hit its wearer with unarmed strikes or
			natural weapons that aren't magical. \\
		\bottomrule\noalign{}
	\end{tabular}

\subsection*{Weapon Upgrades}\label{weapon-upgrades}

\begin{tabular}{p{1.9in} p{1.3in} p{2.7in} }
	\toprule
		Tag & Prerequisite & Properties \\
	\midrule
		\emph{1st Tier (base cost: 100d.)} & &  \\
		\textbf{Balanced} & - & You gain a +1 bonus to attack rolls made with
		this weapon. \\
		\textbf{Critical: Sharpened} & Melee weapons that deal piercing or
		slashing damage only & Attacks with this weapon score a critical hit on
		a roll of one lower than normal. \\
		\textbf{Critical: Sight pin} & Bows and crossbows only & Attacks with
		this weapon score a critical hit on a roll of one lower than normal. \\
		\textbf{Critical: Spiked} & Melee weapons that deal bludgeoning damage
		only & Attacks with this weapon score a critical hit on a roll of one
		lower than normal. \\
		\textbf{Silvered} & - & Attacks with this weapon count as silvered for
		the purposes of overcoming resistance and immunity to nonmagical attacks
		and damage \\
		\textbf{Wounding: Keen} & Melee weapons only & You gain a +1 bonus to
		damage rolls made with this weapon. \\
		\textbf{Wounding: Oiled string} & Bows and crossbows only & You gain a
		+1 bonus to damage rolls made with this weapon. \\
		\emph{2nd Tier (base cost: 1,000d.)} & & \\
		\textbf{Brutal} & \emph{Sharpened} or \emph{Spiked} tag & When you
		attack with this weapon and roll the maximum result for the weapon's
		damage dice, you can roll those dice again and add the new roll to the
		damage of the attack. If you roll the maximum amount again, you can
		repeat this process until you don't. \\
		\textbf{Enchanted} & \emph{Silvered} tag. Can only be applied by a mage
		of 11th Level or higher. & Attacks with this weapon count as magical for
		the purposes of overcoming resistance and immunity to nonmagical attacks
		and damage, and features such as the black pudding's corrosive form. \\
		\textbf{Superior} & \emph{Balanced} and \emph{keen} or \emph{oiled
		string} tag. & The damage die of this weapon is increased by one size
		(for example, 1d4 becomes 1d6, 1d6 becomes 1d8, and 1d8 becomes
		1d10). \\
		\emph{3rd Tier (base cost: 10,000d.)} & & \\
		\textbf{Masterwork} & \emph{Brutal} or \emph{superior} tag & You gain an
		additional +1 bonus to attack and damage rolls made with this weapon. \\
	\bottomrule
\end{tabular}

\newpage

\section{Wilderness Exploration}\label{wilderness-exploration}

In a campaign where the journey is as important as the destination (or
where there isn't even necessarily a main destination \emph{per se}),
overland movement comes into play.

Travel overland is measured in 6-mile hexes. \textbf{Travel points} are
spent to enter a hex; player characters have travel points based on
their \hyperref[item-slots]{encumbrance}. A mounted party has speed 40
(8 travel points per day), and a party travelling with a cart or wagon
has speed 30 (6 travel points per day).

The base cost in travel points to enter a hex depends on its terrain
type. Unlisted terrain features like great rivers, canyons, etc. can
further raise the cost.

\begin{table}[ht]
	\centering
	\begin{adjustbox}{width=1\textwidth,center=\textwidth}
	\begin{tabular}{lccc}
		\toprule
		Terrain Type & 
		Entry/Search Cost & 
		Lost/Encounters & 
		Mounts \& Vehicles \\
		\midrule
		Plains, steppe, farmland & 2 & 1 in 6 & May enter \\
		Hills, woods, desert, rough & 3 & 2 in 6 & Mounts must be led, no
		vehicles \\
		Mountains, jungle, swamps & 4 & 3 in 6 & No mounts or vehicles \\
		\bottomrule
	\end{tabular}
	\end{adjustbox}
\end{table}

If the party is travelling on a road and does not leave, a hex costs 2
travel points to enter; Some roads are marked with mile markers; on
these roads, it costs 2 travel points per 6 miles traveled. Only
roadside features will be explored while travelling on a road; features
further from the road are missed.

Travel may be impeded by poor weather, reducing the party's travel
points by 2. If this reduces the party's travel points to 0 or below,
they can only progress by forced marching.

A party may choose one of two optional march types each day, reflecting
its priorities: either caution or speed.

\begin{itemize}
\tightlist
\item
  \textbf{Cautious March}: Decrease travel points by one, apply a +1
  bonus to all rolls for overland random encounters, and the party has a
  better chance of tactically favorable encounters.
\item
  \textbf{Forced March}: Gain a 50\% increase in travel points, but gain
  one fatigue level at day's end. Can be kept up until a marcher is
  heavily fatigued. A day of full rest is required to remove one level
  of forced march fatigue from most creatures.
\end{itemize}

\subsection*{Entering a Hex}\label{entering-a-hex}

If a party wants to enter a hex, but lacks some of the points needed,
they points they do have are spent towards entering it, but the party
ends the day in their current hex. The party only enters the new hex
once its full travel point cost is paid.

\subsubsection*{Navigation and Getting
Lost}\label{navigation-and-getting-lost}

If the party enters a hex they have not fully searched, the party makes
a \textbf{navigation check} by rolling 1d6. The chance of getting lost
depends on the terrain being explored (see table above).

There is no chance of getting lost if the party is following a road, and
there is only a 1-in-6 chance of getting lost if the party is following
a trail or track.

In conditions of reduced visibility, like a thick fog or blizzard, the
chance of getting lost increases by 1 (e.g., 2-in-6 becomes 3-in-6), and
in darkness by 2.

If the party has a member with the Fieldcraft feat for that hex's
terrain type, the chance of getting lost decreases by 1.

This check is skipped if the party has a respectable map or
knowledgeable guide, or if there is a road, trail, coastline, or river
in the current hex they can logically follow to their next hex.

If the check fails, the party is lost. To leave the hex, the party must
spend travel points equal to the amount they spent to enter it
originally, and then make a new navigation check. Failure means that the
party is still lost.

\subsubsection*{Searching}\label{searching}

Entering a hex allows a party to determine its features. There are two
types: \textbf{overt} and \textbf{hidden}.

Overt features require no special effort to find. Hidden features may or
may not exist, but can only be found if the party searches the hex,
moving off the beaten path to seek out points of interest there. A hex
may have both types.

Searching a hex requires spending the same number of travel points that
it cost to enter the hex, and, like entering a hex, is only complete
when the full hex point cost is paid. Such a search will reveal any
hidden features present in a hex. Once a hidden feature has been
located, it can be found again without searching.

\subsection*{Finding Food in the Wild}\label{finding-food-in-the-wild}

When on a journey, a party may try to find food in the wilds rather than
rely on rations purchased in a settlement. The party may choose to fish,
forage, or hunt.

If the party attempts to find food as they travel, make a Survival
check. If the characters devote a whole day exclusively to finding food,
without travel or rest, they gain a +2 bonus to the check. A single
check is made for the entire party.

\subsubsection*{Fishing}\label{fishing}

Using a fishing pole, hook, and something for bait, fishing is possible
in any hex which contains a lake or river. Successful fishing provides
1d10 rations.

\begin{wrapfigure}[14]{l}{0.4\linewidth}
	\centering
	\includegraphics[width=1\linewidth]{img/foraging.jpg}
	\label{fig:foraging}
\end{wrapfigure}

\paragraph{}

\subsubsection*{Foraging}\label{foraging}

Plants and mushrooms can be found in many hexes. Successful foraging
provides 1d6 fresh poor-quality rations (1d4 in winter, 1d8 in autumn).
Rare and magical herbs may also turn up in the harvest.

\subsubsection*{Hunting}\label{hunting}

Successful hunting means that the party has crept up on game animals,
and must then attempt to kill them. The party has surprise and beginning
combat 1d4 x 30' away from the quarry. A successful hunt yields fresh
common-quality rations based on the HP of game animals killed: 1 ration
per HP for small animals, 2 rations per HP for medium, and 4 rations per
HP for large.

\newpage

\subsection*{Camping in the Wilds}\label{camping-in-the-wilds}

\subsubsection*{Setting up Camp}\label{setting-up-camp}

\textbf{Preparing the Campsite:} At least one character must remain at
the campsite to clear away branches and rocks, set up tents and
bedrolls, prepare a fire pit, and so forth.

\textbf{Fetching Firewood:} Each character who goes fetching wood can
collect enough to keep a campfire burning for 1d6 hours. Situational
modifiers for the amount of wood found may be applied, for example: -1
for damp conditions, -2 in snow, -4 in heavy rain.

\subsubsection*{Fetching Water}\label{fetching-water}

Finding water to drink is assumed to have happened naturally while
traveling, except in an exceptionally dry environment, when it is only
found on a 2-in-6 chance per hex.

\subsubsection*{Building a Fire}\label{building-a-fire}

Given a means of producing flame (e.g.~a tinderbox, magic) and a stash
of wood (either gathered from the forest or carried in packs), a
character may attempt to build a fire.

\begin{wrapfigure}[11]{r}{0.3\textwidth}
	\includegraphics[width=1\linewidth]{img/campfire.jpg}
	\label{fig:campfire}
\end{wrapfigure}

\textbf{Good conditions:} In favorable conditions, with decent wood and
a relatively dry campsite, fire-building automatically succeeds.

\textbf{Bad conditions:} In more troublesome circumstances, getting a
fire going may require a Survival check with a +2 bonus, though
``Dwarves can make a fire almost anywhere out of almost anything, wind
or no wind'' and will automatically succeed. The referee may reduce the
chance of success to account for extreme cold or damp.

\subsubsection*{Cooking}\label{cooking}

Given a fire, cooking pots, and ingredients (e.g., foraged food, fresh
rations, hunted game), someone may cook a meal. The cook must make a
Survival Skill Check.

\textbf{If the check succeeds:} An especially tasty dish is produced.
All who eat the meal gain a bonus to any Con checks required to rest: +1
for meals made with common ingredients, and +2 for meals made with fancy
ingredients. (Eating trail food or food made with poor quality fresh
ingredients does not give any bonuses to Con checks required to rest.)

\textbf{If the check fails:} A palatable but not exemplary dish is
produced. On a natural 1, the cook must make a saving throw against a
target of 11 or produce a ruined and utterly inedible meal, wasting the
ingredients used.

\subsubsection*{Camaraderie}\label{camaraderie}

Time spent around the fireside with one's companions may lift the
spirits and induce restful sleep. A character may attempt to entertain
with music, song, stirring tales, jokes, and so forth. The entertainer
should make a Performance Skill Check.

\textbf{If the check succeeds:} All characters gain a +1 bonus to any
Con checks required to rest.

\textbf{If the check fails:} The attempt to entertain falls flat. On a
natural 1, the entertainer must make a saving throw against a Target of
11 or incur ridicule and discord, incurring a -1 penalty to any Con
checks required to rest.

\subsubsection*{Rest Checks}\label{rest-checks}

When camping in the wild, characters' ability to get a good night's rest
is determined by their equipment (whether they have a bedroll and/or
tent), their warmth (whether they have a fire burning), and the season.
Non-ideal circumstances require PCs to make a Constitution check, with
the difficulty listed below.

\begin{table}[ht]
	\centering

		\begin{tabular}{cccccc}
			\toprule
				Fire & Bed & Winter & Spring & Summer & Autumn \\
			\midrule
				N & No bedding & Auto failure & Difficult & Moderate & Difficult \\
				N & Bedroll or tent & Auto failure & Moderate & Good rest & Moderate \\
				N & Bedroll \& tent & Difficult & Moderate & Good rest & Moderate \\
				Y & No bedding & Auto failure & Difficult & Moderate & Difficult \\
				Y & Bedroll or tent & Difficult & Good rest & Good rest & Good rest \\
				Y & Bedroll \& tent & Moderate & Good rest & Good rest & Good rest \\
			\bottomrule
		\end{tabular}

\end{table}

\textbf{Moderate:} The character must make a Constitution check to get a
good night's rest.

\textbf{Difficult:} The character must make a Constitution check with a
-2 penalty to get a good night's rest.

\textbf{If the check succeeds:} The character gets a good night's sleep
and regains 1 HP overnight if they ate a poor meal or trail food, 2 HP
if they ate a common meal, and 3 HP if they ate a fancy meal.

\textbf{If the check fails:} The character fails to get a good night's
sleep and suffers one level of fatigue due to lack of sleep. For each
spell the character attempts to prepare, there is a 1-in-6 chance of
failure. If the roll fails, the attempt to prepare the spell fails - the
spell slot remains empty and unusable this day.



\subsection*{Weather}\label{weather}

At the start of each day, the referee rolls 2d6 on a seasonal weather
table. Some weather impedes travel (reducing the party's travel points
by 2), while other weather brings poor visibility (halving the outdoor
encounter distance and increases the chance of getting lost while
travelling off-road) or wet conditions (making it harder to start a
campfire).

\subsection*{Wilderness Random
Encounters}\label{wilderness-random-encounters}

One check for wandering monsters is made each day. The chance of an
encounter depends on the type of terrain being traversed. If a party has
one or more members with Fieldcraft for that hex's terrain type, add 1
to the roll. Wandering monsters are encountered 2d6 x 30' away. If both
sides are surprised, this is reduced to 1d4 x 30'.

If the party spends the night in the wilderness, an additional random
encounter check is made. If an encounter occurs, the referee randomly
rolls to see on whose watch (if any) the encounter takes place.

\section{Hirelings}\label{hirelings}

\subsection*{Standard Hirelings}\label{standard-hirelings}

Short-term services of simple craftsmen and laborers are relatively
easily procured, but it is harder to find individuals willing to take
service for longer than a few days, especially if considerable travel is
involved.

\begin{longtable}[]{@{}lll@{}}
\toprule\noalign{}
Hireling & Daily Rate & Monthly Rate \\
\midrule\noalign{}
\endhead
\bottomrule\noalign{}
\endlastfoot
Groom, Laborer, Linkboy, Pack Handler & 1d. & 24d. \\
Cook, Servant & 1d. 2f. & 30d. \\
Limner & 2d. & 50d. \\
Teamster & 3d. & 72d. \\
Carpenter, Mason, Leatherer, Tailor & 4d. & 100d. \\
Guide & 5d. & 130d. \\
\end{longtable}

\textbf{Carpenter}: Skilled in the working of wood, a carpenter might be
retained to construct anything from a table to a palisade. Their
expertise is also invaluable for the manufacturing of shields and
similar items.

\textbf{Cook}: Familiar with the preparation of various types of food,
and sometimes also knows a little herblore.

\textbf{Groom}: Proficient in the care of horses, an attentive groom can
usually tell a good mount from a bad; also known as an ostler or stable
hand.

\textbf{Guide}: Knows the local landmarks within their specialist region
(determined by the referee) and can lead the PCs to and from these
landmarks without the risk of getting lost. If the party does get lost
with a guide, there's a 4-in-6 chance that the guide is able to find the
path again quickly. Guides sometimes enter dangerous regions, but they
charge double for doing so.

\textbf{Laborer}: Essentially unskilled, laborers are suitable for only
the most menial sorts of work; this category includes bearers and
porters, each of which is able to carry up to 12 slots of items, or
twice that if a pole or other contrivance is utilized.

\textbf{Leatherer}: Capable of producing a wide range of leather goods,
such as packs, belts or riding gear; a leatherer is indispensable for
the making of scabbards, sheathes, shields and the other leather
components of arms and armour.

\textbf{Limner}: Adept in the painting of signs and the illumination of
heraldic devices, among other similar tasks.

\textbf{Linkboy}: Usually hired to bear a lantern or torch, a linkboy is
typically (but not always) a youth.

\textbf{Mason}: Expert in the working of stone or plaster, masons are
essential for the construction of many significant buildings and
fortifications.

\textbf{Pack Handler:} Practiced in the burdening, handling and
unburdening of various pack animals.

\textbf{Servant}: Typically serving as valets, butlers, maids,
messengers or simple lackeys, servants are expected to look to the needs
of their master.

\textbf{Tailor}: Accomplished in the repair and making of clothes or
other cloth items; the services of a tailor are also required for the
production of various types of textile based armour and coverings.

\textbf{Teamster}: Experienced drivers of carts and wagons, teamsters
are usually experts at loading and unloading their vehicles, as well as
handling the animals with which they are familiar.

\subsection*{Mercenaries/Men-at-Arms}\label{mercenariesmen-at-arms}

Mercenaries are as on pgs. 112-113 of \emph{OSE Advanced.} The majority
of regular men-at-arms are zero-Level characters with 1d4+3 hit points.
The following additional mercenaries are available (and in some cases
required):

\textbf{Captain}: Equivalent to a 5th- to 8th-Level warrior (1-4=5th,
5-7=6th, 8-9=7th, 0=8th). A captain may lead 20 men at arms and one
lieutenant per Level of experience, plus any necessary sergeants; the
monthly wage demanded by a captain is equal to his Level x 100d..

\textbf{Lieutenant}: Equivalent to a 2nd- (1-7) or 3rd- (8-0) Level
warrior. A lieutenant may lead ten men at arms per Level of experience,
plus any necessary sergeants. A lieutenant serving under a captain
extends the number of troops the captain can effectively command and
control. The monthly wage demanded by a lieutenant is equal to his Level
x 100d..

\textbf{Sergeant}: Equivalent to a 1st-Level warrior. A sergeant can
lead up to ten men independently or in service to a lieutenant or
captain. In any given company, there must be one sergeant for every five
to ten men at arms. The monthly wage required by a sergeant is ten times
that of the troop type he leads.

A player character warrior of the appropriate Level may serve as a
sergeant, lieutenant or captain, as might an allied non-player character
fighter or retainer.

\subsection*{Expert
Hirelings/Specialists}\label{expert-hirelingsspecialists}

Obtaining the services of very skilled craftsmen and other professional
servitors typically involves the expenditure of considerable time and
resources. While it is possible to retain such hirelings for short
periods, few will agree to a term of less than a month and most expect
to serve considerably longer.

\begin{table}[ht]
	\centering
	\begin{threeparttable}
	
		\begin{tabular}{lc}
			\toprule
				Specialist & Monthly Wage \\
			\midrule
			Alchemist & 500d.{*} \\
			Animal Trainer & 250d. \\
			Armourer & 250d.{*} \\
			Blacksmith & 110d. \\
			Engineer (Architect) & 250d.{*} \\
			Engineer (Artillerist) & 150d. \\
			Engineer (Miner or Sapper) & 150d. \\
			Jeweler/Gemcutter & 100d.{*} \\
			Sage & Special \\
			Scribe & 100d. \\
			Ship Crew & 60d. \\
			Ship Captain & 250d. \\
			Spy & Special \\
			Steward/Castellan & Special \\
			Weaponsmith & 150d.{*} \\
			\bottomrule
		\end{tabular}
		\begin{tablenotes}
			\item [*] Cost does not include all remuneration or special fees.
		\end{tablenotes}
	\end{threeparttable}
\end{table}

\newpage

\textbf{Alchemist}: Identify potions and substances. Based on a sample
or recipe, an alchemist can produce a potion at twice the normal speed
and for half the normal cost (see \emph{OSE Advanced, Magical Research,
p126}). An alchemist may also research new or unknown potions, but this
takes twice as long and costs twice as much as normal.

\textbf{Animal Trainer}: Specialized trainers are required for exotic
animals or larger numbers of normal animals. A trainer can have up to
six animals under their care at a time. It will take a minimum of one
uninterrupted month to teach an animal the first new behavior or trick.
After this first month, an animal has become accustomed to the trainer
and can be taught additional behaviors at twice the rate (two weeks per
behavior).

\textbf{Armourer}: Required for the production and maintenance of armour
and shields; for every 60 men at arms or barded warhorses present, there
must be at least one armourer available. Each must be provided with a
workroom, forge, and assistants at an additional cost
(\textasciitilde100d.). An armourer can use spare time (prorated based
on number of supported troops) to make additional armour, helmets, or
shields at 25\% of their usual cost. Per month, an armourer can make
three shields or one suit of armour.

\begin{wrapfigure}[13]{l}{0.3\textwidth}
	\includegraphics[width=1\linewidth]{img/tinker.jpg}
	\label{fig:tinker}
\end{wrapfigure}

\textbf{Blacksmith:} Essential for the basic maintenance of a stronghold
and any resident soldiery; for every blacksmith retained the needs of up
to one hundred and twenty men or horses can be met, but there must be at
least one in every stronghold and a workroom and forge must be provided
for each (\textasciitilde100d.). Besides the usual duties (horseshoes,
nails, hinges, etc.) a hired smith can turn out some basic weaponry each
month: 30 arrowheads or quarrel tips, or 10 spear heads, or 5
morningstars, or 2 flails or polearm heads.

\textbf{Engineer (Architect)}: Necessary for the successful construction
of any but the most simple of surface structures. An architect requires
payment by the month, even for short projects, and expects to receive an
additional sum equal to 10\% of the total building costs. Unless the
construction site was approved by an architect, there is a 75\% chance
that any structure will collapse in 1d100 months.

\textbf{Engineer (Artillerist)}: Mandatory for the construction and
correct operation of siege weapons, such as the trebuchet or ballista.
No such engines can be made or properly used without the services of
such an individual. If employment is for short term only, say a few
months or less, then rates of pay and costs will be increased from 10\%
to 60\%.

\textbf{Engineer (Miner or Sapper)}: Indispensable for the overseeing of
any mining operations, underground construction, or siege and counter
siege works that involve trenches, fortifications, assault towers and
other similar siege devices.

\textbf{Jeweler/Gemcutter}: Able to speedily and accurately appraise the
value of most gems, jewelry and other precious objects, a jeweler is
also capable of repairing, enhancing or newly creating ornamented items
and jewelry. The total value of the materials can be increased by from
10\% to 40\%, depending on the skill of the jeweler. Likewise, a
gemcutter might well increase the value of a rough or poorly cut stone
(those under 5,000d. base value), or the stone might be ruined in the
process. Note that jeweler/gemcutters cannot be held responsible for
damage. Dwarven jeweler/gemcutters add 20\% to skill level determination
rolls, but cost twice as much to employ.

\begin{table}[ht]
	\centering
	\begin{tabular}{cc}
		\toprule
			Jeweler Skill Level & \\
		\midrule\noalign{}
			01-20 & Fair---10\% increase 90\% likely \\
			21-50 & Good---20\% increase 50\% likely, +10\% otherwise \\
			51-75 & Superior---30\% increase 60\% likely, +10\% otherwise \\
			76-90 & Excellent---40\% increase 70\% likely, +10\% otherwise \\
			91-00 & Masterful---40\% increase 60\% likely, +20\% otherwise \\
		\bottomrule\noalign{}
	\end{tabular}
\end{table}

\begin{table}[ht]
	\centering
	\begin{tabular}{cc}
		\toprule
			Gemcutter Skill Level & \\
		\midrule
			01-30 & Shaky---d12, one roll, 1 improves, 10-12 ruins stone \\
			31-60 & Fair---d12, one roll, 1-2 improves, 12 ruins \\
			61-90 & Good---d12, one roll, 1-3 improves, 12 ruins \\
			91-00 & Superb---d20, 1-5 improves, 20 ruins stone \\
		\bottomrule\noalign{}
\end{tabular}
\end{table}

\newpage

\textbf{Sage}: A person with a degree of knowledge on just about
everything, a lot of knowledge in a few specific fields, and
authoritative knowledge in his or her special fields of study. Each sage
specializes in one or more minor fields of study, and a handful of
special categories within a major field of study. A sage will only
accept service on a permanent, lifetime basis. As a sage will bring
nothing save thinking ability and knowledge, an offer of employment must
consider the following:

\begin{table}[ht]
	\centering
	\begin{tabular}{lc}
		\toprule
			Support \& Salary per Month & 100 to 300d. \\
			Research Grants per Month & 100 to 300d. \\
			Initial Material Expenditure & 5,000d. minimum* \\
		\bottomrule
	\end{tabular}
\end{table}

\emph{*A 5,000d. expenditure will allow the sage to operate at 50\% of
normal efficiency, and for each additional 250d. thereafter, the sage
will add 1\% to efficiency until 90\% is reached (upon expenditure of
15,000d.). After 90\%, to achieve 100\% efficiency the cost per 1\% is
1,000d. (for the obviously erudite and rare tomes, special supplies and
equipment, etc. - assuming such are available, of course). All told,
expenditures must be 25,000d. for 100\% sage efficiency in specific and
exacting question areas.}

A sage can translate any language associated with their specialist
field, costing 50d. per inscription or per page of text. A sage can also
identify items and answer basic questions relating to their specialist
field, costing 100d. per consultation. These fees count toward the
sage's monthly support \& salary, but if the salary has been ``capped
out,'' the fee must still be paid for each task.

Characters can inquire occasionally of a sage for individual or one-off
jobs, but longer employment requires a full commitment from a player
character and full hiring as noted above.

\textbf{Scribe}: Practiced in the art of writing, a typical scribe is
expected to keep records, write letters and copy documents. Others may
possess additional skills, such as cartography, counterfeiting,
cryptography, illuminating or the ability to write, read or otherwise
comprehend more than one language. Such accomplished individuals might
command up to ten times the standard wage.

\textbf{Ship Crew}: Skilled workers who can handle a ship. Sailors can
fight to defend their ship, typically being equipped with a sword,
shield, and leather armour.

\textbf{Ship's Captain}: A captain is required for any large ship, is
skilled like a sailor, and has an intimate knowledge of the particular
coasts they frequent.

\textbf{Spy}: Recruited to secretly watch the actions of others and
gather information, fees may vary wildly, from perhaps a mere hundred
silver pieces to many thousands, depending on the individual and the
difficulty of what is asked.

\textbf{Steward/Castellan}: Responsible for the administration of a
stronghold in the absence or inability of a player character, a steward
holds a position of great prestige and trust. Whilst serving within the
stronghold, a steward is capable of leading forty men at arms and two
lieutenants for every Level of experience he possesses, as well as the
necessary number of sergeants. The monthly wage due to a steward is
equal to his Level x 100d.. A retainer of an appropriate class and Level
could be appointed as steward.

\textbf{Weaponsmith}: Required for the production and maintenance of
weaponry; for every sixty men at arms present, there must be at least
one weaponsmith available. Each must be provided with a workroom, forge,
and assistants at an additional cost (\textasciitilde100d.). A
weaponsmith can use spare time (prorated based on number of supported
troops) to make additional weapons at a rate of five weapons per month
at 25\% of their usual cost.

\vfill

\begin{figure}[hb]
	\centering
	\includegraphics[width=.4\linewidth]{img/candle.jpg}
	\label{fig:candle}
\end{figure}

\newpage

\section{Arcane Spells}\label{arcane-spells}
\begin{multicols}{4}
	
\subsubsection*{Abjuration Spells}\label{abjuration-spells}
\begin{table}[H]
	\begin{adjustbox}{width=1\columnwidth,center=\columnwidth}
		\begin{tabular}{cl}
			\toprule\noalign{}
				Rank & Spell \\
			\midrule\noalign{}
				1 & Alarm \\
				1 & Arcane Ward \\
				1 & Endure Heat \& Cold \\
				1 & Shield \\
				2 & Counterspell \\
				2 & Mind Shield \\
				2 & Protection from Missiles \\
				2 & Protection from Poison \\
				3 & Dispel Magic \\
				3 & Nondetection \\
				3 & Protection from Undead \\
				3 & Protection from Unnatural \\
				4 & Free Action \\
				4 & Hex Weaving \\
				4 & Minor Globe of Invulnerability \\
				4 & Sanctum \\
				5 & Antimagic Shell \\
				5 & Banish Supernatural \\
				5 & Spell Turning \\
				5 & Wall of Force \\
				6 & Globe of Invulnerability \\
				6 & Symbol \\
				6 & Unbinding \\
			\bottomrule
		\end{tabular}
	\end{adjustbox}
\end{table}
	
\vfill

\subsubsection*{Evocation Spells}\label{evocation-spells}

\begin{table}[H]
	\begin{adjustbox}{width=1\columnwidth,center=\columnwidth}
		\begin{tabular}{cl}
			\toprule
			Rank & Spell \\
			\midrule
			1 & Crystal Resonance \\
			1 & Firelight \\
			1 & Floating Disc \\
			1 & Ioun Shard \\
			2 & Arcane Lock \\
			2 & Burning Hands \\
			2 & Shocking Strike \\
			2 & Silence \\
			3 & Chromatic Orb \\
			3 & Fireball \\
			3 & Lightning Bolt \\
			3 & Serpent Glyph \\
			4 & Mnemonic Enhancer \\
			4 & Sending \\
			4 & Telekinesis \\
			4 & Wingbind \\
			5 & Blade Barrier \\
			5 & Clenched Fist \\
			5 & Contingency \\
			5 & Delayed Blast Fireball \\
			6 & Disintegrate \\
			6 & Earthquake \\
			6 & Power Word \\
			\bottomrule
		\end{tabular}
	\end{adjustbox}
\end{table}

\subsubsection*{Conjuration Spells}\label{conjuration-spells}
\begin{table}[H]
	\begin{adjustbox}{width=1\columnwidth,center=\columnwidth}
		\begin{tabular}{cl}
			\toprule\noalign{}
				Rank & Spell \\
			\midrule\noalign{}
				1 & Curse \\
				1 & Fog \\
				1 & Grease \\
				1 & Unseen Servant \\
				2 & Gust of Wind \\
				2 & Phantom Steed \\
				2 & Stinking Cloud \\
				2 & Web \\
				3 & Black Tentacles \\
				3 & Conjure Animals \\
				3 & Rope Trick \\
				3 & Solvent \\
				4 & Banishment \\
				4 & Dimension Door \\
				4 & Duplicate \\
				4 & Elemental Wall \\
				5 & Air Sphere \\
				5 & Cloudkill \\
				5 & Conjure Elemental \\
				5 & Teleport \\
				6 & Dweomerfire \\
				6 & Invisible Stalker \\
				6 & Meteor Swarm \\
			\bottomrule\noalign{}
		\end{tabular}
	\end{adjustbox}
\end{table}

\vfill
\subsubsection*{Illusion Spells}\label{illusion-spells}

\begin{table}[H]
	\begin{adjustbox}{width=1\columnwidth,center=\columnwidth}
		\begin{tabular}{cl}
			\toprule
			Rank & Spell \\
			\midrule
			1 & Blur \\
			1 & Faerie Fire \\
			1 & Minor Illusion \\
			1 & Ventriloquism \\
			2 & Change Self \\
			2 & Invisibility \\
			2 & Mirror Image \\
			2 & Phantasmal Force \\
			3 & Blindness \\
			3 & Disguise \\
			3 & Invisibility Sphere \\
			3 & Spectral Force \\
			4 & Hallucinatory Terrain \\
			4 & Improved Invisibility \\
			4 & Programmed Illusion \\
			4 & Woodland Veil \\
			5 & Advanced Illusion \\
			5 & Permanent Illusion \\
			5 & Seeming \\
			5 & Sensory Deprivation \\
			6 & Illusory Kingdom \\
			6 & Phantasmal Slayer \\
			6 & Project Image \\
			\bottomrule
		\end{tabular}
	\end{adjustbox}
\end{table}

\columnbreak
	
	\subsubsection*{Divination Spells}\label{divination-spells}
	
\begin{table}[H]
	\begin{adjustbox}{width=1\columnwidth,center=\columnwidth}
		\begin{tabular}{cl}
			\toprule
				Rank & Spell \\
			\midrule
				1 & Comprehend Languages \\
				1 & Detect Illusion \\
				1 & Detect Magic \\
				1 & Detect Portals \& Passages \\
				2 & Detect Traps \\
				2 & Guide \\
				2 & Mind Crystal \\
				2 & Perceive the Invisible \\
				3 & Crystal Vision \\
				3 & Locate Object \\
				3 & Tongues \\
				3 & Wizard Eye \\
				4 & Clairvoyance \\
				4 & Divination \\
				4 & Know Weakness \\
				4 & Magic Mirror \\
				5 & Anticipation \\
				5 & Find the Path \\
				5 & Oracle \\
				5 & True Seeing \\
				6 & Foresight \\
				6 & Legend Lore \\
				6 & Scrye \\
			\bottomrule
		\end{tabular}
	\end{adjustbox}
\end{table}
	
\vfill
\subsubsection*{Necromancy Spells}\label{necromancy-spells}

\begin{table}[H]
	\begin{adjustbox}{width=1\columnwidth,center=\columnwidth}
		\begin{tabular}{cl}
			\toprule
			Rank & Spell \\
			\midrule
			1 & Chill Touch \\
			1 & Cure Light Wounds \\
			1 & Grave Shroud \\
			1 & Turn Undead \\
			2 & Assist \\
			2 & Unflagging Endurance \\
			2 & Ray of Fatigue \\
			2 & Transfer \\
			3 & Cloak of the Void \\
			3 & Cure Sickness \\
			3 & Cure Serious Wounds \\
			3 & Vampiric Touch \\
			4 & Animate Dead \\
			4 & Blight \\
			4 & Destroy Undead \\
			4 & Speak with Dead \\
			5 & Control Undead \\
			5 & Cure Critical Wounds \\
			5 & Raise Dead \\
			5 & Restoration \\
			6 & Deathshroud \\
			6 & Energy Drain \\
			6 & Finger of Death \\
			\bottomrule
		\end{tabular}
	\end{adjustbox}
\end{table}

\columnbreak

\subsubsection*{Enchantment Spells}\label{enchantment-spells}
	
\begin{table}[H]
	\begin{adjustbox}{width=1\columnwidth,center=\columnwidth}
		\begin{tabular}{cl}
			\toprule
			Rank & Spell \\
			\midrule
				1 & Animal Friendship \\
				1 & Command \\
				1 & Friends \\
				1 & Sleep \\
				2 & Charm \\
				2 & Enrage \\
				2 & Forget \\
				2 & Pacify \\
				3 & Fear \\
				3 & Hold Person \\
				3 & Hypnotic Pattern \\
				3 & Suggestion \\
				4 & Confusion \\
				4 & Dominate \\
				4 & Fumble \\
				4 & Overlook \\
				5 & Feeblemind \\
				5 & Hold Monster \\
				5 & Mass Charm \\
				5 & Mass Suggestion \\
				6 & Geas/Quest \\
				6 & Glamer \\
				6 & Mass Domination \\
			\bottomrule
\end{tabular}
\end{adjustbox}
\end{table}
		
\vfill

\subsubsection*{Transmutation Spells}\label{transmutation-spells}

\begin{table}[H]
	\begin{adjustbox}{width=1\columnwidth,center=\columnwidth}
		\begin{tabular}{cl}
			\toprule
			Rank & Spell \\
			\midrule
	1 & Enlarge (R) \\
	1 & Feather Fall \\
	1 & Nullify Poison \\
	1 & Pass Without Trace \\
	2 & Arcane Cipher \\
	2 & Knock \\
	2 & Spider Climb \\
	4 & Enhance \\
	4 & Fabricate \\
	4 & Polymorph Self \\
	4 & Transmute Rock to Mud (R) \\
	5 & Fly \\
	5 & Iron Body \\
	5 & Passwall \\
	5 & Reverse Gravity \\
	6 & Control Water \\
	6 & Control Weather \\
	6 & Move Terrain \\
	6 & Shape Change \\
			\bottomrule
\end{tabular}
\end{adjustbox}
\end{table}

\end{multicols}
\newpage

\subsection*{Spell Details}\label{spell-details}

Each spell has a spell rank, range, and duration.

\subsubsection*{Range}\label{range}

Range is the maximum distance of the spell's target, not its area of
effect (though sometimes these are the same). Ranged spells must obey
the normal rules on line of sight and blocked targets, but do not suffer
range modifiers. Spells typically have one of five ranges:

\begin{itemize}
	\tightlist
	\item
	Self: The spell can only target the caster. However, if the range is
	``Self'' followed by an area, the effect emits from the caster and
	covers the area specified.
	\item
	Touch: The spell requires the caster to touch the target creature or
	object to affect it, which means that the caster must be within 5 feet
	of the target at the start of the Magic Phase (when the spell
	resolves). The target can be the caster: if so, the ``touch'' is
	automatic. A normal attack roll is required if the target is hostile.
	\item
	Short: The spell reaches up to 40 feet. This range increases by 5 feet
	for every two caster levels.
	\item
	Medium: Up to 100 feet, plus 10 feet per two caster levels.
	\item
	Long: Up to 400 feet, plus 20 feet per two caster levels.
\end{itemize}

\subsubsection*{Duration}\label{duration}

This is the amount of time the spell is in effect. A caster can dispel
their own non-Instant spell effect at will, as a free action.

\begin{itemize}
	\tightlist
	\item
	Instant: The spell effect comes and goes the moment the spell is cast,
	though the consequences (such as damage) might last beyond this.
	\item
	Timed: Unless dispelled by the caster, the spell lasts until its
	listed time has passed. If a spell's duration is variable, the
	duration is rolled secretly by the GM (i.e.~the caster doesn't know
	how long their spell will last).
	\item
	Concentration: The spell requires that the caster concentrate on it to
	keep its effect active. This prevents the casting of another spell and
	other actions, but movement is still allowed. Anything that would
	disrupt a spell's casting also breaks a caster's concentration while
	maintaining one, thus ending the spell.
	\item
	Permanent: Unless dispelled by the caster, the effect is permanent.
	The effect may still be vulnerable to Dispel Magic or other means of
	destruction; if so, this will be specified in the spell description.
\end{itemize}

\subsubsection*{Area of Effect}\label{area-of-effect}

For spells with an area of effect, the caster selects a point within
range to centre the effect area, but cannot control what a spell affects
within that area unless the spell states otherwise. If using a grid map,
a spell's centre point must be at a grid-line intersection, not a
square.

The area of effect listed for spells (AoE) may have two values. If so,
the first is radius, while the second is the full area of effect in
5-foot squares, if using a grid map. The latter can lead to
counterintuitive results (e.g.~square fireballs), but is an ease-of-use
abstraction to avoid annoyances with diagonals.

\subsubsection*{Reversible Spells (R)}\label{reversible-spells-r}

Some spells are reversible: this will be noted in the spell's title and
description. When one gains a reversible spell both forms are learned,
but each is a different spell for the purposes of preparing spells. For
example, Haste and Slow would occupy separate spell slots (and each
could be prepared up to twice).

\subsubsection*{Saving Throws}\label{saving-throws}

Spells that allow a save have this noted in their description, with the
relevant Ability modifier to add along with the target needed to
succeed. If the spell does damage, a successful save reduces the damage
to that subject by half (round normally); for all other spells a save
nullifies the effect on that subject unless stated otherwise. If a spell
indicates an Ability*2, this means to add the target's Ability Score
modifier twice to their saving throw roll.

\paragraph{Saving Throws \& Illusions}\label{saving-throws-illusions}

Illusions only affect intelligent creatures. A creature is not granted
an extra save to recognize an illusion unless they carefully study it,
directly interact with it (a touch or attack usually counts), make a
deliberate effort to disbelieve it (this requires an action if in
combat), or see a result contravening established reality (e.g.~falling
through an illusionary ``floor'' covering a pit). A save is granted up
to once per round in such circumstances, made in secret by the GM.

A failed save does not dictate one's chosen actions; it only means that
the creature believes in the illusory reality. This also does not make
that reality in any way real. For example, one might choose to cross an
illusory bridge if they fail their save, and if so would of course fall
through it. However, chosen actions will unconsciously adjust to match
perceived reality. For example, if a victim believes they are in a
swamp, they will unknowingly slow their movement to normal swamp speeds.

Passing a save against an illusion does not dispel it, but reveals it to
be false for that person. Any who know an illusion to be false can use
an action to try and convince others of this; those told can reroll
their next failed save against that illusion.

\subsubsection*{Stacking Spell Effects}\label{stacking-spell-effects}

Regardless of the source, the same spell (or equivalent effect from
another source) cannot be stacked on a single target unless specified
otherwise. For example, two Shield spells on one target does not provide
a cumulative effect, and neither does a Shield spell and a Shield effect
from a magical item.

\newpage
\begin{adjustwidth}{-40pt}{-40pt}
	
\begin{multicols}{2}
\subsection*{Rank 1 Spells}\label{rank-1-spells}
\small
\paragraph{Alarm}\label{alarm}
\emph{Abjuration (R: Short, D: 1 day)}\\
AoE: 60-ft radius. Whenever a creature of size level Small and up
touches or enters the warded area, a mental alarm goes off. The caster
must be within one mile of the warded area to receive this alarm; if
sleeping, they are automatically woken. The caster can designate
specific individuals, creatures, or creature types that won't trip
the alarm.

\paragraph{Animal Friendship}\label{animal-friendship}
\emph{Enchantment (R: Short, D: 1 hour)}\\
This spell makes a normal animal regard the caster as a trusted friend
and ally. If the animal is currently being threatened by the caster or
its allies, it is allowed a Willpower Saving Throw to resist the spell.

\paragraph{Arcane Ward}\label{arcane-ward}
\emph{Abjuration (R: Short, D: 1 hour)}\\
AoE: 50-ft radius / 20 × 20. All chosen subjects apply a +1 AC bonus.

\paragraph{Blur}\label{blur}
\emph{Illusion (R: Self, D: 1 hour)}\\
The caster's form appears to constantly shift and waver. Attacks
directly targeting the caster that rely on sight must reroll hits scored
on the caster. If successful, the illusion is dispelled.

\paragraph{Chill Touch}\label{chill-touch}
\emph{Necromancy (R: Touch, D: 1 minute)}\\
Save: Arcana 14. This spell deals 1d6+2 necrotic damage. If the target
is living, it cannot be healed for the spell's duration; undead
creatures instead must reroll any hits scored against the caster.

\paragraph{Command}\label{command}
\emph{Enchantment (R: Medium, D: 1 round)}\\
Save: Willpower 17. The caster issues a one-word command to the target,
which it must obey on its next set of actions. The command must be
unequivocal, and something the target can accomplish (e.g.~back, halt,
flee, fall, approach, kneel, dance; ``die'' or ``suicide'' do not
qualify).

\paragraph{Comprehend Languages}\label{comprehend-languages}
\emph{Divination (R: Self, D: 1 hour)}\\
AoE: Up to 40-ft radius / 16 × 16. Up to five creatures (+5 per caster
name level) can comprehend (but not communicate using) unknown text or
speech. Codes / ciphers are not broken; nor are references explained.
More esoteric communications, such as pheromones, are not understood.

\paragraph{Crystal Resonance}\label{crystal-resonance}
\emph{Evocation (R: Short, D: Special)}\\
Either Imprint or Release a crystal worth at least 50d..

Imprint: Hold the crystal or gem aloft for 1 Turn, imprinting a single
type of energy upon it from the surroundings and replacing any
previously absorbed energy. There is a 1-in-20 chance of the crystal
shattering and becoming useless.

Release: Reproduce the stored energy of a previously imprinted gem or
crystal in the environment it now occupies (the area within range).
Released energies only reproduce normal ambient conditions and cannot
cause damage, blindness, deafness, or other special effects.

Types of Energies:

\begin{itemize}
	\tightlist
	\item
	Light: The lighting qualities of an environment may be absorbed and
	reproduced, causing an area of light, gloom, shadow, and so forth to
	be emitted from the crystal for 1 Turn per Level of the caster.
	\item
	Images: A static snapshot of the crystal's environment can be absorbed
	for later examination. The image is reproduced for 1 Turn.
	\item
	Sound: Any sound emitted during the 1 Turn absorption time is recorded
	and may be reproduced.
	\item
	Temperature: Warmth or cold may be absorbed into the crystal and
	re-emitted for 1 Turn per Level of the caster.
\end{itemize}

\paragraph{Cure Light Wounds}\label{cure-light-wounds}
\emph{Necromancy (R: Touch, D: Instant)}\\
This spell heals the subject of 1d6+2 damage. Add 1d6 damage for each
name level the subject (not the caster) has.

\paragraph{Curse}\label{curse}
\emph{Conjuration (R: Short, D: 1 hour)}\\
Save: Arcana 14. AoE: 50-ft radius / 20 × 20. All chosen targets receive
a --2 attack penalty, as well as --1 to Morale checks.

\paragraph{Detect Illusion}\label{detect-illusion}
\emph{Divination (R: Touch, D: 10 minutes)}\\
All illusions within view are revealed as such to the subject.

\paragraph{Detect Magic}\label{detect-magic}
\emph{Divination (R: Self, D: 10 minutes)}\\
AoE: 60-ft radius. All magical emanations glow softly. The caster knows
the magic's strength (faint, moderate, strong, overwhelming) and the
school(s) associated with that magic. The imprinted energy patterns of
memorised arcane spells glow, surrounding arcane spell-casters' heads
with a halo of rainbow hues.

\paragraph{Detect Portals \& Passages}\label{detect-portals-passages}
\emph{Divination (R: Short, D: 10 minutes)}\\
AoE: 60-ft radius. All hidden doors, compartments, and passages within
range glow softly (unless invisible). Obstructions may obscure the glow.

\paragraph{Endure Heat \& Cold}\label{endure-heat-cold}
\emph{Abjuration (R: Touch, D: 12 hours)}\\
Up to one subject per two caster levels is immune to environmental
temperature effects within most natural bounds (more extreme forms like
magma flows, as well as fire- and cold-based attacks, are not affected).

\paragraph{Enlarge (R)}\label{enlarge-r}
\emph{Transmutation (R: Short, D: 1 hour)}\\
Save: Arcana 14. This spell causes instant growth of one creature and
its carried/worn items, or one object, in both size and weight.
Creatures gain one size level---two at level 10. Objects increase 10\%
per level of the caster (max. 100\%), to a limit of 5 cubic feet per
caster level. If insufficient space is available to fully grow, the
creature/object attains the maximum possible size, bursting weak
enclosures but constrained without harm by stronger materials---the
spell cannot crush something by growth. The reverse, Reduce, either
negates Enlarge or makes a creature or object smaller in the same ratios
as Enlarge. Unwilling victims of either spell are entitled to a saving
throw. In all cases, ability scores, damage, HP, and AC are all
unaffected.

\paragraph{Faerie Fire}\label{faerie-fire}
\emph{Illusion (R: Short, D: 10 minutes)}\\
LoS not required. One creature or object per caster level can be
wreathed in a pale glowing light visible out to 80 ft; even invisible
things can be chosen as targets (this does not make them visible).
Against those targets this applies a +2 attack bonus, but the fire is
concealed by Darkness.

\paragraph{Feather Fall}\label{feather-fall}
\emph{Transmutation (R: Medium, D: 10 minutes)}\\
Up to five falling (or about-to-fall) creatures within range have their
rate of descent slow to 60 ft per round (occurring in the Movement
Phase). A subject that lands before the spell ends takes no falling
damage and can land on its feet, and the spell ends for that creature.

\paragraph{Firelight}\label{firelight}
\emph{Evocation (R: Self, D: 1 hour plus 10 minutes per level)}\\
Conjures a ball of flickering flame around the caster's hand, floating
above their shoulder, or around object held. The flame sheds light in a
15' radius, sufficient for reading, but not as bright as daylight. The
conjured flame does not produce heat and cannot be used to ignite
objects or cause damage.

With a magic word, the caster may command the flame to flare suddenly.
All within 30' who see the flare must make a Saving Throw or be dazzled,
suffering a --2 penalty to Attack Rolls for 1d4 Rounds. The caster is
not affected by the flare, but allies may be unless warned to close
their eyes. After flaring, the flame disappears and the spell ends.

\paragraph{Floating Disc}\label{floating-disc}
\emph{Evocation (R: Self, D: 4 hours)}\\
This spell creates a circular, horizontal disc in an empty space within
5 ft of its caster. The disc is 3 ft in diameter and 1 inch thick, and
floats 3 ft above the surface beneath it. It can support up to 500
pounds (about 20 item slots). The disc follows 5 ft behind its caster.
It cannot be ridden by the caster, and cannot cross abrupt elevation
changes of over 10 ft.

\paragraph{Fog}\label{fog}

\emph{Conjuration (R: Long, D: 1 hour)}\\
AoE: Up to 50-ft radius / 20 × 20. The caster summons a wall of thick
fog that reduces vision (including infravision) to 5 ft. A wind of at
least 20 miles per hour disperses it.

\paragraph{Friends}\label{friends}

\emph{Enchantment (R: Self, D: 1 hour)}\\
Save: Willpower 14. AoE: 200-ft radius. Intelligent creatures that fail
their save ignore the fact that the caster just cast a spell and instead
tend towards being impressed with the caster, providing +4 to the
caster's party reaction roll.

\paragraph{Grave Shroud}\label{grave-shroud}
\emph{Necromancy (R: Short, D: 1 hour)}\\
Undead creatures perceive up to six subjects (+5 per caster name level)
as undead. Mindless undead thus ignore the subjects, unless specifically
commanded to attack them directly, while intelligent undead do not
instinctively know them to be alive, but will likely wonder at their
actions or appearance. If a subject attacks an undead creature, the
spell ends for them.

\paragraph{Grease}\label{grease}
\emph{Conjuration (R: Medium, D: 10 minutes)}\\
Save: Dexterity 14. AoE: Up to 40-ft radius / 16 × 16. Slick,
non-flammable grease covers the ground. Creatures in the area when the
grease appears or who move into a greased space (even if already in one)
must save or fall prone (limit of one such save per round). For those
who make their save, the greased area is still treated as difficult
terrain.

\paragraph{Ioun Shard}\label{ioun-shard}
\emph{Evocation (R: Medium, D: 2 Turns)}\\
Conjures a glowing shard of vividly coloured crystal that slowly orbits
the caster's head at a distance of 1'. At the time of casting or at any
point in the spell's duration, the caster may choose to fire the shard
at a visible target within range, automatically striking the target for
1d6+1 damage. Once fired, the shard is destroyed. Higher-Level casters
may conjure one additional shard per 3 Levels: two at Level 4, three at
Level 7, four at Level 10, etc. Multiple shards may be fired
simultaneously and may be directed at a single target. Unlike standard
spell targeting, if something can be seen it can be targeted, even if
normal spellcasting line of sight is blocked.

\paragraph{Minor Illusion}\label{minor-illusion}
\emph{Illusion (R: Medium, D: 1 hour)}\\
Save: Perception 14. This spell creates an illusionary sound, scent, or
image (only one). If a sound, its volume can range from a whisper to a
scream, and vary in pitch and tempo as desired. If an image, it is
silent and immobile, and no larger than a 5-ft cube (2 × 2 × 2).

\paragraph{Nullify Poison}\label{nullify-poison}
\emph{Transmutation (R: Short, D: Instant)}\\
AoE: Up to 10-ft radius / 4 × 4. This spell nullifies all non-magical
poison applied to items (e.g.~coated blades and needles; poison in gas
capsules, food, or vials). Creatures naturally using their own poison
are unaffected.

\paragraph{Pass Without Trace}\label{pass-without-trace}
\emph{Transmutation (R: Short, D: 1 hour)}\\
AoE: Up to 40-ft radius / 16 × 16. Up to five subjects (+5 per caster
name level) can move through any terrain --- mud, snow, dust, etc. ---
and leave neither footprint nor scent, making tracking by these means
impossible. However, areas radiate a dweomer for 6d6 turns after an
affected creature passes through them.

\paragraph{Shield}\label{shield}
\emph{Abjuration (R: Self, D: 10 minutes)}\\
This spell invisibly wards the caster, providing +5 AC and negating any
Magic Missiles cast at them.

\paragraph{Sleep}\label{sleep}
\emph{Enchantment (R: Short, D: 4d4 x 10 minutes)}\\
Save: Willpower*2 14. AoE: 30-ft radius / 12 × 12. The caster rolls 2d6
and adds their level to the result: that many Hit Dice of living enemies
are put into a comatose slumber. Targets with the fewest HD are affected
first, and only creatures with 4 HD or less are affected. Shaking and
slapping a victim as an action gives them an extra save; mere noise does
not.

\paragraph{Turn Undead}\label{turn-undead}
\emph{Necromancy (R: Self, D: 1d6 × 10 minutes)}\\
Save: None. AoE: 30-ft radius / 12 × 12. The caster rolls 2d6 to
determine the result:

\begin{itemize}
	\tightlist
	\item
	4 or lower: The undead are unaffected
	\item
	5-6: 2d4 undead are stunned for 1 Round, unable to act.
	\item
	7-12: 2d4 undead flee from the caster for 1 Turn.
	\item
	13 or higher: 2d4 undead are permanently destroyed.
\end{itemize}

If the caster is a higher level than the targeted undead, add a bonus of
+2 per level, to a maximum of +6. If the caster is a lower level than
the targeted undead, add a penalty of -2 per level, to a maximum of -6.

Attacking a turned undead creature frees it from the effects of the
spell at the end of the current combat round.

In a mixed group of undead with multiple Levels, those of lowest Level
are affected first. On a successful turning roll, the caster may make
another roll the following Round, affecting the next lowest Level type
of undead present. This process may be repeated until all types of
undead have been affected or a turning roll fails.

\paragraph{Unseen Servant}\label{unseen-servant}

\emph{Conjuration (R: Short, D: 2 hours)}\\
The spell summons an invisible spirit, weightless and Small in size,
that serves tirelessly and fearlessly without question. The servant must
remain within 40 feet of its summoner or it is dispelled. It moves as a
normal PC, but ignores terrain. It can carry up to 1 point of
encumbrance, and open normal doors, lids, drawers, etc, but perform no
greater feats of strength. The servant cannot fight or take any other
action requiring an attack roll. Nor can it block spaces, provide cover,
or lock opponents in melee. It can be magically dispelled, banished by
anything that banishes spirits, or eliminated after taking 6 points of
magical damage.

\paragraph{Ventriloquism}\label{ventriloquism}

\emph{Illusion (R: Short, D: 20 minutes)}\\
The caster's voice emanates from anywhere within range (e.g.~a statue, a
tapestry, an animal).

\subsection*{Rank 2 Spells}\label{rank-2-spells}

\paragraph{Arcane Cypher}\label{arcane-cypher}

\emph{Transmutation (R: 5', D: Permanent)}\\
Script which the caster passes their hand over glows and is transformed
into arcane sigils incomprehensible to all except the caster. The text
of a single spell in a spell book may be affected, or up to 1 page of
normal text per Level of the caster. Magical text cannot be affected.
The Arcane Cypher is only decoded by magic (e.g.~by the Decipher spell).

\paragraph{Arcane Lock}\label{arcane-lock}

\emph{Evocation (R: Touch, D: Permanent)}\\
This spell bars the opening mechanism of a door, chest, or the like with
magical force. A caster ignores their own lock, but otherwise nullifying
the lock requires force or a successful Dispel Magic spell. The caster
can optionally set a password for other characters to freely ignore the
lock. A character 4+ levels higher than the lock's caster can bypass or
physically force the lock, as can any Knock spell, but in such cases the
lock is not nullified and reengages after 10 minutes.

\paragraph{Assist}\label{assist}

\emph{Necromancy (R: Touch, D: 1 hour)}\\
The subject is imbued with a burst of life force, gaining 2d6+2
temporary Hit Points, plus 2d6 per name level of the caster (not the
subject). Unlike healing effects, additional HP from Assist can take a
subject over their limit; if so, such HP are always lost first. HP above
the caster's maximum cannot be healed. All Assist HP are lost at the end
of the hour; this might conceivably kill a recipient. A creature can
only be the subject of one active Assist at any one time.

\paragraph{Burning Hands}\label{burning-hands}

\emph{Evocation (R: Self---30 ft × 10 ft line, D: Instant)}\\
Save: Dexterity 14. A burst of flame erupts from the caster's hands.
This deals 2d6 damage, plus 1D6 for every two levels of the caster
beyond the second (e.g. 3d6 at level 4, 4d6 at level 6, 5d6 at level 8;
maximum 10d6 at level 16) and sets combustibles alight.

\paragraph{Change Self}\label{change-self}

\emph{Illusion (R: Self, D: 1 hour)}\\
Save: Perception 17. The subject's appearance---including all personal
items---is altered. Height can be altered up to 1 ft shorter or taller,
and weight can be altered to a similar degree along with it. The same
basic limb arrangement must be kept.

\paragraph{Charm}\label{charm}

\emph{Enchantment (R: Touch, D: 1 day)}\\
Save: Willpower 14. Enchants a small object with the power to charm a
living humanoid of Medium size or smaller who willingly accepts it as a
gift. A single recipient who accepts the enchanted object must succeed
on the save or regard the caster as a trusted friend and come to their
defence. The caster can give the subject orders, language permitting.
The subject will not knowingly endanger its life beyond reason. Orders
not befitting a friend will grant an extra save. At 10th level, the
caster can affect any one living creature, language permitting, of up to
Large size.

\paragraph{Counterspell}\label{counterspell}

\emph{Abjuration (R: Short, D: Instant)}\\
A caster can always pick this spell (if prepared) as their declared
spell in response to an enemy casting a spell, replacing any other
declared spell. In the Magic Phase, before spells are resolved, the
dispeller and the enemy each make a dispel check: roll 1d6 + caster
level + rank of the spell being cast (e.g., Counterspell will usually be
2). The highest result wins; ties go to the dispeller. If the dispeller
wins, the enemy spell is disrupted.

\paragraph{Detect Traps}\label{detect-traps}

\emph{Divination (R: Short, D: 10 minutes)}\\
AoE: 60-ft radius. All traps within range glow softly. The spell does
not detect natural hazards, such as a weakness in a floor or an unstable
ceiling, and the glow can be obscured by obstructions (e.g.~a trapped
door hidden behind a bookshelf).

\paragraph{Enrage}\label{enrage}

\emph{Enchantment (R: Short, D: 10 minutes)}\\
Save: Willpower 17. The subject flies into a berserker rage. Regardless
of its armament, it always attacks in melee the creature closest to it,
friend or foe, and will fight as offensively as possible, though it will
not throw itself over a cliff or the like in order to reach an opponent.

\paragraph{Forget}\label{forget}

\emph{Enchantment (R: Short, D: Permanent)}\\
Save: Willpower 14. AoE: Up to 20-ft radius / 8 × 8. Chosen subjects
within the area of effect forget the events of the previous minute. For
every name level of the caster, another two minutes of past time is
forgotten. A Restoration will recover the lost memories.

\paragraph{Guide}\label{guide}

\emph{Divination (R: Self, D: 12 hours)}\\
The caster gains a supernatural knowledge of the paths and perils of
their surroundings while travelling overland. The caster applies a +1
bonus that day to their party's navigation checks.

\paragraph{Gust of Wind}\label{gust-of-wind}

\emph{Conjuration (R: Self---60 ft × 10 ft line, D: 1 minute)}\\
Wind blasts from the caster in the direction faced. Unless braced,
creatures that start their round in the line must make a Strength check
or be pushed 5 ft away for each point that they failed the roll by;
apply a --2 roll modifier to flying creatures. Any creature in the line
must spend 2 ft of movement for every 1 ft it moves within the line. The
gust disperses Fog, Stinking Cloud, and Cloudkill spells, smoke, vapour
etc, and extinguishes small open flames like torches. Protected flames
such as lanterns have a 50\% chance to be extinguished. The caster can
change the direction the line travels as an action during their round.

\paragraph{Invisibility}\label{invisibility}

\emph{Illusion (R: Touch, D: Concentration---1 hour per caster level)}\\
One creature or object of up to one size level larger than the caster
vanishes from sight. An invisible creature can see itself. Any light
source carried is invisible, but the light it casts is unaffected;
dropped personal items become visible. The spell ends if the subject
attacks, though surprise is very likely. Attacks against invisible
targets apply a --4 penalty.

\paragraph{Knock}\label{knock}

\emph{Transmutation (R: Touch, D: Instant)}\\
The caster knocks on a single closed door, gate, lid, or similar portal
with their hand or a staff. The portal groans, grumbles, and magically
opens. Locks and bars are unlocked or removed. Arcane Locks are disabled
for 1 Turn. Shackles or chains are loosened, but the spell will not
raise a portcullis or the like. Secret doors may be opened, but they
must be known to the caster.

\paragraph{Mind Crystal}\label{mind-crystal}

\emph{Divination (R: Short, D: 2 hours)}\\
Attunes a gem or crystal (of at least 250gp value) to the subtly
radiating energies of living minds, allowing the caster to detect the
presence of creatures and perceive their thoughts. To pick up thoughts,
the caster must hold the crystal aloft and focus their concentration in
a particular direction for 1 Turn without moving. After 1 Turn, the
crystal projects the thoughts of all creatures within range in the
chosen direction into the caster's mind. If multiple creatures are
within range in the direction being focused on, the caster perceives an
incomprehensible mix of all their thoughts. If the caster focuses for an
additional Turn, they can isolate a single creature's thoughts. The
spell can be maintained for the duration only if the caster continues to
hold the crystal. This spell is blocked by 2' of rock or a thin layer of
lead.

\paragraph{Mind Shield}\label{mind-shield}

\emph{Abjuration (R: Touch, D: 12 hours)}\\
The subject cannot be detected by mental scans or have their mind or
thoughts read, and their saves against mental attacks are one difficulty
level lower. However, the subject cannot use or receive telepathic
signals.

\paragraph{Mirror Image}\label{mirror-image}

\emph{Illusion (R: Self, D: 1 hour)}\\
In a flash of prismatic light, 1d6 illusory duplicates of the caster
appear. The mirror images look and behave exactly as the caster,
remaining within 3'. The caster's voice, as well as any subsequent
spells cast, emanate randomly from one of the mirror images or the
caster's true form. Attacks on the caster destroy one of the mirror
images (even if the attack misses).

\paragraph{Pacify}\label{pacify}

\emph{Enchantment (R: Medium, D: 1 hour)}\\
Save: Willpower 14. AoE: Up to 40-ft radius / 16 × 16. This spell calms
1d6 living creatures, ending any arguments, combats, or inclinations to
them. This does not create friendly feelings, however. Creatures with
the fewest HD are affected first. A pacified creature that takes damage
or is attacked breaks free of the spell.

\paragraph{Perceive the Invisible}\label{perceive-the-invisible}

\emph{Divination (R: 10' per caster level, D: 1 hour)}\\
Invisible creatures and objects in range become visible to the caster,
outlined in glittering gold.

\paragraph{Phantasmal Force}\label{phantasmal-force}

\emph{Illusion (R: Medium, D: Concentration)}\\
Save: Perception 14. This spell creates the visual or auditory illusion
of a known object, creature, scene, attack, or force of up to Medium
size (5' square). The caster can move the image within the spell's
range. Any ``damage'' dealt directly by the illusion is believed to be
real (illusory arrows, pit trap, monster, etc), but vanishes when the
illusion is disbelieved or gone and does no real damage and has no
permanent effect. A creature reduced to 0 HP in this way falls
unconscious. Illusory damage and effects last for 1d4 Turns. An illusory
monster can be of a Level no greater than the caster's. The monster may
be directed to attack, having all its normal capabilities and attack
forms. It has AC 10 and vanishes if hit in combat. If the illusion is of
something the caster has not personally seen, viewers gain a +4 bonus to
their save.

\paragraph{Phantom Steed}\label{phantom-steed}

\emph{Conjuration (R: Short, D: 12 hours)}\\
Up to five (+5 per caster name level) ghostly draft or riding
horses---caster's choice---instantly appear, which serve their riders
willingly and well. Each behaves as a normal horse, except that it will
not tire, panic, or otherwise be frightened, and does not need food or
water.

\paragraph{Protection from Missiles}\label{protection-from-missiles}

\emph{Abjuration (R: Touch, D: 1 hour)}\\
The subject cannot be damaged by normal missile attacks; these will only
break a caster's concentration if they hit on a natural 20. Magic or
non-personal scale missiles (such as a boulder from a giant or a
catapult) are instead reduced in damage by 1 point per die of damage.

\paragraph{Protection from Poison}\label{protection-from-poison}

\emph{Abjuration (R: Touch, D: 12 hours)}\\
The subject is immune to poison for the spell's duration (including to
any afflicting them). If cast on one who has failed a lethal poison save
within a turn of their being poisoned, the poison is nullified and the
subject lives.

\paragraph{Ray of Fatigue}\label{ray-of-fatigue}

\emph{Necromancy (R: Short, D: 10 minutes)}\\
Save: Constitution 14. The subject becomes heavily fatigued. This has no
effect if the target is already this fatigued or worse, or does not tire
(e.g.~undead, constructs).

\paragraph{Shocking Strike}\label{shocking-strike}

\emph{Evocation (R: Self, D: 2 hours)}\\
This spell charges the caster's hand or a weapon they are holding with a
number of charges equal to the caster's level. While charged, attacks
made with the hand or weapon deal an additional 1d6 electrical damage.
For a missile weapon, one charge is dissipated per attack whether or not
the attack hits, but a melee weapon's charges are dissipated only on a
hit. The spell adds one damage die per name level of the caster.

\paragraph{Silence}\label{silence}

\emph{Evocation (R: Medium, D: Concentration---10 minutes)}\\
AoE: Up to 15-ft radius / 6 × 6. This spell creates a void of silence in
which no sound can be created or pass through. Anything inside the area
of effect is immune to sonic attacks, and creatures inside it are
deafened. Casting a spell that includes a verbal component is impossible
there.

\paragraph{Spider Climb}\label{spider-climb}

\emph{Transmutation (R: Short, D: Concentration---10 minutes)}\\
So long as its hands are free, a subject can climb and travel on
vertical surfaces or even traverse ceilings, with a combat speed of 20
ft per round. Climbing checks are not required for this, but if one is
somehow forced, the difficulty is one level lower than normal.

\paragraph{Stinking Cloud}\label{stinking-cloud}

\emph{Conjuration (R: Medium, D: 10 minutes)}\\
Save: Constitution 14. AoE: Up to 15-ft radius / 6 × 6. A mass of
nauseous vapors is created; any within that fail their save is rendered
a helpless target for 1d6 rounds. Passing reduces a victim's combat
speed by half; if in the cloud at the round's end the creature must save
anew.

\paragraph{Transfer}\label{transfer}

\emph{Necromancy (R: Touch, D: Instant)}\\
The caster can transfer their life force to one recipient, living or
undead. Hit Points are transferred to the subject on a 1:1 basis, as
determined by the caster. There is no limit to the amount of Hit Points
the caster can transfer, except that the caster cannot go below 1 HP in
this fashion; this is treated as healing.

\paragraph{Unflagging Endurance}\label{unflagging-endurance}

\emph{Necromancy (R: Short, D: 12 hours)}\\
Up to one subject per caster level becomes nearly immune to exhaustion.
Subjects can force-march that day with no ill effect, ignore the first
burden level from encumbrance, and gain a +4 on saves vs.~fatigue,
sleep, weakness, or enfeeblement.

\paragraph{Water Breathing}\label{water-breathing}

\emph{Transmutation (R: Short, D: 1 day)}\\
Up to one subject per caster level can breathe water freely.

\paragraph{Web}\label{web}

\emph{Conjuration (R: Short, D: 20 minutes per caster Level)}\\
Conjures a volume of sticky webbing, blocking a 10' cube area. Must be
anchored on at least two opposite sides to solid points. Any creature of
size Large or smaller that moves into the web or is in its area when it
is cast is stuck and cannot move. Characters with Strength below 13
cannot break free. Those with Strength 13 or above can break free in 1d4
Turns. Large creatures of greater than human Strength can break free in
1d4 Rounds. The web can be destroyed by fire in 2 Rounds. Creatures
caught within flaming webs suffer 1d6 damage.

\subsection*{Rank 3 Spells}\label{rank-3-spells}

\paragraph{Black Tentacles}\label{black-tentacles}

\emph{Conjuration (R: Medium, D: 10 minutes)}\\
Save: Dexterity 14. AoE: 20-ft radius / 8 × 8. Flailing ebony tentacles
erupt from the ground. The area becomes difficult terrain. A creature
size Large or smaller that starts its round in or enters the area must
save or take 3d6 crushing damage and become immobile. Those restrained
save at the start of each round to free themselves; failure means the
damage is dealt again.

\paragraph{Blindness}\label{blindness}

\emph{Illusion (R: Short, D: Permanent)}\\
Save: Arcana 14. One target sees only an impenetrable grey mist (--4
attack penalty, no ranged casting possible unless LoS is not required by
the spell). A blinded creature can still deliver gaze attacks. Only
Dispel Magic, Hex Weaving, or the caster dismissing the spell ends the
effect.

\paragraph{Chromatic Orb}\label{chromatic-orb}

\emph{Evocation (R: Medium, D: Instant)}\\
The spell hurls an orb of energy with a total attack bonus equal to the
caster's level. When cast, the caster chooses one of acid, cold,
electrical, fire, or poison: on a hit, the target takes 4d6 damage of
the chosen type. If the same number is rolled on two or more of the
damage dice, the orb can leap to a new target of the caster's choice
within 30 feet of the original target (a new attack is rolled). This can
only occur once per orb cast.

\paragraph{Cloak of the Void}\label{cloak-of-the-void}

\emph{Necromancy (R: Self, D: 10 minutes)}\\
The caster is wreathed in black flames of negative energy. Any living
creature touching the caster with its body or handheld weapons takes 2d6
necrotic damage, +1d6 per caster name level. However, undead touching
the caster are healed for the same amount. This effect is not applied if
the caster strikes a target, only if they are struck.

\paragraph{Conjure Animals}\label{conjure-animals}

\emph{Conjuration (R: Short, D: 1 hour)}\\
This spell calls local wild animal spirits of the caster's choice, which
manifest as living examples of those creatures. Hit Dice equal to twice
the caster's level are summoned, in any combination desired. No single
creature can be above Level 10; none can be intelligent or magical. The
creatures serve without question, but if abused by the caster the
spirits will refuse future calls.

\paragraph{Crystal Vision}\label{crystal-vision}

\emph{Divination (R: Short, D: 2 hours)}\\
Attunes a gem or crystal (of at least 500gp value) to pick up the visual
impressions of living creatures, allowing the caster to see through
others' eyes. To establish a connection with a creature, the caster must
gaze into the crystal and focus their concentration in a particular
direction for 1 Turn without moving. After 1 Turn, facets of the crystal
reflect silent images of the current visual perceptions of one creature
within range. If multiple creatures are within range in the direction
being focused on, the crystal connects with the closest. Once a
connection is established, the caster may choose to maintain it or to
change to another subject. This spell is blocked by 2' of rock or a thin
layer of lead.

\paragraph{Cure Sickness}\label{cure-sickness}

\emph{Necromancy (R: Short, D: Permanent)}\\
This spell cures the subject of artificial or magical blindness and
deafness (assuming the subject still has its eyes/ears) and all
sicknesses and diseases, including mummy rot, green slime infection, and
lycanthropy.

\paragraph{Cure Serious Wounds}\label{cure-serious-wounds}

\emph{Necromancy (R: Touch, D: Instant)}\\
This spell heals the subject of 2d6+2 damage. Add 2d6 for each name
level the subject has.

\paragraph{Darkvision}\label{darkvision}

\emph{Transmutation (R: Touch, D: 1 day)}\\
The subject gains the ability to see normally in darkness up to 60'.

\paragraph{Disguise}\label{disguise}

\emph{Illusion (R: Long, D: Concentration---2 hours)}\\
Save: Perception 17. The appearance of up to one willing subject per two
caster levels---including all carried items---is altered. Height can be
altered up to one full size category shorter or taller; weight can be
altered to a similar degree. The same basic limb arrangement must be
kept. If only one subject is disguised, the need for concentration is
waived.

\paragraph{Dispel Magic}\label{dispel-magic}

\emph{Abjuration (R: Short, D: Instant)}\\
All spell effects within range are unravelled, disintegrating in coils
of coloured energy. Magic items are unaffected. Curses incurred by a
spell are affected; those incurred by a magic item are not. Effects
created by lower Level casters are automatically dispelled. If the
effect is created by a higher-level spellcaster, each caster involved
makes a dispel check: roll 1d6 + caster level, with the higher result
winning (ties go to the dispeller).

\paragraph{Fear}\label{fear}

\emph{Enchantment (R: Medium, D: 1 minute)}\\
Save: Willpower 17. AoE: 30-ft radius / 12 × 12. Chosen enemies in the
area of effect that fail their save drop anything held and flee from the
caster by the quickest safe route; if cornered, they cower. Targets must
be able to feel fear (even instinctually) to be affected.

\paragraph{Fireball}\label{fireball}

\emph{Evocation (R: Medium, D: Instant)}\\
Save: Dexterity 14. AoE: 20-ft radius / 8 × 8. LoS not required. The
blast deals 1d6 fire damage per caster level (max 12d6). Combustibles in
the AoE are set alight, and failed saves force item saves.

\paragraph{Gaseous Form}\label{gaseous-form}

\emph{Transmutation (R: Touch, D: 1 hour)}\\
A willing subject and its carried items become a floating, misty cloud.
The subject has the same size, flies with a combat speed of 10 (though
wind will propel it at the wind's speed), and can enter any space not
airtight. While gaseous, the subject cannot talk, manipulate objects,
attack, or cast spells, and can only be harmed by magic. The spell ends
if the subject drops to 0 or fewer Hit Points.

\paragraph{Haste (R)}\label{haste-r}

\emph{Transmutation (R: Medium, D: 30 minutes)}\\
Save: Arcana 14. Up to one creature per caster level has their Speed
doubled and can take two actions each combat round. The number of spells
a subject may cast per round is not doubled, nor is the use of magic
items such as wands. The reverse, Slow, instead forces targets to act
last in the Melee Phase, and their combat speed is halved. One of these
spells can negate its reverse; such a use has no other effect.

\paragraph{Hold Person}\label{hold-person}

\emph{Enchantment (R: Medium, D: 1 minute)}\\
Save: Arcana 17. One living intelligent creature of up to size Medium is
hypnotised and becomes helpless. Damage does not break the affected
creature out of its hold, but the death or unconsciousness of the caster
does.

\paragraph{Hypnotic Pattern}\label{hypnotic-pattern}

\emph{Enchantment (R: Short, D: 10 minutes)}\\
Save: Willpower 14. AoE: 20-ft radius / 12 × 12. A twisting pattern of
colours is created. On a failed save, intelligent enemies viewing this
can no longer act: they do not move and become helpless targets. The
spell ends for an affected creature if it takes any damage.

\paragraph{Invisibility Sphere}\label{invisibility-sphere}

\emph{Illusion (R: Short, D: Concentration---1 hour per caster level)}\\
A selected creature and all creatures within 10' of it shimmer and
disappear from sight. The spell's effect moves with the chosen creature.
Invisibility ends for anything that leaves the sphere or attacks; this
does not affect the others. Creatures that move into the area after the
spell is cast do not become invisible. Any gear a subject is carrying is
also rendered invisible (this includes clothing and armour). Items
subsequently put down become visible. Carried light sources become
invisible, but the emitted light does not.

\paragraph{Lightning Bolt}\label{lightning-bolt}

\emph{Evocation (R: Medium, D: Instant)}\\
Save: Dexterity 14. AoE: 80 ft long \& 5 ft wide or 40 ft long \& 10 ft
wide; doubled in water. LoS not required. This bolt deals 1d6 electrical
damage per caster level (max 12D6). Failed saves force item saves (see
p.~21).

\paragraph{Locate Object}\label{locate-object}

\emph{Divination (R: Self, D: 30 minutes)}\\
The caster senses the direction of an object on this plane. There is no
range limit, but distance is only indicated vaguely (close, far, etc).
The caster can search for a general item category well known to them
(e.g.~``sword'', ``jewel''), in which case the nearest one (if any) is
located. Trying to locate a specific item requires an accurate mental
image of that item.

\paragraph{Nondetection}\label{nondetection}

\emph{Abjuration (R: Short, D: 4 hours)}\\
Up to one subject per caster level are invisible to all forms of magical
location, scrying, and divination, including Wizard Eye.

\paragraph{Plant Growth}\label{plant-growth}

\emph{Transmutation (R: Long, D: 1 day)}\\
AoE: Up to 50-ft radius / 20 × 20. This spell causes normal vegetation
to thicken and grow. Within the area of effect, combat speed drops to 5
for creatures up to Medium size (10 for creatures Large and greater).
Line of sight drops to 10 ft; anything in sight in the growth has heavy
cover. The area must already have at least living brush in it for this
spell to function.

\paragraph{Protection from Undead}\label{protection-from-undead}

\emph{Abjuration (R: Short, D: 1 hour)}\\
Up to one subject per caster level gains a +4 bonus to saving throws
against Necromantic spells, and are immune to ability score and level
draining by undead. This does not protect against other drain types.

\paragraph{Protection from Unnatural}\label{protection-from-unnatural}

\emph{Abjuration (R: Short, D: 1 hour)}\\
All chosen creatures within range are warded. Undead, constructs, and
supernatural creatures cannot charm, frighten, or possess the warded
creatures, and must reroll any hits scored against them.

\paragraph{Rope Trick}\label{rope-trick}

\emph{Conjuration (R: Short, D: 2 hours)}\\
This spell creates a glowing rope hanging in midair which can be climbed
to an empty extradimensional chamber. When the rope is pulled up behind
a climber, anything inside is hidden from all senses, even magic. The
chamber holds up to 32 points in creatures---a Medium creature occupies
4 points, a Small creature 2, and a Tiny creature 1---along with
sufficient air to breathe. Anything inside is expelled when the spell
ends.

\paragraph{Serpent Glyph}\label{serpent-glyph}

\emph{Evocation (R: Touch, D: Permanent until triggered)}\\
The caster traces a magical, serpent-shaped warding glyph upon a page of
text or a surface, followed by a sprinkling of powdered amber (100gp
value). Cast on a text, the glyph quickly mingles into the script. It
can only be detected by magic. Reading the page triggers the warding
magic. Cast on a surface, the glyph remains visible, glowing pale
yellow. Touching the surface triggers the warding magic. When triggered,
a glowing, serpent-like form leaps from the glyph and makes a single
attack against the nearest creature. Its Attack is equal to the caster's
Level. If the attack hits, the victim is frozen in a glittering amber
bubble of time distortion for 1d4 days (or until dispelled or released
by the caster). While trapped, the victim is in temporal stasis and
cannot move, perceive, think, or act. Likewise, the bubble cannot be
moved or penetrated. If the attack misses, the serpent dissipates with a
flash, a bang, and a puff of smoke.

\paragraph{Solvent}\label{solvent}

\emph{Conjuration (R: Short, D: 1 hour)}\\
Save: Arcana 14. This spell conjures acid covering up to 1-ft area per
caster level. The acid eats through 6'' of wood, leather, or bone, 4''
of stone, or 1'' of metal, per round for three rounds. Creatures made of
such, take 1d6 acid damage per caster level in the first round, half
that (round up) in the second. Other creatures only take the damage for
one round. Saves are made each round. The target must make one set of
item saving throws, even if they save against the solvent damage.

\paragraph{Spectral Force}\label{spectral-force}

\emph{Illusion (R: Medium, D: Concentration)}\\
Save: Perception 14. This spell creates the visual illusion of a known
object, creature, or force of up to a 20' cube area, plus 10' per
caster's name level. The illusion can also have auditory, olfactory and
temperature components, and can be moved within the spell's range. Any
``damage'' dealt directly by the illusion is believed to be real, but
vanishes when the illusion is disbelieved or gone and does no real
damage and has no permanent effect. A creature reduced to 0 HP in this
way falls unconscious. Illusory damage and effects last for 1d4 Turns.
The illusion has AC 18 and vanishes if hit in combat. If the illusion is
of something the caster has not personally seen, viewers gain a +4 bonus
to their save.

\paragraph{Suggestion}\label{suggestion}

\emph{Enchantment (R: Short, D: 12 hours)}\\
Save: Willpower 14. One intelligent living creature will follow a single
spoken course of action (language permitting, and limited to a sentence
or two), if the caster can phrase it to sound reasonable---even if it
isn't.

\paragraph{Tongues}\label{tongues}

\emph{Divination (R: Touch, D: 4 hours)}\\
AoE: Up to 40-ft radius / 16 × 16. Up to one subject per caster level
can understand and speak with almost all living beings. This includes
animals, but not plants. Intelligent creatures fall under no compulsion.
Most others will provide information about nearby locations, including
what they know of the past day. Animals may be convinced to perform a
small favour.

\paragraph{Vampiric Touch}\label{vampiric-touch}

\emph{Necromancy (R: Touch, D: 1 hour)}\\
The first touch the caster makes on a living target deal 1d6 damage for
every two caster levels, to a maximum drain of 6d6. The damage is added
to the caster's total; any Hit Points over the caster's normal total
treated as temporary. Such additional HP can take a subject over their
limit, are always lost first, and cannot be healed or otherwise
restored; HP above the caster's maximum are lost at the end of the hour.

\paragraph{Wizard Eye}\label{wizard-eye}

\emph{Divination (R: Short, D: Concentration---1 hour)}\\
The caster creates an invisible, magical eye through which they can see
up to 60' even in complete darkness. The eye can travel up to 240 ft
from the caster, at a rate of 120 ft per turn. Though invisible, the eye
is tangible (as big as a normal human eye). It cannot pass through solid
barriers. If a caster moves, it cannot move the eye that same turn /
round.

\subsection*{Rank 4 Spells}\label{rank-4-spells}

\paragraph{Animate Dead}\label{animate-dead}

\emph{Necromancy (R: Short, D: 1 day)}\\
Up to 1 medium humanoid corpses or skeletons of the caster's choosing
per Level of the caster rise as undead under the caster's command.
Whatever their capabilities during life, all created undead use the
stats below. They are unable to use any special traits or powers
(including spell casting) possessed in life. The dead are animated until
destroyed, a Dispel Magic spell is cast upon them, or the duration
expires. The caster can make the duration permanent by consuming 100 sp
per Hit Die of the creature(s) animated in ritual components when the
spell is cast.

\textbf{Medium Undead}\\
Level 1 AC 12 HP 1d8 (4) Att Weapon (+0) Speed 20 Morale 12 XP 10\\
Undead: Silent before attacking. Immune to biological effects
(e.g.~disease, poison) and mind-affecting spells.

\paragraph{Banishment}\label{banishment}

\emph{Conjuration (R: Medium, D: 1 minute)}\\
Save: Arcana 14. This spell banishes one creature to another plane of
existence. It remains there until the spell ends, then reappears in the
nearest unoccupied space to its point of banishment. If the subject is
native to the plane of existence the caster is on, the subject is
instead sent to a harmless demiplane. If native to a different plane of
existence than the one the caster is on, the subject is banished to its
home plane and must find another way to return.

\paragraph{Blight}\label{blight}

\emph{Necromancy (R: Medium, D: Instant)}\\
Save: Arcana 14. The subject, which must be living, takes 8d6 necrotic
damage. If the caster targets a plant creature or a magical plant, the
target's saving throw is Arcana 17 instead, and the spell deals double
damage. If a non-magical plant that isn't a creature is targeted, such
as a tree or shrub, it withers and dies without a save.

\paragraph{Clairvoyance}\label{clairvoyance}

\emph{Divination (R: Self, D: 10 minutes)}\\
This spell enables the caster to concentrate upon a familiar locale, no
matter how far away, to see and hear as if the caster were there. Lead
blocks this effect; the caster knows that this has occurred.

\paragraph{Confusion}\label{confusion}

\emph{Enchantment (R: Medium, D: 1 minute)}\\
Save: Willpower 14. AoE: 30-ft radius/12 x 12. This spell causes
creatures to behave unpredictably. It affects 1d6 Levels of enemy
creatures per caster level (up to 15d6). Roll 1d6 to determine the
affected group's actions each round: 1-2 act normally, 3-4 do nothing,
5-6 attack friends. Only creatures of Level 8 or lower are affected;
creatures with the lowest Level are affected first.

\paragraph{Destroy Undead}\label{destroy-undead}

\emph{Necromancy (R: Short, D: Instant)}\\
Save: Arcana 14. This spell disintegrates corporeal undead of equal
Level or lower than the caster's own. Undead of Level 3 or less are
automatically destroyed. Undead of Level 3+1 to 7 can save to avoid
destruction, but suffer 3d6 damage even if they succeed. Undead of
higher than Level 7 instead suffer 1d6x10 damage (2d6 damage if they
succeed). Non-corporeal undead (e.g.~banshees, wraiths, spectres,
shadows, ghosts) are not affected.

\paragraph{Dimension Door}\label{dimension-door}

\emph{Conjuration (R: 10'/360', D: 1 Round)}\\
A pair of glowing, door-shaped rifts in the fabric of space open up for
a moment--one within 10' of the caster and the other at a chosen
destination location; the target location can be known, or the caster
can just state a direction and distance. A single creature may step
through the nearby door and instantly exit the destination door.
Transfer in the other direction is impossible. The caster may forcibly
transport an unwilling creature within 10' by manifesting the nearby
door beside them. The target must make a Strength 14 saving throw or be
sucked through the dimensional door. The subject arrives on a safe and
stable surface as close as possible to the target; this may be the
starting point; the spell has no effect if the destination is occupied
by a solid object.

\paragraph{Divination}\label{divination}

\emph{Divination (R: Self, D: Instant)}\\
This spell places the caster in contact with a planar power. The caster
may ask one question concerning a specific goal or event to occur within
seven days. The reply might be short or cryptic, but will be truthful.
However, the future is as yet unwritten and so events may render the
reply inaccurate.

\paragraph{Dominate}\label{dominate}

\emph{Enchantment (R: Medium, D: Permanent until broken)}\\
Save: Willpower 14. The spell affects one intelligent subject, who
becomes powerless to resist the caster. Commands can be given verbally
or mentally, as a free action, as long as the subject is within range.
The subject can be forced to use their powers and abilities, assuming
the caster knows of them (a victim's secrets are not automatically
revealed). If the victim is ordered to do something they would find
abhorrent, they are first granted another saving throw.

\paragraph{Duplicate}\label{duplicate}

\emph{Conjuration (R: Medium, D: 8 hours)}\\
This spell creates an exact copy of any single non-magical item the
caster touches. The item to be copied must be able to fit inside a
20-foot cube. The duplicated item is identical to the original in every
way, but Detect Magic reveals its true nature. At the duration's end, or
if a Dispel Magic is cast on the item, the item vanishes.

\paragraph{Elemental Wall}\label{elemental-wall}

\emph{Conjuration (R: Medium, D: Concentration or permanent)}\\
The caster creates a wall of fire, ice, stone, or wind up to 1,000 cubic
feet of volume (e.g.~2' thick, 50' long, and 10' high). The caster must
choose the wall's element at the time of memorizing the spell. If the
wall is made of fire, ice, or wind, creatures of Level 3 or lower cannot
pass through the wall; creatures of Level 4 or higher can pass through
the wall, but take 3d6 damage of the chosen element per 10' traveled. A
wall of stone is permanent and must be supported (e.g.~by the ground)
and cannot be created in an area occupied by objects.

\paragraph{Enhance}\label{enhance}

\emph{Transmutation (R: Touch, D: 1 hour)}\\
One chosen Ability score doubles for the subject.

\paragraph{Fabricate}\label{fabricate}

\emph{Transmutation (R: Short, D: Permanent)}\\
Disassembles raw, inanimate materials within range and weaves them into
objects of the caster's choosing. Any simple object typically
constructed from the raw materials may be fabricated. For example, a
door could be made from trees, clothing from wool, etc. The object's
size may be up to 1 cubic yard per Level of the caster. Objects
fabricated from minerals may be up to 1 cubic foot per Level of the
caster.

\paragraph{Free Action}\label{free-action}

\emph{Abjuration (R: Touch, D: 1 hour)}\\
The subject is not slowed by difficult terrain or affected by spells
(such as Web or Black Tentacles) or other effects that are physically
restrictive in nature (as opposed to mind-control effects like Hold
Person). They cannot be Slowed. The subject automatically escapes all
non-magical restraints, including grappling. Finally, being in or
underwater imposes no movement or Dex penalties (though this does not
allow one to breathe water).

\paragraph{Fumble}\label{fumble}

\emph{Enchantment (R: Medium, D: 1 hour)}\\
Save: Willpower 14. One subject becomes clumsy and awkward. Running
creatures will trip and fall, and those reaching for an item or wielding
a weapon will drop it. While affected, recovery from a fall or of a
fumbled object requires a full round (no movement possible). If the
victim makes their save, they are still affected as per a Slow spell
(see Haste).

\paragraph{Hallucinatory Terrain}\label{hallucinatory-terrain}

\emph{Illusion (R: 1 mile, D: 1 day)}\\
Save: Perception 14. This spell makes natural terrain within a 1-mile
radius look, sound, smell, and feel different. The general shape of the
terrain cannot be altered, but minor aberrations can be added
(e.g.~occasional small buildings). Creatures in the area cannot be added
or removed, but their secondary elements, such as bird calls, can be.
True Seeing does not automatically pierce this illusion but does grant
another save with a +4 bonus. If an intelligent being touches the
illusion, that being can see through the illusion.

\paragraph{Hex Weaving}\label{hex-weaving}

\emph{Abjuration (R:Touch, D: Instant or permanent)}\\
Weaving the threads of fate surrounding the subject, this spell
manifests one of the following effects when cast:

Removing a curse: The subject is permanently cured of the affliction of
a single curse. This spell may be used to enable a character to discard
a cursed magic item. The duration is instant.

Placing a curse: The subject must Save Versus Spell or suffer from a
deleterious effect of the caster's choosing. Maximum possible effects
include: a --2 Saving Throw penalty, a --4 Attack Roll penalty, halving
an Ability Score. The duration is permanent.

Multiple curses: So long as each has a unique effect, multiple curses
may afflict the same creature.

\paragraph{Improved Invisibility}\label{improved-invisibility}

\emph{Illusion (R: Touch, D: 1 turn, +1 per name level)}\\
This spell renders the subject invisible, but they are able to attack or
cast their own spells without automatically nullifying the spell. The
invisible creature can attacked with the usual --4 penalty applied to
invisible targets, but in addition its saves are one difficulty level
lower.

\paragraph{Know Weakness}\label{know-weakness}

\emph{Divination (R: Short, D: Instant)}\\
The caster learns the target's current weaknesses and resistances. This
includes anything that causes the target more than the normal amount of
damage, and effects that reduce or negate damage.

\paragraph{Magic Mirror}\label{magic-mirror}

\emph{Divination (R: Short, D: 10 minutes)}\\
The caster creates a mirror-like surface on any solid object. The caster
can see and hear through the mirror as if they were in the location
reflected. The mirror can be moved up to 60 ft away from the caster.

\paragraph{Minor Globe of
	Invulnerability}\label{minor-globe-of-invulnerability}

\emph{Abjuration (R: Self, D: 10 minutes)}\\
The caster is immune to all spells of 3rd Rank or lower. Dispel Magic
can be cast at the globe and affects it normally.

\paragraph{Mnemonic Enhancer}\label{mnemonic-enhancer}

\emph{Evocation (R: Self, D: Instant)}\\
The caster instantly recalls one spell of Rank 3 or lower they have
previously cast.

\paragraph{Overlook}\label{overlook}

\emph{Enchantment (R: Short, D: 1 hour)}\\
Save: Willpower 14. Any intelligent creature wishing to make a direct
spell or weapon attack on the caster must make a save: failure means
that they pick the next logical enemy target instead. If there are no
other targets, the creature can take some other action, but can't attack
that round.

\paragraph{Polymorph}\label{polymorph}

\emph{Transmutation (R: Short, D: 1d6x10 + 10 minutes or permanent)}\\
Transforms the caster or another living subject into another type of
living creature of the same size, as chosen by the caster. The spell
cannot be used to duplicate a specific individual. If the subject dies
while polymorphed, they return to their original form. An unwilling
subject may make a Dexterity 14 saving throw to resist being
transformed.

\textbf{Cast on self:} The transformation fails if the new form's Level
is greater than the caster's. The transformation does not alter the
caster's Hit Points, Saving Throw bonus, Attack, or intelligence.
Physical capabilities of the new form (e.g.~strength, physical attack
forms, modes of movement) are acquired. Non-physical special traits or
powers (e.g.~immunities, breath weapons, spell casting) are not
acquired. While polymorphed, the caster is unable to cast spells. The
duration is 1d6 Turns + 1 Turn per Level.

\textbf{Cast on another:} The transformation fails if the new form's
Level is greater than double the caster's. The transformation does not
alter the subject's Hit Points, but the subject otherwise truly becomes
the new form: all special traits and powers are acquired, along with
behavioural patterns, tendencies, and intelligence. The duration is
permanent.

\paragraph{Programmed Illusion}\label{programmed-illusion}

\emph{Illusion (R: Long, D: Permanent)}\\
Save: Perception 17. This spell creates a preset, fixed Spectral Force
up to 40-ft radius (16 × 16) and up to 5 minutes long that activates
upon a caster-specified trigger condition. Damage cannot be dealt by
this illusion. Once complete, it vanishes, but after 10 minutes it can
be triggered again.

\paragraph{Sanctum}\label{sanctum}

\emph{Abjuration (R: Touch, D: 1 day)}\\
This spell wards a cubic area up to 100 ft per side, blocking any or all
of the following from penetrating into it: sight, smell, hearing,
divination, teleportation, planar travelling. No creature can attack
another while inside it.

\paragraph{Sending}\label{sending}

\emph{Evocation (R: Unlimited, D: Instant)}\\
The caster sends a mental message to another creature. The caster must
be familiar with the recipient, including their name and appearance. The
recipient must be of at least Animal Intelligence. The message may be up
to 25 words. It is understood by the recipient, irrespective of
language.

\paragraph{Speak with Dead}\label{speak-with-dead}

\emph{Necromancy (R: Touch, D: 10 minutes)}\\
The caster can ask a corpse one question. Its spirit must answer and
will be understood, but knows only what it knew in life. If the spirit
recognizes the caster as hostile to the agenda it had in life, it will
answer as literally as it can. A corpse can only be the target of this
spell once per fortnight.

\paragraph{Telekinesis}\label{telekinesis}

\emph{Evocation (R: Long, D: Concentration)}\\
The caster can move objects with their mind. The maximum weight that can
be moved is 20 pounds per caster level. The caster may move the target
up to 20' per Round in any direction desired (including vertically), but
can also hurl the object, dealing 1d6 damage per 20 pounds and ending
the spell. If the caster is harmed, moves, or takes any other action,
the spell ends.

If a creature is targeted, it may make a Strength 14 to resist the
spell. If a held object is targeted, the owner may likewise make a
Strength 14 to resist the spell.

\paragraph{Transmute Rock to Mud (R)}\label{transmute-rock-to-mud-r}

\emph{Transmutation (R: Short, D: 1 hour)}\\
Save: Arcana 14. AoE: Up to 40-ft radius / 16 × 16. This spell turns a
patch of non-magical rock and/or earth into mud 5 ft deep. Combat speed
through the mud is 5 ft per round. Creatures in the area when the spell
is cast must save or fall prone. If cast on a stone golem or the like,
it must make a Constitution 17 save or be slain. The Transmute Mud to
Rock reverse permanently hardens normal mud into soft stone (sandstone
or similar).

\paragraph{Wingbind}\label{wingbind}

\emph{Evocation (R: Medium, D: 10 minutes)}\\
Save: Arcana 17. One flying creature has its wings bound and cannot fly.
If the creature fails its save, it crashes to the ground, taking damage
as appropriate.

\paragraph{Woodland Veil}\label{woodland-veil}

\emph{Illusion (R: Medium, D: Permanent)}\\
Up to 100 Small or Medium creatures (chosen by the caster) within range
are veiled by illusion, appearing as a copse of trees or an orchard.
Once the illusion is in place, even creatures moving among the veiled
subjects are deceived. Subjects who leave the affected area cease to be
veiled. The caster may dismiss the illusion in its entirety at any time.

\subsection*{Rank 5 Spells}\label{rank-5-spells}

\paragraph{Advanced Illusion}\label{advanced-illusion}

\emph{Illusion (R: Long, D: Concentration + 10 minutes)}\\
Save: Perception 17. This spell creates a Spectral Force, but the
maximum size is Gargantuan (20 × 20), plus 5 ft AoE per name level, and
if a combatant it has an AC of 25. The illusion lasts for 10 minutes
after the caster ceases to concentrate upon it.

\paragraph{Air Sphere}\label{air-sphere}

\emph{Conjuration (R: Self, D: 10 minutes per Level)}\\
When immersed in water, the caster is surrounded by a 10' radius sphere
of breathable air. The 10' radius sphere of the spell's effect moves
with the caster. Swimming is not enhanced or affected.

\paragraph{Anticipation}\label{anticipation}

\emph{Divination (R: Self, D: 1 round)}\\
AoE: 50-ft radius / 20 × 20. The caster knows what actions all within
the area of effect will attempt on the following round: moves; precise
spells, weapons, or items to be used; intended targets, etc. The spell
reveals the presence (but not the location) of invisible or ethereal
beings in this way.

\paragraph{Antimagic Shell}\label{antimagic-shell}

\emph{Abjuration (R: Self, D: Concentration--1 hour)}\\
An invisible barrier surrounds and moves with the caster. The caster is
impervious to all magical effects, but the spell also prevents the
functioning of magic items or spells within its confines. It cannot be
dispelled.

\paragraph{Banish Supernatural}\label{banish-supernatural}

\emph{Abjuration (R: Medium, D: 1 minute)}\\
Save: Arcana 14. This spell banishes up to 20 HD worth of chosen
supernatural creatures back to their home plane.

\paragraph{Blade Barrier}\label{blade-barrier}

\emph{Evocation (R: Medium, D: 1 minute)}\\
AoE: 100 ft long, 20 ft high, 5 ft thick. A vertical wall of whirling
blades appears. Any creature passing through the barrier takes 1d6
damage per level of the caster, maximum 10d6; a Morale check is required
to even attempt it.

\paragraph{Clenched Fist}\label{clenched-fist}

\emph{Evocation (R: Short, D: Concentration)}\\
The spell evokes a magical fist as large as a giant's that strikes one
creature per round, never missing. Roll 1D6 each round to determine its
damage for the round: 1-3 = 1D6 damage; 4 = 2D6 damage, 5 = 3D6 damage
and the opponent is helpless for one round, and 6 = 4D6 damage and the
opponent is helpless for 3 rounds. The hand cannot be attacked.

\paragraph{Cloudkill}\label{cloudkill}

\emph{Conjuration (R: Medium, D: 10 minutes)}\\
Save: Constitution 14. AoE: 20-ft radius / 8 × 8. A billowing cloud of
poisonous gas appears, killing any living creature up to Level 6 unless
they save. It moves 10 ft away from the caster each round. Creatures in
the cloud take 2d6 poison damage per round.

\paragraph{Contingency}\label{contingency}

\emph{Evocation (R: Self, D: 1 week)}\\
When this spell is cast, the caster must cast another spell immediately
thereafter, which must have a range of Self and cannot be higher than
4th level. That spell is contained by the Contingency spell, and takes
effect when the precise conditions detailed at the time of casting
Contingency occur; it is already cast and cannot be disrupted or
countered.

\paragraph{Conjure Elemental}\label{conjure-elemental}

\emph{Conjuration (R: Medium, D: Permanent until dismissed or slain)}\\
Conjures a giant being formed of pure elemental matter (air, earth,
fire, or water) to do the caster's bidding. The spell requires a large
volume of the appropriate element. If the caster is harmed, moves at
greater than half Speed, or takes any other action, their command over
the elemental ends. An elemental unchained from a caster's control
immediately attempts to murder its creator and any who get in its way.
While control over the elemental is maintained, the caster may dismiss
it at any time, returning it to inert matter.

\textbf{Elemental} Large Construct---Mindless---Neutral\\
Level 16 AC 21 HP 16d8 (72) Att Blow (+11, 3d8) Speed By type Morale 12
XP 3,050

\begin{itemize}
	\tightlist
	\item
	Construct: Immune to biological effects (e.g.~disease, poison) and
	mind-affecting spells (e.g.~Vapours of Dream, Paralysation, Dominate).
	\item
	Immunities: Only harmed by magic or magic weapons.
	\item
	Air: 32' high, 8' wide vortex of whirling air. +1d8 damage against
	flying creatures. Speed 120.
	\item
	Earth: 16' high humanoid figure of earth or stone. Cannot cross water
	wider than 16'. +1d8 damage against creatures on the ground. Speed 20.
	\item
	Fire: 16' high, 16' wide column of whirling fire. Cannot cross water
	wider than 16'. +1d8 damage against cold-based creatures. Speed 40.
	\item
	Water: 8' high, 32' wide wave of water. Must remain within 60' of
	water. +1d8 damage against creatures in water. Speed 20 (swimming 60).
\end{itemize}

\paragraph{Control Undead}\label{control-undead}

\emph{Necromancy (R: Self, D: 1d6 x 10 minutes)}\\
Save: Arcana 17. AoE: 50-ft radius / 20 × 20. This spell functions as
Turn Undead, except that creatures turned or destroyed are controlled
instead. Only intelligent undead are allowed a save.

\paragraph{Cure Critical Wounds}\label{cure-critical-wounds}

\emph{Necromancy (R: Touch, D: Instant)}\\
This spell heals the subject of 3d6+3 damage. Add 3d6 for each name
level the subject has.

\paragraph{Delayed Blast Fireball}\label{delayed-blast-fireball}

\emph{Evocation (R: Medium, D: Instant)}\\
Save: Dexterity 17. AoE: 20-ft radius / 8 × 8. LoS not required. The
blast deals 1d6+1 fire damage per caster level. While it can explode as
normal, it can also be set to not explode for up to 10 minutes from its
original casting, as stated by the caster at the time of casting.
Combustibles in the AoE are set alight, and failed saves force item
saves.

\paragraph{Feeblemind}\label{feeblemind}

\emph{Enchantment (R: Medium, D: Permanent)}\\
Save: Willpower 17. The subject's intelligence drops to that of an
imbecile. It loses all knowledge and memory of any spells, plus the
ability to communicate and so on, though it can still defend itself in a
primitive fashion. Only Dispel Magic or Hex Weaving can cancel the
effect.

\paragraph{Find the Path}\label{find-the-path}

\emph{Divination (R: Self, D: 1 hour)}\\
The caster learns the most practicable and direct physical route to, out
of, or into a locale (not to creatures / objects within it). If seeking
an area inside a locale, relatively precise criteria must be specified
when the spell is cast (e.g.~``the throne room'', ``the centre''), and
cannot be altered afterwards. The path revealed may not be the safest
route.

\paragraph{Fly}\label{fly}

\emph{Transmutation (R: Touch, D: 1d6x10 minutes, plus 10 minutes per
	caster level)}\\
The subject grows wings that enable them to fly at Speed 40. Encumbrance
speed reductions apply. Clothing is torn; worn armour prevents the spell
from working.

\paragraph{Hold Monster}\label{hold-monster}

\emph{Enchantment (R: Medium, D: 1 minute)}\\
Save: Special (Arcana). One or more living creatures of any size are
hypnotised and become helpless. The caster can choose one target (Arcana
17 save), or up to four (separate Arcana 14 saves).

\paragraph{Iron Body}\label{iron-body}

\emph{Transmutation (R: Touch, D: 1 hour)}\\
The subject's body transforms into living iron, granting +8 AC. Falls do
no damage, nor do non-weapon crushing/constricting attacks, or anything
affecting respiration. The subject's Strength becomes 19 and their
combat speed becomes 10. The subject cannot cast spells. Things that
affect metal function as normal, making rust monsters very dangerous.

\paragraph{Mass Charm}\label{mass-charm}

\emph{Enchantment (R: Medium, D: 1 day)}\\
Save: Willpower 17. Creatures up to combined Levels of twice the
caster's Level within a 30' radius (starting with the lowest-Level
creatures) become friendly to the caster and come to their defence. The
caster can give the subjects orders, language permitting. The subjects
will not knowingly endanger their lives beyond reason. Orders not
befitting a friend will grant an extra save.

\paragraph{Mass Suggestion}\label{mass-suggestion}

\emph{Enchantment (R: Medium, D: 1 day)}\\
Save: Willpower 17. This spell makes one living creature per two caster
Levels follow a suggested course of action (language permitting, limited
to a sentence or two), so long as the caster can phrase it to sound
reasonable---even if it isn't.

\paragraph{Oracle}\label{oracle}

\emph{Divination (R: Special, D: Special)}\\
Cast while burning 1,000d. of rare dust, the caster attempts to contact
a being on another plane of existence. The caster must make a Arcana
check to succeed. If successful, the caster can ask up to three
questions. The questions must relate to a specific subject---an object,
place, or creature. The caster must be in the presence of the subject or
a closely related item (e.g.~the tombstone of a deceased person, the
sheath of a lost sword). Questions receive a brief, cryptic answer. Each
answer has a 1-in-6 chance of being false or misleading. Each casting of
this spell takes 1d6 Turns. The spell may be cast at most once per week.
Contact with powerful extra-dimensional beings can shatter the caster's
mind. The caster must make an Arcana 12 saving throw or enter a coma for
1d6 weeks.

\paragraph{Passwall}\label{passwall}

\emph{Transmutation (R: Touch, Duration: 1 hour)}\\
A 5' diameter hole is temporarily opened in solid rock or stone, forming
a passageway up to 10' deep.

\paragraph{Permanent Illusion}\label{permanent-illusion}

\emph{Illusion (R: As original spell, D: Permanent)}\\
This spell creates a permanent version of any illusion spell of 1st or
2nd level with a non-Instant duration. It can still be dispelled, but
any Concentration requirement is removed. Saves required by the modified
spell have a -4 penalty.

\paragraph{Raise Dead}\label{raise-dead}

\emph{Necromancy (R: Touch, D: Permanent)}\\
Save: Willpower*2 14. The caster restores life to a human or humanoid
dead no more than 1 day for each level of the caster. The subject must
make a save, or the resurrection fails. If it succeeds, the subject
loses 1 point of Constitution permanently, and has heavy fatigue until
one day of bedrest passes for each day that they were dead. Missing
limbs, digits, eyes and other parts will still be missing when the
subject returns to life.

\paragraph{Restoration}\label{restoration}

\emph{Necromancy (R: Touch, D: Permanent)}\\
This spell immediately ends any and all of the following adverse
conditions affecting the target: addiction, blindness, deafness,
disease, fatigue, insanity, poison, and Feeblemind. It heals all damage.
The subject's severed body appendages (digits, hands, feet, limbs,
tails, even heads of multi-headed creatures), broken bones, and ruined
organs grow back.

\paragraph{Reverse Gravity}\label{reverse-gravity}

\emph{Transmutation (R: Self, D: Instant)}\\
AoE: 30-ft radius (12 × 12). This spell reverses gravity in the area of
effect, causing all unfixed objects and creatures within it to ``fall''
upwards 50 feet in one round. If some solid object is encountered in
this ``fall'', the object strikes it in the same manner as a normal
downward fall. During the movement phase of the next round, the affected
objects and creatures fall back down.

\paragraph{Seeming}\label{seeming}

\emph{Illusion (R: Long, D: 12 hours)}\\
Save: Perception 20. The appearance of up to one subject per two caster
levels is altered. The spell disguises physical appearance as well as
worn and carried items. Each creature can seem 1 ft shorter or taller
and appear thin, fat or in between, but the form of each subject must
have the same basic limb arrangement.

\paragraph{Sensory Deprivation}\label{sensory-deprivation}

\emph{Illusion (R: Medium, D: 1 minute)}\\
Save: Perception 17. This spell surrounds the target in a void only it
perceives, rendering it deaf and blind but also blocking all olfactory,
tactile, and taste sensations. Insights via divination or mental links
with others are unaffected.

\paragraph{Spell Turning}\label{spell-turning}

\emph{Abjuration (R: Self, D: 1 hour)}\\
Spells cast directly at the caster rebound back on the original caster.
This includes innate spell-like abilities, but not spell effects from
devices such as wands, or spells with radii unless the caster is the
target point. 1d6+6 spell Ranks are affected by the turning, the amount
rolled in secret by the referee. If there are not enough levels
remaining to turn a spell, that spell works with full effect and the
remaining turn Ranks are kept. If both the protected caster and original
caster have spell turning effects operating, a resonating field is
created: roll 1d6. On a 1-4 the spell is negated, while on a 5-6 it
affects both casters with full effect.

\paragraph{Teleport}\label{teleport}

\emph{Conjuration (R: Touch, D: Instant)}\\
The subject vanishes and reappears at a location of the caster's
choosing with all its gear, up to its maximum load. An unwilling subject
may make an Arcana 14 saving throw to resist the teleportation. The
caster may not intentionally teleport a subject into mid-air or into
solid matter.

Success is automatic if the location is very familiar to the caster.
Otherwise, the caster must make an Arcana check if the location has been
seen casually, with a -2 penalty if viewed only once. On a failure, roll
d100 and consult the below table:

\begin{table}[H]
	\centering
	\begin{tabular}[]{@{}lccc@{}}
		\toprule\noalign{}
			Knowledge of Dest. & Off by 5\% & Too High & Too Low \\
		\midrule\noalign{}
			Only viewed once & 01--50 & 51--75 & 76--00 \\
			Seen casually & 01--80 & 81--90 & 91--00 \\
		\bottomrule\noalign{}
	\end{tabular}
\end{table}

\begin{itemize}
	\tightlist
	\item
	Off by 5\%: The subject appears 5\% of the distance from the starting
	location to the destination away on the nearest safe surface in a
	random lateral direction (e.g., a subject teleporting 1000' might
	appear 50' to the east of the intended destination).
	\item
	Too high: The subject appears 1d10 × 10' above the intended
	destination. If this is inside solid matter, the subject dies
	instantly. Otherwise, they fall from a height.
	\item
	Too low: The subject appears 1d10 × 10' below the intended
	destination. If this is inside solid matter, the subject dies
	instantly.
\end{itemize}

\paragraph{True Seeing}\label{true-seeing}

\emph{Divination (R: Touch, D: 1 hour)}\\
This spell grants the ability to see all things as they actually are.
The subject sees through normal and magical darkness, notices secret
doors, sees invisible creatures or objects, ignores displacement
effects, sees through all illusion spells of 5th level or lower unless
specified otherwise, and sees the true form of shape-changed,
polymorphed, possessed, or transmuted things.

\paragraph{Wall of Force}\label{wall-of-force}

\emph{Abjuration (R: Medium, D: 1 hour)}\\
AoE: Up to 50-ft radius / 20 × 20. The spell creates an invisible
barrier of magic energy, in a straight line, a hemisphere, or a
sphere---chosen when cast. It cannot move, and nothing can damage it or
pass through it. It is unaffected by any spell (including Dispel Magic)
except Disintegrate, which immediately destroys it.

\subsection*{Rank 6 Spells}\label{rank-6-spells}

\paragraph{Control Weather}\label{control-weather}

\emph{Transmutation (R: Self, 3-mi radius, D: Concentration)}\\
Weather conditions of the caster's choosing manifest in the local area
(see list of common conditions below). If the caster is harmed, moves,
or takes any other action, the spell ends. This spell only functions
outdoors.

The following are common weather conditions. Others may be possible, at
the Referee's discretion.

\begin{itemize}
	\tightlist
	\item
	Calm: Clears bad weather (though side-effects---e.g.~mud after
	rain---remain).
	\item
	Extreme heat: Dries up snow or mud (including the Mire spell).
	Creatures in the area move at half Speed.
	\item
	Fog: Visibility drops to 20'. Creatures in the area move at half Speed
	and, at the Referee's option, may also have a chance of getting lost.
	\item
	High winds: Creatures in the area move at half Speed. Missile fire and
	flight are impossible. High winds may be used to increase the sailing
	speed of ships by 50\%. In sandy areas, high winds cause sandstorms,
	reducing visibility to 20'.
	\item
	Rain: --2 penalty to Attack Rolls with missile weapons. Mud forms
	after 3 Turns, halving Speed.
	\item
	Snow: Visibility drops to 20'. Creatures in the area move at half
	Speed. Bodies of water begin to freeze. After the snow thaws, mud
	remains and impedes movement.
\end{itemize}

\paragraph{Deathshroud}\label{deathshroud}

\emph{Necromancy (R: Self, D: 10 minutes)}\\
Save: Arcana 17. A living creature touched by the caster has a black
haze form around it, draining 20\% of its current Hit Points. At the end
of every round thereafter, the same loss is taken. A living creature
that touches or strikes the caster must also save, but at with a +4
bonus. The spell can be ended by casting Dispel Magic, Restoration, or
Cure Critical Wounds on the target, or killing the caster.

\paragraph{Disintegrate}\label{disintegrate}

\emph{Evocation (R: Medium, D: Instant)}\\
Save: Arcana 14. The spell targets a single non-magical creature or
object. If the target fails its save, it is reduced to dust. If the
target succeeds, it takes 10d6 damage. An Antimagic Shell or Globe of
Invulnerability are immune. Resurrection from disintegration is
impossible.

\paragraph{Dweomerfire}\label{dweomerfire}

\emph{Conjuration (R: Short, D: 1 Round)}\\
The caster conjures a magical, prismatic flame that briefly engulfs any
magical energy in a selected 20' cube within range. Objects, areas, or
creatures under the influence of magic are wreathed in flame. Creatures
in contact suffer 1d6 damage. The imprinted energy patterns of memorised
arcane spells burn the mind. Arcane spell-casters must make an Arcana 14
saving throw or suffer 1 damage per memorised spell. Spells cast by
affected creatures this round and next explode in a conflagration of
vivid energy. Creatures casting a spell must succeed on an Arcana 14
saving throw or suffer 3d6 damage, rendering the cast spell ineffective.

\paragraph{Earthquake Wave}\label{earthquake-wave}

\emph{Evocation (R: Special, D: 1 round per 2 Levels of the caster)}\\
A repulsive earthquake 10' wide and 60' long emanates from the caster's
outstretched hands, pushing other creatures away and knocking them
prone. The caster may turn to affect a new path each Round. All
creatures in the path are pushed directly away from the caster at 30'
per Round and must make a Dexterity 14 saving throw or fall prone.
Creatures attempting to move towards the caster are repelled at their
Speed, if it is greater than 30. The caster must concentrate (no
movement or other actions allowed) while repelling creatures. They may
halt to perform other actions and then resume concentration.

\paragraph{Energy Drain}\label{energy-drain}

\emph{Necromancy (R: Touch, D: Instant)}\\
Save: Arcana 17. The caster touches a living creature, draining 1d4
levels.

\paragraph{Finger of Death}\label{finger-of-death}

\emph{Necromancy (R: Medium, D: Instant)}\\
Save: Arcana 17. The caster points at one living target; if they fail
their save, they die.

\paragraph{Foresight}\label{foresight}

\emph{Divination (R: Self, D: 2 hours)}\\
The caster gains a limited form of precognition. The caster gains +5 AC,
cannot be surprised, gains a +4 bonus to attack rolls and saving throws,
and gains a +1 to ability checks.

\paragraph{Geas/Quest}\label{geasquest}

\emph{Enchantment (R: Touch, D: Special)}\\
Save: Willpower 17. This spell places a magical command on a creature to
perform--or avoid performing--a certain action. The subject must follow
the command to the best of its ability. If the subject fails to follow
the command, it takes 1d6 damage per day per name level and suffers a
level of fatigue. Dispel Magic is ineffective against a geas. Remove
Curse and Hex Weaving only work if the caster is 2 or more Levels higher
than the caster of Geas.

\paragraph{Glamer}\label{glamer}

\emph{Enchantment (R: Long, D: Special)}\\
Save: Willpower 17. Up to four intelligent creatures (+1 per caster Name
Level) can be chosen as targets. Subjects cannot move, drop what they
are carrying, and become helpless targets. Enraptured targets make
another save at the end of each round past the first. Further
enchantments cast upon those under the effects of this spell have a -4
penalty applied to their saving throws.

\paragraph{Globe of Invulnerability}\label{globe-of-invulnerability}

\emph{Abjuration (R: Self, D: 10 minutes)}\\
The caster is immune to all spells of 4th rank or lower. Dispel Magic
can be cast at the globe and affects it normally.

\paragraph{Illusory Kingdom}\label{illusory-kingdom}

\emph{Illusion (R: Long, D: 1 year)}\\
Save: Perception 20. The caster makes natural terrain within a 1-mile
radius look, sound, smell, and feel different. The general shape of the
terrain cannot be altered, but minor aberrations can be added
(e.g.~occasional small buildings). Creatures in the area cannot be added
or removed, but their secondary elements, such as bird calls, can be.
True Seeing does not automatically pierce this illusion but does grant
another save with a +4 bonus.

\paragraph{Invisible Stalker}\label{invisible-stalker}

\emph{Conjuration (R: Short, D: Special)}\\
Summons an invisible, extra-dimensional entity to the caster's presence,
magically binding it to perform a mission of the caster's choosing. The
caster must be careful with the wording of the mission. Invisible
stalkers are intelligent and treacherous. Unless the assigned mission
can be easily and quickly accomplished, the stalker follows the letter
of the command while twisting the intent. The creature is bound to
attempt the mission until it succeeds or is destroyed.

\textbf{Invisible Stalker} Medium Monstrosity---Sentient---Neutral\\
Level 8 AC 16 HP 8d8 (36) Att Crush (+7, 4d4) Speed 40 Morale 12 XP
1,040\\
Tracking: Without fault.\\
Surprise: 5-in-6, unless target can detect invisibility.\\
If killed: Returns to dimension of origin.

\paragraph{Legend Lore}\label{legend-lore}

\emph{Divination (R: Self, D: Instant)}\\
This spell brings to the caster's mind information about a noteworthy
person, place, or thing. If the person or thing is at hand, or if the
caster is in the place in question, the information is gained in 1d6
turns. If detailed information on the subject is known, the information
is gained in 1 week. If only rumours are known, the information is
gained in 2d6 weeks. Information revealed is often cryptic.

\paragraph{Mass Dominate}\label{mass-dominate}

\emph{Enchantment (R: Medium, D: Permanent)}\\
Save: Willpower 17. AoE: 50-ft radius (20 × 20). The spell imposes the
effects of the Dominate spell on up to 2 Levels of intelligent creatures
per caster Level. Commands are transmitted telepathically, and language
and distance are no barrier. The caster can release individuals at will.

\paragraph{Meteor Swarm}\label{meteor-swarm}

\emph{Conjuration (R: Long, D: Instant)}\\
Save: Arcana 17. AoE: 10-ft radius / 4 × 4 per meteor. Blazing orbs of
fire plummet to the ground at up to four different visible points. Each
creature in a meteor's takes 6d6 bludgeoning damage and 6d6 fire damage.
The areas of effect cannot overlap; if the target area is too small to
allow this, meteors are removed from the spell until no overlap occurs.

\paragraph{Move Terrain}\label{move-terrain}

\emph{Transmutation (R: Long, D: 1 hour)}\\
The land bulges and warps as a terrain feature moves under the caster's
control. A single feature (e.g.~a hill, ridge, grove, pool, etc.)
contained in a 120' square area may be moved. The caster can move
terrain at up to 60' per Turn. Any buildings or creatures present move
with the terrain feature. The caster must concentrate (no movement or
other actions allowed) while moving terrain. They may halt to perform
other actions and then resume concentration.

\paragraph{Petrification}\label{petrification}

\emph{Transmutation (R: Medium, D: Permanent or instant)}\\
Manifests one of the following effects when cast:

\begin{itemize}
	\tightlist
	\item
	Flesh to stone: Permanently transforms a living creature (including
	equipment) into stone. The victim may make a Constitution 14 to
	resist.
	\item
	Stone to flesh: Restores a magically petrified creature (and its
	equipment) to life.
\end{itemize}

\paragraph{Phantasmal Slayer}\label{phantasmal-slayer}

\emph{Illusion (R: Short, D: 1 minute)}\\
Save: Perception 17. This spell creates an illusory manifestation of a
target's worst fears. Mindless creatures are immune, as are those immune
to fear. The slayer attacks as a Level 10 monster; on a hit, if the
target fails their save, they die. On a successful save, the spell ends.

\paragraph{Power Word}\label{power-word}

\emph{Evocation (R: Short, D: Special)}\\
This spell has a casting time of 1. When declared, the caster picks one
word of power, which determines the spell's effect.

\begin{itemize}
	\tightlist
	\item
	\textbf{Blind:} The word affects up to 75 Hit Points of creatures. If
	50 or fewer HP are affected, the duration is 1d6 turns; if 51 or more
	HP are affected, it is 1d6 rounds. Blindness can be removed by the
	usual means, and by Dispel Magic.
	\item
	\textbf{Kill:} The word destroys a creature with up to 50 HP, or
	multiple creatures with 10 or fewer HP, up to a maximum of 90 HP.
	\item
	\textbf{Stun:} One target is stunned: unable to act, dropping anything
	held, and counting as surprised. This lasts a base 10+1d6 rounds, --1
	per Level of the target.
\end{itemize}

\paragraph{Project Image}\label{project-image}

\emph{Illusion (R: Long, D: 1 hour)}\\
Save: Perception 20. The caster creates an illusory duplicate that
looks, sounds, and smells like the caster. It reacts naturally to the
environment (e.g. its clothes will move in a draft). The illusion mimics
the caster's actions, including speech, unless the caster makes it act
differently; if the caster chooses such, then while doing so the spell
is a concentration spell. The caster can replace their own sensory input
with that of the illusion's, thus seeing, smelling and so on only what
the illusion would; if so, the caster's spells can originate from the
image. The image appears unaffected by spells or missile weapons, but if
the image is touched or hit in melee, it disappears.

\paragraph{Scrye}\label{scrye}

\emph{Divination (R: Special, D: Concentration---2 hours)}\\
The spell targets one creature; if willing, the target can permit the
spell to work automatically. Otherwise, to scrye the target requires an
Arcana check with a -1 penalty. If the caster has a possession of the
target (e.g.~clothes, jewelry), there is no penalty; if the caster has a
body part, lock of hair, etc. of the target, the caster gains a +1 bonus
to the check. The spell creates an invisible eye about the size of a
fist within 10 ft of the target. The caster can see and hear through the
eye as if they were there. The eye moves with the target; True Seeing
reveals it. Alternatively, the caster can choose a location they have
seen before as the target. If so, the eye appears at that location and
doesn't move.

\paragraph{Symbol}\label{symbol}

\emph{Abjuration (R: Touch, D: Special)}\\
The caster inscribes one of the following symbols on a surface.
Creatures that pass over, touch, read, or pass through a portal upon
which the symbol is inscribed trigger it, after which it flares
brightly, then vanishes.

\begin{itemize}
	\tightlist
	\item
	Death --- Creatures totalling no more than 60 Hit Points are slain.
	\item
	Discord --- All creatures are affected and immediately fall to loud
	bickering. There is a 50\% chance that creatures in different groups
	attack one other. The bickering lasts 3d6+2 rounds; any fighting 1d6+1
	rounds.
	\item
	Fear --- This operates as an extra-strong Fear spell, with a target of
	Willpower 20.
	\item
	Hopelessness --- All creatures are affected and are overcome with
	dejection unless they make a Willpower 17 save. Those affected will
	submit to the demands of any opponent, i.e.~get out, surrender, etc.
	Roll 2d6 for the duration in turns, then 1d6: on a 1-2, those affected
	take no action, while on a 3-4 they turn back or retire from battle,
	as applicable (5-6 no additional effect).
	\item
	Insanity --- Creatures totalling no more than 90 Hit Points act as if
	a Confusion spell had been placed upon them until a Restoration spell
	removes the madness.
	\item
	Pain --- All creatures are affected. Wracking pains shoot through
	their bodies, inflicting a --4 attack penalty for 3d6+2 turns.
	\item
	Persuasion --- All creatures are affected, becoming friendly to the
	caster for 3d6+2 turns unless they make an Arcana 17 save.
	\item
	Sleep --- All creatures under Level 8+1 fall into a catatonic slumber.
	They cannot be awakened for 2d6+4 turns.
	\item
	Stunning --- Creatures totalling no more than 120 Hit Points are
	stunned for 2d6 rounds: they are unable to act, drop anything held,
	and count as surprised.
\end{itemize}

\paragraph{Unbinding}\label{unbinding}

\emph{Abjuration (R: Medium, D: Instant)}\\
AoE: 40-ft radius / 16 × 16. All spell effects present are ended, all
magic constructs and animated dead are destroyed (Heroic save to
prevent), and all magically summoned creatures are banished to their
place of origin. The spell does not affect non-animated undead, or
creatures that have been gated or otherwise transported to a plane by
means other than direct summoning. If a spell effect is permanent, and
greater than 2nd Rank, then it is instead merely suppressed for 2d6
hours.

\newpage

\section{Holy Spells}\label{holy-spells}

Holy Spells are taken directly from
\emph{\href{https://www.dolmenwood.necroticgnome.com/rules/doku.php?id=holy_magic}{Dolmenwood}},
with a few edits and exceptions. Our version is as follows:

\subsection*{Rank 1 Spells}\label{rank-1-holy}

\paragraph{Detect Evil (St.~Whittery's
	Vision)}\label{detect-evil-st.-whitterys-vision}\mbox{}\\
\emph{(R: 120', D: 6 Turns)}\\
The caster perceives a faint halo of wicked, grinning spirits flickering
around objects under an evil enchantment and living beings with evil
intentions.

\begin{itemize}
	\tightlist
	\item
	\textbf{Intent only}: The caster cannot read the thoughts of creatures
	with evil intent.
	\item
	\textbf{Definition of evil}: The Referee must judge what is classified
	as evil. Beings of Chaotic Alignment do not always have evil intent.
	Traps and poisons, while potentially harmful, are not evil.
\end{itemize}

\paragraph{Detect Magic (Wisdom of
	St.~Thorm)}\label{detect-magic-wisdom-of-st.-thorm}\mbox{}\\
\emph{(R: 60', D: 2 Turns)}\\
Enchanted objects, areas, or creatures within range of the caster are
wreathed in a shimmering, golden glow. Both permanent and temporary
enchantments are revealed.

\paragraph{Frost Ward (St.~Abthius'
	Rebuke)}\label{frost-ward-st.-abthius-rebuke}\mbox{}\\
\emph{(R: 30', D: 6 Turns)}\\
A soothing warmth comes upon all allies within range, rebuking the
malicious effects of cold and frost.

\begin{itemize}
	\tightlist
	\item
	\textbf{Normal cold}: Subjects are untroubled by non-magical freezing
	temperatures.
	\item
	\textbf{Save bonus}: Subjects gain a +2 bonus to Saving Throws versus
	cold-based effects (e.g.~magic or breath attacks).
	\item
	\textbf{Cold-based damage}: Reduce cold damage by 1 per damage die
	rolled. (For example, 4d6 damage is reduced by 4.)
\end{itemize}

\paragraph{Lesser Healing (Breath of
	St.~Lillibeth)}\label{lesser-healing-breath-of-st.-lillibeth}\mbox{}\\
\emph{(R: Touch, D: Instant)}\\
The fluttering of doves' wings and the sweet scent of blossom manifest
as the caster recites this prayer. A living subject receives one of the
following ministrations:

\begin{itemize}
	\tightlist
	\item
	\textbf{Healing}: Restores 1d6+1 Hit Points. This cannot raise the
	subject's Hit Points above the normal maximum.
	\item
	\textbf{Curing paralysis}: Paralysing effects are negated.''
\end{itemize}

\paragraph{Light (St.~Foggarty's
	Benediction)}\label{light-st.-foggartys-benediction}\mbox{}\\
\emph{(R: 120', D: 12 Turns)}\\
A bobbing wisp of light floats from the caster's palm and manifests one
of the following effects:

\begin{enumerate}
	\def\labelenumi{\arabic{enumi}.}
	\tightlist
	\item
	\textbf{Golden radiance}: Casts holy light in a 15' radius. The light
	is sufficient for reading, but is not as bright as daylight. The spell
	may be cast upon an object, in which case the golden light moves with
	the object.
	\item
	\textbf{Blinding a creature}: A flash of divine light blinds a
	creature for the duration. The target may make a Willpower 14 save to
	resist.
	\item
	\textbf{Cancelling magical darkness}: St Foggarty's Benediction may
	cancel a 15' radius area of magical darkness.
\end{enumerate}

\paragraph{Mantle of Protection (St.~Benester's
	Word)}\label{mantle-of-protection-st.-benesters-word}\mbox{}\\
\emph{(R: Self, D: 12 Turns)}\\
Invoking the name of an archangel, the caster is warded from attacks.

\begin{itemize}
	\tightlist
	\item
	\textbf{AC and Saving Throw bonus}: The caster gains a +1 Armour Class
	and Saving Throw bonus against attacks and special powers.
	\item
	\textbf{Magically created or summoned creatures}: The prayer
	additionally prevents such creatures from making melee attacks against
	the caster, though they may still make ranged attacks. If the caster
	engages such a creature in melee, this protection is broken (though
	the caster still gains the Armour Class and Saving Throw bonuses
	mentioned above).
\end{itemize}

\paragraph{Purify Food and Drink (St.~Gretchen's
	Sublimation)}\label{purify-food-and-drink-st.-gretchens-sublimation}
\emph{(R: Touch, D: Permanent)}\\
The sound of distant goat bells echoes and a quantity of poisoned,
rotten, spoiled, or contaminated food and drink is purified.

\begin{itemize}
	\tightlist
	\item
	\textbf{Quantity}: Up to 12 portions of food and drink, in any
	combination. 1 ration counts as a portion of food and 1 pint counts as
	a portion of drink.
\end{itemize}

\paragraph{Rally (St.~Jorrael's
	Counsel)}\label{rally-st.-jorraels-counsel}\mbox{}\\
\emph{(R: Touch, D: 2 Turns)}\\
The emboldening words of this prayer reverberate in the subject's mind,
calming them and purging them of fear.

\begin{itemize}
	\tightlist
	\item
	\textbf{Magically induced fear}: Make a Willpower 14 saving throw with
	a +1 bonus per Level of the caster to counter magical terror. This
	applies to effects active when Rally is cast and subsequent effects
	during the duration.
\end{itemize}

\subsection*{Rank 2 Spells}\label{rank-2-holy}

\paragraph{Bless (Righteousness of
	St.~Gondyw)}\label{bless-righteousness-of-st.-gondyw}\mbox{}\\
\emph{(R: 60', D: 6 Turns)}\\
Accompanied by the triumphant blaring of trumpets, a surge of divine
righteousness bolsters the morale of allies within a 20' × 20' area.

\begin{itemize}
	\tightlist
	\item
	\textbf{Bonuses}: Subjects gain a +1 bonus to Attack and Damage Rolls.
	\item
	\textbf{Retainers}: Also gain a +1 bonus to Loyalty.
\end{itemize}

\paragraph{Charm Serpents (St.~Dank's
	Plea)}\label{charm-serpents-st.-danks-plea}\mbox{}\\
\emph{(R: 60', D: 1d4 Rounds or 1d4 Turns)}\\
This prayer hypnotises snakes; the serpents rear upright and sway to and
fro, but they never attack while charmed. Number of snakes affected:
Snakes whose Levels total up to twice the caster's Level. For example, a
Level 3 caster could affect three Level 2 snakes, two Level 3 snakes,
etc. The caster selects which individuals are affected. Duration: When
cast on snakes that are already attacking, the effect lasts for 1d4
Rounds. Otherwise, it lasts for 1d4 Turns.

\paragraph{Find Traps (Path of
	St.~Gripe)}\label{find-traps-path-of-st.-gripe}\mbox{}\\
\emph{(R: 30', D: 2 Turns)}\\
Trapped objects and areas within range of the caster are wreathed in an
apparition of faint, flickering flames.

\begin{itemize}
	\tightlist
	\item
	\textbf{Magical and mechanical traps}: All are revealed.
	\item
	\textbf{No disarm}: The spell provides no assistance bypassing or
	deactivating traps, only revealing their locations.
\end{itemize}

\paragraph{Flame Ward (Boldness of
	St.~Hollyhock)}\label{flame-ward-boldness-of-st.-hollyhock}\mbox{}\\
\emph{(R: 30', D: 2 Turns)}\\
A puff of flour and the aroma of freshly baked bread manifest around a
single creature of the caster's choosing, which is bestowed with
supernatural resistance to fire.

\begin{itemize}
	\tightlist
	\item
	\textbf{Normal heat}: The subject is unharmed by non-magical heat or
	fire.
	\item
	\textbf{Save bonus}: The subject gains a +2 bonus to Saving Throws
	versus fire-based effects (e.g.~magic or breath attacks). Fire-based
	damage: Reduce fire damage by 1 per damage die rolled. (For example,
	4d6 damage is reduced by 4.)
\end{itemize}

\paragraph{Hold Person (St.~Waylaine's
	Reproof)}\label{hold-person-st.-waylaines-reproof}\mbox{}\\
\emph{(R: 180', D: 9 Turns)}\\
The caster castigates one or more mortals, fairies, or demi-fey for
their misdeeds, causing them to halt in their tracks if they fail a
Willpower 14 saving throw. The spell may be used in two ways:

\begin{enumerate}
	\def\labelenumi{\arabic{enumi}.}
	\tightlist
	\item
	\textbf{Against an individual}: The target's Saving Throw is penalised
	by --2.
	\item
	\textbf{Against a group}: 1d4 individuals in the group are targeted.
	The individuals are selected by the caster.
\end{enumerate}

\begin{itemize}
	\tightlist
	\item
	\textbf{Restrictions}: Large creatures are immune.
	\item
	\textbf{Paralysis}: Affected targets are aware but cannot move their
	limbs. They may, at the caster's option, speak.
	\item
	\textbf{Freeing}: The caster may free paralysed targets with a word.
\end{itemize}

\paragraph{Reveal Alignment (St.~Willofrith's
	Warning)}\label{reveal-alignment-st.-willofriths-warning}\mbox{}\\
\emph{(R: 30', D: Instant)}\\
An angelic voice whispers in the caster's mind, revealing the Alignment
of a selected character, monster, object, or area within range. (Most
objects and areas do not have an Alignment, but magic items or holy
places may.)

\paragraph{Silence (Abjuration of
	St.~Signis)}\label{silence-abjuration-of-st.-signis}\mbox{}\\
\emph{(R: 180', D: 12 Turns)}\\
Raising a finger to their lips, the caster whispers a hushed prayer to
St Signis the Silent. A 15' radius area within range is shrouded in holy
silence.

\begin{itemize}
	\tightlist
	\item
	\textbf{Within the area}: All sound is prevented. Conversation and
	spell casting are impossible.
	\item
	\textbf{Noise from outside}: Those within the area of silence can only
	hear sounds from outside it.
	\item
	\textbf{Casting upon a creature}: Silence may be cast upon a creature,
	which must make an Arcana 14 saving throw. If the Saving Throw fails,
	the shrouded 15' radius area of silence moves with the target. If the
	Saving Throw succeeds, the spell has no effect.
\end{itemize}

\paragraph{Speak With Animals (Speech of
	St.~Hamfast)}\label{speak-with-animals-speech-of-st.-hamfast}\mbox{}\\
\emph{(R: Self, D: 6 Turns)}\\
Animals of one selected species prick up their ears to the lilting words
of this prayer. The caster can speak with all animals of that type for
the duration.

\begin{itemize}
	\tightlist
	\item
	\textbf{Type of animal}: The spell translates for one species of
	normal or giant animals at a time. Intelligent animals and fantastic
	monsters are not affected.
	\item
	\textbf{Services}: Animals which are friendly towards the caster may
	be persuaded to perform a service. The service must be within an
	animal's comprehension and capabilities.
\end{itemize}

\subsection*{Rank 3 Spells}\label{rank-3-holy}

\paragraph{Animal Growth (Mercy of
	St.~Vinicus)}\label{animal-growth-mercy-of-st.-vinicus}\mbox{}\\
\emph{(R: 120', D: 12 Turns)}\\
A single animal is briefly haloed with glimmering Mithric script (the
text of this prayer), turns its eyes toward the heavens, and doubles in
size.

\begin{itemize}
	\tightlist
	\item
	Damage: Double the damage of the animal's melee attacks.
	\item
	Load: Double the maximum weight the animal can carry.
	\item
	Restrictions: This spell may be used on normal or giant animals, but
	intelligent animals, magical animals, and fantastic monsters are
	unaffected.
\end{itemize}

\paragraph{Bless Weapon (Courage of
	St.~Sedge)}\label{bless-weapon-courage-of-st.-sedge}\mbox{}\\
\emph{(R: 30', D: 1 Turn)}\\
Silver light arcs from the caster's hand and wreathes a single weapon,
enchanting it with holy power.

\begin{itemize}
	\tightlist
	\item
	Damage: The weapon deals an additional damage die.
	\item
	Treated as magical: The weapon now harms monsters which can only be
	harmed by magic.
\end{itemize}

\paragraph{Cure Affliction (St.~Pastery's
	Blessing)}\label{cure-affliction-st.-pasterys-blessing}\mbox{}\\
\emph{(R: 30', D: Instant)}\\
The lowing of cattle can be heard as the caster recites this prayer. A
living subject receives one of the following ministrations:

\begin{enumerate}
	\def\labelenumi{\arabic{enumi}.}
	\tightlist
	\item
	\textbf{Cure disease}: The subject is purged of a single disease of
	mundane or magical origin.
	\item
	\textbf{Cure blindness or deafness}: The subject's sight or hearing
	are restored, nullifying any disease, curse, or enchantment that
	veiled them.
\end{enumerate}

\paragraph{Holy Light (Devotion of
	St.~Eggort)}\label{holy-light-devotion-of-st.-eggort}\mbox{}\\
\emph{(R: 120', D: Permanent or 6 Turns)}\\
A ray of white light beams from the caster's raised hand and manifests
one of the following effects:

\begin{enumerate}
	\def\labelenumi{\arabic{enumi}.}
	\tightlist
	\item
	\textbf{Eternal radiance}: Casts holy light in a 30' radius. The light
	is as bright as full daylight (creatures that suffer penalties in
	daylight are affected). The prayer may only be cast upon a fixed
	location within range---the eternal radiance does not move. The
	duration is permanent.
	\item
	\textbf{Beacon of sanctity}: White light radiates in a 15' radius
	around a chosen object. Undead creatures suffer a --2 penalty to
	Attack Rolls and Saving Throws while within the radius of light. The
	duration is 6 Turns.
	\item
	\textbf{Cancelling magical darkness}: Devotion of St Eggort may cancel
	a 30' radius area of magical darkness.
\end{enumerate}

\paragraph{Locate Object (St.~Keye's
	Revelation)}\label{locate-object-st.-keyes-revelation}\mbox{}\\
\emph{(R: 120', D: 6 Turns)}\\
A cherub-like apparition of the infant St Keye points in the direction
of a sought object, adjusting the indicated direction as the caster
moves. The cherub disappears when the sought object is within range. One
of two types of objects may be located:

\begin{enumerate}
	\def\labelenumi{\arabic{enumi}.}
	\tightlist
	\item
	\textbf{General class}: An object of a general class (e.g.~a stairway,
	an altar, etc.). In this case, the nearest object of that type is
	located.
	\item
	\textbf{Specific object}: A specific object which the caster can
	clearly visualise in all aspects. Restrictions: This spell cannot be
	used to locate creatures.
\end{enumerate}

\paragraph{Remove Curse (St.~Primula's
	Grace)}\label{remove-curse-st.-primulas-grace}\mbox{}\\
\emph{(R: Touch, D: Instant)}\\
Milk drips from the caster's hand as they recite this prayer. The
subject, doused in holy milk, is permanently cured of the affliction of
a single curse.

\begin{itemize}
	\tightlist
	\item
	\textbf{Cursed items}: Use of this spell enables a character to
	discard a cursed magic item.
\end{itemize}

\subsection*{Rank 4 Spells}\label{rank-4-holy}

\paragraph{Circle of Protection (St.~Faxis'
	Abjuration)}\label{circle-of-protection-st.-faxis-abjuration}\mbox{}\\
\emph{(R: 10' around the caster, D: 12 Turns)}\\
A sword of glowing light manifests and inscribes a 10' radius circle
around the caster, leaving a shimmering ring of light which moves with
the caster. The caster and all allies within the circle come under the
effect of a Mantle of Protection, warded against attacks.

\begin{itemize}
	\tightlist
	\item
	\textbf{AC and Saving Throw bonus}: Those warded gain a +1 Armour
	Class and Saving Throw bonus against attacks and special powers.
	\item
	\textbf{Magically created or summoned creatures}: The prayer
	additionally prevents such creatures from making melee attacks against
	those warded, though they may still make ranged attacks. If any of the
	warded party engages such a creature in melee, this protection is
	broken (those warded still gain the Armour Class and Saving Throw
	bonuses mentioned above).
\end{itemize}

\paragraph{Create Water (St.~Quister's
	Defence)}\label{create-water-st.-quisters-defence}\mbox{}\\
\emph{(R: Touch, D: Permanent)}\\
A magical fount of wine-red water springs forth from the ground or a
wall. The liquid looks and tastes like a sweet, refreshing wine, but it
is actually pure water.

\begin{itemize}
	\tightlist
	\item
	\textbf{Volume}: The fount produces enough to sustain 12 people and 12
	mounts for one day (approximately 50 gallons).
	\item
	\textbf{Higher Level casters}: If the caster is higher than Level 9,
	water sufficient for an additional 12 people and mounts is produced
	for each Level beyond 9.
\end{itemize}

\paragraph{Greater Healing (Steadfastness of
	St.~Wick)}\label{greater-healing-steadfastness-of-st.-wick}\mbox{}\\
\emph{(R: Touch, D: Instant)}\\
The rustic voice of St Wick manifests, whispering a parable as the
caster touches a living subject.

\begin{itemize}
	\tightlist
	\item
	\textbf{Healing}: The prayer restores 2d6+2 Hit Points to the subject.
	This cannot raise the subject's Hit Points above the normal maximum.
\end{itemize}

\paragraph{Remove Poison (St.~Torphia's
	Respite)}\label{remove-poison-st.-torphias-respite}\mbox{}\\
\emph{(R: Touch, D: Instant)}\\
The rattling of manacles echoes as a warm light envelops the caster's
hand. The light neutralises poison in one touched creature or object.

\begin{itemize}
	\tightlist
	\item
	\textbf{Living creature}: Neutralise the effects of poison on a living
	subject. A subject that has died from poisoning can be revived
	(restored to 1 Hit Point) if Remove Poison is cast within 10 Rounds.
	\item
	\textbf{Object or substance}: Remove poison from an item, food, or
	liquid.
\end{itemize}

\paragraph{Speak With Plants (Salvation of
	St.~Wort)}\label{speak-with-plants-salvation-of-st.-wort}\mbox{}\\
\emph{(R: Self, D: 3 Turns)}\\
A vision of a great yew tree briefly manifests behind the caster, its
limbs arching protectively overhead. The caster gains the miraculous
ability to communicate with plants.

\begin{itemize}
	\tightlist
	\item
	\textbf{Normal plants}: Flora informs the caster of creatures that
	have recently passed or performs simple favours. For example, plants
	may clear a passageway for the caster's party to pass through or
	writhe and tangle into an impassable thicket to hinder pursuers.
	\item
	\textbf{Monstrous plants}: The caster can communicate with plant-like
	or plant-based monsters.
\end{itemize}

\paragraph{Serpent Transformation (St.~Horace's
	Requital)}\label{serpent-transformation-st.-horaces-requital}\mbox{}\\
\emph{(R: 120', D: 6 Turns)}\\
Accompanied by the merry laughter of St Horace and the waft of freshly
unearthed mushrooms, 2d8 normal sticks leap to attention and
miraculously transform into adders.

\begin{itemize}
	\tightlist
	\item
	\textbf{Commands}: The snakes follow the caster's orders for the
	duration of the spell.
	\item
	\textbf{Reversion}: When the spell ends or a snake is killed, the
	serpent reverts back to a normal stick.
\end{itemize}

\textbf{Adder} Small Animal---Animal Intelligence---Neutral\\
Level 1 AC 13 HP 1d8 (4) Att Bite (+0, 1d3 + poison) Speed 30 Morale 7
XP 15\\
Poison: Con 11 saving throw or suffer 1 damage per Round for the next
1d6 Rounds.\\
When killed: The snake reverts back to a normal stick.

\subsection*{Rank 5 Spells}\label{rank-5-holy}

\paragraph{Communion (St.~Elsa's
	Visitation)}\label{communion-st.-elsas-visitation}\mbox{}\\
\emph{(R: Self, D: 3 Turns)}\\
This prayer allows the caster to enter into communion with one of the
ascended saints, asking questions and receiving answers relaying the
divine wisdom of the One True God. The spell must be cast upon the
saint's feast day.

\begin{itemize}
	\tightlist
	\item
	\textbf{Trance}: For the duration of the spell, the caster is in a
	spiritual trance, unaware of the world around them.
	\item
	\textbf{Questions}: The caster may ask 3 questions of the saint.
	\item
	\textbf{Answers}: Each question receives a simple ``yes'' or ``no''
	answer. The answers are guaranteed to be true.
	\item
	\textbf{Usage limit}: Communion may be cast once per week. The Referee
	can limit it to once per month if overused.
	\item
	\textbf{St Clewyd}: For theologically unclear reasons, St Clewyd does
	not respond to this spell.
\end{itemize}

\paragraph{Create Food (St.~Ponch's
	Feast)}\label{create-food-st.-ponchs-feast}\mbox{}\\
\emph{(R: Appears in the caster's presence, D: Permanent)}\\
Barrels and hampers of fresh food wash up on a foaming swell of
seawater. The brine disappears instantly, but the food remains.

\begin{itemize}
	\tightlist
	\item
	\textbf{Volume}: Food sufficient for 12 people and 12 mounts is
	conjured---enough to last one day.
	\item
	\textbf{Higher Level casters}: If the caster is higher than Level 9,
	food sufficient for an additional 12 people and mounts is produced for
	each Level beyond 9.
\end{itemize}

\paragraph{Holy Fire (St.~Goodenough's
	Rebuke)}\label{holy-fire-st.-goodenoughs-rebuke}

\emph{(R: 30' or Self or Touch, D: Concentration up to 1 Turn or
	Instant)}\\
A column of holy fire surrounds the caster, channelling divine grace for
one of three purposes:

\begin{enumerate}
	\def\labelenumi{\arabic{enumi}.}
	\tightlist
	\item
	\textbf{Circle of warding}: By concentrating and remaining stationary,
	the caster creates a ward against undead and creatures created or
	summoned by magic. Any such creature that comes within 30' must make a
	Willpower 14 saving throw or be destroyed or banished to its place of
	origin. If a monster's save succeeds, it flees the warded area.
	\item
	\textbf{Target single monster}: Instantly banish or destroy a single
	magically-summoned, created, or undead creature within range. The
	monster may make a Willpower 14 saving throw (with a --2 penalty) to
	avoid banishment or destruction. If the monster's save succeeds, it
	flees the affected area.
	\item
	\textbf{Restoration}: Touching a subject instantly removes a curse,
	disease, or other affliction.
\end{enumerate}

\paragraph{Holy Quest (Righteousness of
	St.~Galaunt)}\label{holy-quest-righteousness-of-st.-galaunt}\mbox{}\\
\emph{(R: 30', D: Until quest is completed)}\\
Accompanied by a clap of thunder and a ray of holy light, the caster
commands a single subject to perform a specific quest or task.

\begin{itemize}
	\tightlist
	\item
	\textbf{Example quests}: Rescuing a prisoner, killing a specific
	monster, bringing a magic item to the caster, going on a journey to a
	holy site.
	\item
	\textbf{Suicidal quests}: The prescribed quest must not be obviously
	suicidal.
	\item
	\textbf{Refusal}: The subject must undertake the quest or suffer a --2
	penalty to Attack Rolls and Saving Throws.
	\item
	\textbf{Completion}: Once the task is completed, the spell ends.
	\item
	\textbf{Saving throw}: The subject may make a Willpower 14 saving
	throw to resist the holy compulsion, negating the effects of the
	spell.
\end{itemize}

\paragraph{Insect Plague (Amity of
	St.~Cornice)}\label{insect-plague-amity-of-st.-cornice}\mbox{}\\
\emph{(R: 360', D: 1 Turn per Level)}\\
A writhing, 60' diameter swarm of biting insects manifests at a location
within range.

\begin{itemize}
	\tightlist
	\item
	\textbf{Movement}: The swarm does not move from the area it is
	summoned in.
	\item
	\textbf{Inside the swarm}: Vision is limited to 30'. Biting insects
	inflict 1 damage per Round on all creatures.
	\item
	\textbf{Creatures of Level 1--2}: Low-level creatures caught within
	the swarm flee in horror and only stop when at least 240' away.
\end{itemize}

\paragraph{Raise Dead (Mercy of
	St.~Clewyd)}\label{raise-dead-mercy-of-st.-clewyd}\mbox{}\\
\emph{(R: 120', D: Instant)}\\
The ultimate miracle---by the grace of God, a deceased person may be
returned to life.

\begin{itemize}
	\tightlist
	\item
	\textbf{Restrictions}: Mortals may be raised. Fairies and demi-fey may
	not, the fate of their souls being outside of the purview of the
	Pluritine Church.
	\item
	\textbf{Ritual}: Raise Dead requires a grandiose ritual in a cathedral
	(e.g.~in Castle Brackenwold), the participation of a dozen priests,
	and the burning of rare incenses to the value of 2,000gp. Despite its
	title, the ritual makes no reference to St Clewyd.
	\item
	\textbf{Time limit}: A person deceased for no longer than 2 days per
	Level of the caster can be raised. (e.g.~a Level 10 caster can revive
	someone who has been dead up to 20 days.)
	\item
	\textbf{Weakness}: Returning from death is an ordeal. Until the
	subject gets 2 full weeks of bed rest, they have 1 Hit Point, move at
	half Speed, cannot carry heavy items, and cannot attack, cast spells,
	make Skill Checks, or use other Class capabilities. Recovery from this
	weakness cannot be hastened by magic of any kind.
\end{itemize}

\end{multicols}

\end{adjustwidth}


\end{document}
