\documentclass{article}
\usepackage{fontspec}
\usepackage{geometry}
\usepackage{multicol}
\usepackage{titlesec}
\usepackage{hyperref}
\usepackage{enumitem}
\usepackage{booktabs}
\usepackage{longtable}
\usepackage{array}
\usepackage{tabularx}

\geometry{margin=1in}
\setlength{\columnsep}{20pt}

\setmainfont{Minion Pro}
\newfontface\headerfont{Futura}

\titleformat{\section}{\headerfont\Large\bfseries}{\thesection}{1em}{} % h1
\titleformat{\subsection}{\headerfont\large\bfseries}{\thesubsection}{1em}{} % h2
\titleformat{\subsubsection}{\headerfont\normalsize\bfseries}{\thesubsubsection}{1em}{} % h3
\titleformat{\paragraph}{\headerfont\normalsize\bfseries}{\theparagraph}{1em}{} % h4
\titleformat{\subparagraph}{\headerfont\normalsize\bfseries}{\thesubparagraph}{1em}{} % h5

\newcommand{\newpageh1}{\newpage\section}

\begin{document}

\begin{multicols}{2}

\section{Purple Hack}\label{purple-hack}

\begin{quote}
\emph{The following are ``house rules'' and clarifications, a mish-mash
of D\&D Basic/Expert a la OSE, and a fantastic ruleset called
\href{https://osrsimulacrum.blogspot.com/2021/06/simulacrum-beta-release.html}{Simulacrum}.
Rampant stealing has occured below, and is intended only for use in my
own home games.}
\end{quote}

\section{Character Creation}\label{character-creation}

\subsection{Roll Your Stats}\label{roll-your-stats}

Roll 3d6 down the line for each of the skills below. You can swap one
set of stats.

\begin{itemize}
\tightlist
\item
  \textbf{Strength} measures physical might. It affects your lifting,
  carrying, and grappling. Any Str modifier is applied to your melee and
  thrown weapon attack damage.
\item
  \textbf{Dexterity} measures agility, reflexes, and hand-eye
  coordination. Any Dex mod is applied to your Armor Class, as long as
  you can move freely.
\item
  \textbf{Constitution} measures endurance and vitatliy. Any Con mod is
  applied to your Hit Point total at the start of the game and each time
  you go up a level.
\item
  \textbf{Perception} measures your observational acuity. Any Per mod
  affects your ability to see through illusions, and to spot traps,
  secret doors, and ambushes.
\item
  \textbf{Willpower} measures discipline and mental endurance. Any Will
  mod affects your ability to resist enchantment, fear, mind control,
  possession, and other mental attacks, as well as to avoid death.
\item
  \textbf{Arcana} measures your connection to magical forces. Any Arc
  mod is applied to chanes to learn spells and to saving throws to
  resist spell damage.
\end{itemize}

\subsection{Choose a Character Race \& Make Any
Adjustments}\label{choose-a-character-race-make-any-adjustments}

\begin{itemize}
\tightlist
\item
  \textbf{Humans}: The most flexible \& common. +1 to one stat of your
  choice.
\item
  \textbf{Dwarf}: +1 Con, -1 Dex. Add Con mod to saves vs poison. Cannot
  wield large weapons.
\item
  \textbf{Elf}: +1 Per, -1 Str. 90\% resistance to sleep and charm
  spells. Immune to ghoul paralysis.
\item
  \textbf{Half-Elf}: 30\% resistance to sleep and charm spells.
\item
  \textbf{Halfling}: -1 Str, +1 Dex. +1 bonus to missile weapon attacks.
  +2 AC vs large enemies. Cannot wield large weapons.
\end{itemize}

\subsection{Note Ability Score
Modifiers}\label{note-ability-score-modifiers}

\begin{longtable}[]{@{}cc@{}}
\toprule\noalign{}
Ability Score & Modifier \\
\midrule\noalign{}
\endhead
\bottomrule\noalign{}
\endlastfoot
2-3 & -3 \\
4-5 & -2 \\
6-8 & -1 \\
9-12 & 0 \\
13-15 & +1 \\
16-17 & +2 \\
18-19 & +3 \\
\end{longtable}

\subsection{Choose Your Class \& Note Hit
Die}\label{choose-your-class-note-hit-die}

\subsubsection{Warrior}\label{warrior}

1d8 Hit Die (8hp+Con mod at 1st level). Receive attack bonus of +1 every
level, starting at level 1, stopping at +15 at level 15. Can use any
weapon or armor, so long as they meet its strength minimum, if any (see
\hyperref[armor-weapons-and-equipment]{equipment}). Every name level,
warriors add an extra weapon die of damage to their armed combat attacks
and an additional feat. At 1st level, warriors select one style,
reflecting their preferred manner of fighting:

\begin{itemize}
\tightlist
\item
  \textbf{Arcanist}: You can read Mithric and gain access to two schools
  of magic of your choice. You can cast spells in armor as well as when
  being jostled. You cannot create or cast from scrolls, though you can
  copy spells from them. You have half the spell slots of a mage of your
  same level (rounded down, to a minimum of 1 slot). You start with a
  spellbook and spells the same way a mage does, but no schools or
  spells are automatically acquired through gaining levels. The source
  of your power must be defined.
\item
  \textbf{Hordeslayer}: If you kill an opponent with an attack, you can
  immediately make a bonus attack of the same kind (melee or ranged).
  You can make a maximum number of melee bonus attacks per round equal
  to your level, and a maximum number of missile bonus attacks equal to
  your number of name levels plus 1.
\item
  \textbf{Smiter}: Once per combat encounter, you can declare a smite
  after you score a hit, melee or ranged, which doubles the number of
  damage dice rolled. If a smite was announced on a critical hit, the
  extra weapon dice do not automatically deal maximum damage.
\end{itemize}

\subsubsection{Mage}\label{mage}

1d6 Hit Die (6hp+Con mod at 1st level). Can cast one spell per round.
Can use any weapon or armor, so long as they meet its strength minimum,
if any, but they can't normally cast spells in armor. Receive attack
bonus of +1 every two levels, starting at level 2. All spells are
divided into eight schools (see \hyperref[magic-spells]{Magic \&
Spells}). By default, mages have access to four schools (one chosen, and
three rolled randomly). After rolling, a mage can trade away access to
one school (up to two maximum) in exchange for one of the following
special abilities (each can be chosen only once, and only at character
creation):

\begin{itemize}
\tightlist
\item
  \textbf{Battlemage}: You can cast spells while wearing light armor,
  and your class attack bonus improves to +1 every level, starting at
  level 1, stopping at +15.
\item
  \textbf{Focused}: Select one additional mage feat.
\item
  \textbf{Innate}: While you can still use them, you need not own a
  spellbook or consult one to prepare your spells.
\item
  \textbf{Specialist}: Choose one school you can already access. You
  have one additional spell slot at each spell level, to fill with a
  spell from this school. When you gain a level, you learn one extra
  random new spell from this school. Spells from this school are easier
  to bind.
\end{itemize}

\textbf{Starting Spells}: Mages start with one random 1st-level spell
from each school to which they have access. Specialists then select one
extra 1st-level spell of their choice from their specialty school. Mages
can prepare a limited number of spells each day, and gain further spell
slots as they gain levels (see \hyperref[magic-spells]{Magic \&
Spells}).

\textbf{New Spells}: Mages learn one random new spell each time they
gain a level. You also gain access to a new school of your choice at
each name level. When this happens, you learn random spells of that
school, one at each spell level you can cast.

\subsection{Note Your Saving Throw
Bonus}\label{note-your-saving-throw-bonus}

Saving throws are based on a 1d20 roll, adding the relevant stat
modifier, and adding your saving throw bonus. At 1st level, this bonus
is 0, but goes up by +1 every even level (so +1 at level 2, +2 at level
4, etc.).

The default save is Hard (14+ to succeed); instant-death effects are
usually Daunting (11+). A natural 20 always saves, and a natural 1
always fails.

\subsection{Choose Alignment}\label{choose-alignment}

Your character may align themselves with one of each of the great cosmic
ideologies (and the forces that lie behind them): either Law or Chaos,
and either Good or Evil. A character may choose just one alignment
(e.g., a commitment to Law, but neither Good nor Evil), or one of each
(i.e., Lawful Good, Lawful Evil, Chaotic Good, or Chaotic Evil).

\begin{itemize}
\tightlist
\item
  \textbf{Law}: Fervently believe in order and stability. Disorder is
  anathema.
\item
  \textbf{Chaos}: Thrive on mutability and change. Stagnation is
  anathema.
\item
  \textbf{Good}: Fervently believe in altruism, compassion, and justice.
\item
  \textbf{Evil}: The only restraints are imposed restraints; actions are
  usually weighed solely on how they benefit oneself.
\end{itemize}

Characters may instead forego an alignment and choose to be
\textbf{Unaligned}. This does not mean the person is neutral or
incapable of taking a position on something. Neither does it mean that
they cannot serve the cause of the aligned, simplyt aht they lack their
fervent, formal commitment.

Players can choose to conceal their choice of alignment (or lack
thereof) from outsiders or even the rest of their party, but must tell
the referee what they've picked.

\subsection{Choose Languages}\label{choose-languages}

You start play knowing your native language(s). Additionally, roll 1d6.
On a 5, you gain one more language. On a 6, you gain two more languages.
Mages also know Mithric (the language of magic), plus one additional
language of their choice.

\subsection{Choose Feat(s)}\label{choose-feats}

All characters receive one feat at 1st level, and another feat at each
name level. A feat can only be taken once, unless noted otherwise. No
ability score can be raised above 19 through feats.

\subsubsection{Warrior feats}\label{warrior-feats}

\begin{itemize}
\tightlist
\item
  \textbf{Brawler}: If your melee opponent is no more than one size
  level larger than you, and your attack roll against them is a natural
  18 or 19, then in addition to your regular damage you can either:

  \begin{enumerate}
  \def\labelenumi{\arabic{enumi}.}
  \tightlist
  \item
    Disarm them
  \item
    Trip them (they become prone; two-legged creatures only)
  \item
    Drive them directly back 5 feet, if the space is available to do so.
    You may follow up immediately with a free 5-foot move of your own,
    even if you have already made your full move this turn or are
    otherwise locked in combat.\\
    If none of those are possible or desired, then instead you bash them
    for an additional 1d4 damage (+2 per name level).
  \end{enumerate}
\item
  \textbf{Captain}: Some are born to command. Your party gains a +1
  initiative bonus. Add a +1 attack bonus to all other party members and
  associated NPCs, raised to +2 at level 10 or higher (this does not
  benefit yourself). Apply +2 to friendly Morale checks. These bonuses
  apply only as long as your orders can be understood and the
  individuals benefitting are willing to be led by you. Multiple
  captains in a group do not stack these benefits.
\item
  \textbf{Defender}: If you decide that none will pass, then \emph{none
  will pass}. You ignore all magical commands to move aside, flee,
  surrender and the like, and are immune to all fear-based effects,
  magical or not. You gain a 4-point modifier in your favor when
  resisting any other effect that would result in you being
  involuntarily moved. Also, in combat, you can always choose to receve
  the effects of the Guard combat stance, even if using another stance.
  If you actually choose the Guard stance, you can intercept up to four
  enemies instead of two.
\item
  \textbf{Great-Weapon Fighter}: When attacking with a two-handed melee
  weapon, your critical hit range improves by 1, plus 1 per name level
  (e.g.~you score critical hits on a natural to-hit roll of 19-20 at
  level 1, 18-20 at level 5, etc.).
\item
  \textbf{Marksman}: If firing into melee, you can pick your target
  instead of rolling randomly. Your ranged to-hit penalties are reduced
  by 2 points, plus 2 per name level. Your rate of fire with small
  thrown weapons inscreases from 1 to 2.
\item
  \textbf{Read Scrolls}: You can read Mithric, as well as cast from
  scrolls containing spells from four schools of your choice; if an
  Arcanist, two of these schools must be the two schools you already
  know. This does not grant the ability to create scrolls.
\item
  \textbf{True Grit}: You may reroll failed death saves.
\item
  \textbf{Whirlwind}: +2 AC (and a further +1 AC per name level). If not
  casting a spell that round, you receive the effects of the Dash combat
  stance (this does not count as your stance pick for the round). You
  are not locked in melee combat unless in melee with at least three
  opponents. To gain these benefits, you must be able to move freely,
  not wearing medium or heavy armor, and cannot be encumbered.
\end{itemize}

\subsubsection{Mage feats}\label{mage-feats}

A mage can combine multiple feats on a spell. The spell adjustments from
these stack (e.g.~a \emph{Silent Magic Missile} spell with extended
range would in all ways be treated as a 3rd level spell). A feat cannot
take a spell over 6th level. Note that a spell cannot be both Silent and
Stilled.

\begin{itemize}
\tightlist
\item
  \textbf{Concentration}: You can cast spells while being jostled
  (e.g.~on a ship or a horse).
\item
  \textbf{Dextrous}: You can cast 1st-level spells using only one hand.
  For each name level you have, the level of spells that can be cast in
  this way increase by one.
\item
  \textbf{Familiar}: You acquire a Tiny or Small mundane creature
  appropriate to the area that obeys your commands. You can see through
  its eyes, and gain a small power appropriate to the creature while
  doing so (e.g.~+2 visual Perception for a bird). Regardless of its
  normal statblock, the creature has 2 Hit Dice. Its death applies one
  level of fatigue to you for the next week, after which you may take a
  new familiar.
\item
  \textbf{Metamagic}: You can double the duration of your spells (not
  concentration, permanent, or instant spells), or you can double the
  base range of spells with Short, Medium or Long range (range increases
  due to gaining levels are unaffected). Preparing a Metaspell uses up a
  spell slot one level higher than the spell's normal level for each
  alteration you make (e.g.~duration is +1 spell level; duration \&
  range is +2).
\item
  \textbf{Quickcast}: Casting times of your spells are reduced by 1 (see
  \hyperref[combat-phases]{magic phase}). This can take them to - or
  below. This feat may be taken up to twice.
\item
  \textbf{Spell Silence}: You can craft your spells to require no vocal
  component. Preparing a Silent spell uses up a spell slot one level
  higher than the spell's normal level. This feat may be taken twice.
  The second time removes the vocal component from your spells
  permanently, with no level adjustment to them required.
\item
  \textbf{Stillcasting}: You can craft your spells to require no somatic
  components. Preparing a Stilled spell uses up a spell slot one level
  higher than the spell's normal level. This feat may be taken twice.
  The second time removes the somatic component from your spells
  permanently, with no level adjustment to them required.
\item
  \textbf{Undeniable}: Once an Undeniable spell is declared, loss of
  concentration does not disrupt it (unless you are killed or otherwise
  rendered incapable of casting). All other restrictions apply.
  Preparing an Undeniable spell uses up a spell slot one level higher
  than the spell's normal level.
\end{itemize}

\subsubsection{Unrestricted feats}\label{unrestricted-feats}

\begin{itemize}
\tightlist
\item
  \textbf{Anointed}: You are zealous for your god or extra-planar
  benefactor. Spells or effects targeting your alignment always affect
  you, even if they normally only affect creaturs that are supernatural
  (e.g.~\emph{Protection from Law}). You cannot be Unaligned.

  \begin{itemize}
  \tightlist
  \item
    Your weapon attacks against creatures of one opposing alignment that
    are undead or extraplanar add +1 weapon die of damage (add/gain
    another if level 10 or higher). This always counts as a magical
    attack.
  \item
    When resisting spells and effects that would place you in direct
    conflict with your patron's known tenets and goals, the saving throw
    difficulty is one level lower.
  \item
    Gain a small power appropriate to your patron (e.g.~healing touch
    once per day for a god of life or healing).
  \end{itemize}
\item
  \textbf{Conditioning}: Gain +2 to a chosen ability score. This feat
  can be taken only once for a given ability score.
\item
  \textbf{Fieldcraft}: Gain +1 Constitution. Pick two broad terrain
  types (forest, desert, swamps, jungle, tundra, mountains, etc.). In
  these terrain types, your difficulties for tasks such as stealth,
  tracking, and concealing tracks are lowered, and you:

  \begin{itemize}
  \tightlist
  \item
    Heal +2 HP per day, and still heal even if marching in terrain that
    normally prevents such.
  \item
    Succeed more often when hunting.
  \item
    Are less likely to get lost, have random encounters, or be
    surprised.
  \end{itemize}
\item
  \textbf{Lockpicking}: Gain +1 Dexterity. You have the tools and
  expertise to pick locks. Your search for traps on a lock are one
  difficulty level lower.
\item
  \textbf{Tough}: Your hit die is one die higher (i.e.~a warrior will
  use a d10, and a mage will use a d8).
\end{itemize}

\subsection{Choose Skills}\label{choose-skills}

Skills represent specialized knowledge of or training in a particular
field. All characters select two skills at 1st level, and another at
each name level.

It's assumed that characters possess all the skills and knowledge
appropriate to their background, along with a host of everyman abilities
that nearly all posses, such as climbing, hiding, or moving silently.
Skills for adventurers' purposes are almost always specialized (like
reading lips) or non-intuitive (like lockpicking), and reflect special
training, deep knowledge, or intense \& focused practice in that
particular narrow area. Broad interaction (e.g.~Conversation), social
skills that achieve a result otherwise possible only through roleplay
(e.g.~Deception or Intimidation), skills that boil down a non-linear
non-standard task down to a simple roll (e.g.~Dungeoneering or
Investigation) or skills that allow for ready identification of magical
items cannot be taken. \textbf{Skills are largely meant to measure a
proficiency over and above what one could reasonably be expected to
have, NOT to define what is possible!}

Concretely, skills might, depending on the circumstances, allow you to:

\begin{itemize}
\tightlist
\item
  Avoid what might otherwise involve a roll.
\item
  Lower the difficulty of a check or negate penalties.
\item
  Lessen the consequences of failure.
\item
  Gain information not obvious to the average observer.
\end{itemize}

Sample skills include: Acrobatics, Blindfighting, Climbing, Disguise,
Etiquette, Gambling, Herbalism, Jumping, Language (\emph{choose one}),
Lockpicking (\emph{a character cannot have both the skill and the
feat}), Lore (\emph{specific subject}), Performing, Pick Pockets, Read
Lips, Riding, Running, Seafaring, Shadowing, Stealth, Swimming,
Tracking, Wrestling.

\subsection{Buy Equipment}\label{buy-equipment}

Roll 6d6 and multiply the number by 10. That's your starting silver
pieces (sp). See the \hyperref[armor-weapons-and-equipment]{equipment}
list for things to buy (don't forget armor!) or choose a
\hyperref[quick-packs]{quick pack}. Make a note (an asterisk is a good
way) of what items you're declaring as ``readied'' (see
\hyperref[readied-items]{readied items}).

If you plan to cast Turn Undead, be sure to buy a holy symbol! If you're
a thief, you'll want lockpicks. A magic-user needs a book to use as a
spellbook.

\subsection{Note a Motivation or
Motto}\label{note-a-motivation-or-motto}

\emph{Why is your character risking their life for adventure? This can
be one word, or a short sentence. Examples:}

\begin{itemize}
\tightlist
\item
  Rejected from his clan, Gorend is on a mission to prove himself.
\item
  Percival always has to be the hero.
\item
  Cedric can't resist a good story.
\item
  Spread the faith
\item
  Earn glory
\item
  Amass wealth
\item
  Take revenge
\item
  Master a skill
\item
  Obey duty
\item
  Discover truth
\item
  Do good
\item
  Help others
\item
  Instill chaos
\end{itemize}

\begin{center}\rule{0.5\linewidth}{0.5pt}\end{center}

\section{Armor, Weapons, and
Equipment}\label{armor-weapons-and-equipment}

\subsection{Item Slots}\label{item-slots}

We're using ``item slot'' encumbrance, measuring both weight and
awkwardness.

Assuming proper carrying gear, you can carry 10 + your Str mod slots of
items.

\begin{itemize}
\tightlist
\item
  \textbf{Small} items (like chalk or potions) fit four to a slot.
\item
  \textbf{Medium} items (most things) are 1 slot.
\item
  \textbf{Large} items like 2h weapons and medium armor are 2 slots.
  Heavy armor takes 3 slots and is bulky (see below).
\end{itemize}

Coins \& gems stack 500 to a slot. A typical body (willing or
unconscious), fills 9 slots, before gear, and is bulky.

\subsection{Encumbrance Effects}\label{encumbrance-effects}

\begin{longtable}[]{@{}cccc@{}}
\toprule\noalign{}
Slots Over & Encumbrance Level & Combat Speed & Hex Point Allowance \\
\midrule\noalign{}
\endhead
\bottomrule\noalign{}
\endlastfoot
0 & -- & Full & -- \\
1-3 & Light & -25\% & -1 \\
4-6 & Medium & -50\% & -2 \\
7-10 & Heavy & -75\% & -3 \\
11+ & Immobile & 0 & N/A \\
\end{longtable}

Each encumbrance level also raises the difficulty of applicable tasks,
such as climbing, by one level.

\textbf{Bulky}: Bulky items, such as worn heavy armor, impose a greater
encumbrance than just item points. Each automatically adds one
encumbrance level \emph{after} calculating the bearer's item slot load.

\subsection{Basic Gear}\label{basic-gear}

The following base gear is automatically added to your character for
free and does not count against your slots:

\begin{itemize}
\tightlist
\item
  One weapon of your choice (+ 20 arrows/bolts if missile weapon)
\item
  Backpack, pouches, waterskin, tinderbox (for lighting torches or small
  fires)
\item
  Misc tiny items (within reason); worn items/clothing
\end{itemize}

After this, you can have whatever basic gear you wish (within reason, as
determined by the referee), which does fill slots normally. Some common
basic adventuring items include but are not limited to (size listed
after each):

\begin{itemize}
\tightlist
\item
  \textbf{Bedroll (M)}: A blanket that can double as a sleeping bag.
  Helps protect against the cold when
  \hyperref[sleeping-in-the-wilds]{sleeping in the wilds}.
\item
  \textbf{Caltrops, bag (M)}: Small metal spikes sufficient to cover a
  5' × 5' area. Creatures moving through have a 2-in-6 chance of
  treading on a spike for a 50\% penalty to movement rate for 24 hours
  (or until magically healed). Intelligent creatures can move cautiously
  through areas with known caltrops, which requires their entire
  Movement Phase to travel 5', but eliminates any risk of impalement.
\item
  \textbf{Candle (S)}: Dimly illuminates a 5' radius, and burns for 4
  hours.
\item
  \textbf{Chalk (S)}: Useful for marking trails in dungeons.
\item
  \textbf{Crowbar (M)}: Makes it easier to force open doors, and doubles
  as an improvised medium weapon.
\item
  \textbf{Hammer (S)}: Doubles as an improvised small weapon.
\item
  \textbf{Iron spike (S)}: One of these can be hammered in to block one
  typical door.
\item
  \textbf{Lantern (M)}: A lantern can be closed to hide its light, burns
  one flask of lamp oil every 4 hours, and illuminates a 20' radius.
\item
  \textbf{Oil flask (M)}: Fuels lantern 4 hours. Poured on ground and
  lit burns for 1 turn. Thrown on monster (roll to hit) \& set on fire
  does 1d6 damage each round for two rounds.
\item
  \textbf{Pickaxe (L)}: This excavating tool doubles as an improvised
  medium weapon.
\item
  \textbf{Pole, 10' (L)}: Often used to prod potential dangers at a safe
  distance. When wielded during cautious exploration, has a 2-in-6
  chance of setting off most traps.
\item
  \textbf{Rations, iron (M)}: One day of food and water. Also called
  trail rations. As long as kept dry, will keep almost indefinitely.
\item
  \textbf{Rations, standard (L)}: One day of food and water. Fresh, and
  will not keep for more than 1d4 days, fewer if in poor conditions like
  high heat. The type of rations that hunting provides. More appealing
  to monsters than iron rations if dropped during pursuit.
\item
  \textbf{Rope, hemp (L)}: Can hold the weight of roughly three
  human-sized beings. 50' long. Comes with grapnel.
\item
  \textbf{Scrollcase (M)}: Each holds up to 10 spell levels in spell
  scrolls.
\item
  \textbf{Shovel (L)}: Excavating tool doubles as an improvised medium
  weapon.
\item
  \textbf{Sledgehammer (L)}: Excavating tool doubles as an improvised
  medium weapon.
\item
  \textbf{Torch (M)}: Burns for 1 hour, illuminating a 40' radius. Will
  remain lit if dropped to the ground or thrown. Can be used as a small
  weapon without dousing it; some targets take further damage due to the
  fire. Also useful to burn away rot grubs, green slime, and webs.
\item
  \textbf{Twine ball (100') (S)}: Can hold no more than 10 lbs of weight
  before snapping.
\item
  \textbf{Vial (S)}: Glass. Holds 4 oz of liquid.
\end{itemize}

\subsection{Armor}\label{armor}

\begin{quote}
{[}!NOTE{]} 100cp=10sp=1gp
\end{quote}

\begin{longtable}[]{@{}lll@{}}
\toprule\noalign{}
Armor & AC & Cost \\
\midrule\noalign{}
\endhead
\bottomrule\noalign{}
\endlastfoot
Shield (M) & +1 & 20sp \\
Leather or Furs (M) & 12 & 80sp \\
Ring (L) & 13 & 130sp \\
Scale/Lamellar (L) & 14 & 180sp \\
Chainmail (L) & 15 & 280sp \\
Splint (L*) & 16 & 580sp \\
Plate (L*) & 17 & 850sp \\
\end{longtable}

*Heavy armor (L* = 3 slot) requires minimum 9 Str to wear. A Shield also
provides a +2 save bonus vs non-gaseous breath weapons.

\subsection{Weapons}\label{weapons}

\subsubsection{Melee Weapons}\label{melee-weapons}

Weapons are divided into three basic damage categories: small, medium,
and large.

\begin{itemize}
\tightlist
\item
  \textbf{Small} weapons deal 1d4 damage, are one-handed, and include
  knives, daggers, and small handaxes. Can be dual-wielded for +1 to hit
  (doesn't grant additional attacks). Average cost: 12sp.
\item
  \textbf{Medium} weapons deal 1d6+1 damage, are one-handed, and include
  most swords, spears, maces, flails, and battle axes. Can be wielded
  two-handed, which adds +1 damage. Average cost: 20sp.
\item
  \textbf{Large} weapons deal 1d8+3 damage, are two-handed, and include
  polearms and pikes, as well as large swords, spears, axes, and maces.
  A minimum Str of 9 is required to wield large weapons. Average cost:
  75sp.
\end{itemize}

\emph{Note: You can get a weapon ``silvered'' for 3x its base cost.}

\subsubsection{Missile Weapons}\label{missile-weapons}

Missile weapons have a rate of fire of 1 shot per attack (except for
small thrown weapons with the marksman feat). Loading does not require
an action if ammunition is readily at hand (except for heavy crossbows,
which require an action to load). All missile weapons have three range
brackets (Short, Medium, Far). Attacks made against targets at a given
range apply the attack penalty listed.

\begin{longtable}[]{@{}
  >{\raggedright\arraybackslash}p{(\linewidth - 12\tabcolsep) * \real{0.4107}}
  >{\centering\arraybackslash}p{(\linewidth - 12\tabcolsep) * \real{0.2321}}
  >{\centering\arraybackslash}p{(\linewidth - 12\tabcolsep) * \real{0.0714}}
  >{\centering\arraybackslash}p{(\linewidth - 12\tabcolsep) * \real{0.0893}}
  >{\centering\arraybackslash}p{(\linewidth - 12\tabcolsep) * \real{0.0536}}
  >{\centering\arraybackslash}p{(\linewidth - 12\tabcolsep) * \real{0.0714}}
  >{\centering\arraybackslash}p{(\linewidth - 12\tabcolsep) * \real{0.0714}}@{}}
\toprule\noalign{}
\begin{minipage}[b]{\linewidth}\raggedright
Weapon
\end{minipage} & \begin{minipage}[b]{\linewidth}\centering
Size \& Damage
\end{minipage} & \begin{minipage}[b]{\linewidth}\centering
Cost
\end{minipage} & \begin{minipage}[b]{\linewidth}\centering
Hands
\end{minipage} & \begin{minipage}[b]{\linewidth}\centering
S:0
\end{minipage} & \begin{minipage}[b]{\linewidth}\centering
M:-4
\end{minipage} & \begin{minipage}[b]{\linewidth}\centering
F:-8
\end{minipage} \\
\midrule\noalign{}
\endhead
\bottomrule\noalign{}
\endlastfoot
Bow, short* & M & 40sp & 2 & 45' & 90' & 180' \\
Bow, long* (req 9 Str) & M & 60sp & 2 & 90' & 180' & 360' \\
Crossbow, light & M & 50sp & 2 & 45' & 90' & 180' \\
Crossbow, heavy** & L & 75sp & 2 & 90' & 180' & 360' \\
Holy Water & *** & 25sp & 1 & 10' & 20' & 30' \\
Javelin/Spear & M† & 16sp & 1 & 20' & 40' & 60' \\
Sling & S & 1sp & 1 & 45' & 90' & 180' \\
Small thrown weapon & S† & 12sp & 1 & 10' & 20' & 30' \\
\end{longtable}

*If fired indoors, halve all ranges.\\
**+2 attack bonus, reload requires action.\\
***See costed equipment list below for details.\\
†Plus user's Str modifier.

\subsection{Costed Items}\label{costed-items}

\begin{longtable}[]{@{}ll@{}}
\toprule\noalign{}
Item & Cost \\
\midrule\noalign{}
\endhead
\bottomrule\noalign{}
\endlastfoot
Acid, vial (S) & 50sp \\
Arrowhead, silver & 5sp \\
Book, leatherbound, 64pgs (M) & 30sp \\
Chain, 10' (L) & 45sp \\
Garlic, bulb & 1cp \\
Holy symbol, iron & 5sp \\
Holy symbol, silver & 50sp \\
Holy symbol, wooden & 1sp \\
Holy water vial (S) & 25sp \\
Mapping kit (M) & 10sp \\
Mirror, small, bronze (S) & 2sp \\
Rope, silk, 50' (M) & 50sp \\
Tent (L) & 20sp \\
Thieves' picks \& tools (M) & 12sp \\
Wolfsbane sprig (S) & 1sp \\
\end{longtable}

\textbf{Acid, vial}: Can splash contents on a target within 5' or hurl
it as a small thrown weapon; it shatters on impact. A hit deals 1d6+1
acid damage. A vial can also be used to open most mundane locks in 1
turn.

\textbf{Garlic}: Repels vampires.

\textbf{Holy symbols}: Wooden symbols incur -1 penalty to Turn Undead
hit dice roll; silver symbols receive +1 bonus.

\textbf{Holy water}: Can splash contents onto an undead creature within
5' or hurl it as a small thrown weapon; it shatters on impact. Causes
2d4 damage when thrown on most undead.

\textbf{Mapping kit}: A cased roll of parchment plus quills and vials of
ink, sufficient to map all but the largest areas.

\textbf{Rope, silk}: Lighter and stronger than hemp, can bear the weight
of five human-sized beings. Comes with grapnel.

\textbf{Tent}: Protects against adverse weather when adventuring in the
wilds. See \hyperref[sleeping-in-the-wilds]{sleeping in the wilds}.

\textbf{Wolfsbane}: If a lycanthrope is hit by wolfsbane, it must make a
Very Hard (17+) save or run away in fear. The sprig must be swung or
thrown as a weapon.

\subsection{Quick Packs}\label{quick-packs}

\begin{itemize}
\tightlist
\item
  \textbf{The Gygax} (10 slots, 25sp cost): Hempen rope (L), 10' pole
  (L), Iron spikes x2 (S), Hammer (S), Lantern (M), Oil flask x2 (M),
  Holy water vial (S, 25sp), Iron rations x2 (M).
\item
  \textbf{The Generalist} (6 slots, 10sp cost): Candle (S), Chalk (S),
  Hammer (S), Iron spike (S), Crowbar (M), Iron rations (M), Oil flask
  (M), Lantern (M), Mapping kit (M, 10sp).
\item
  \textbf{The Cautious} (6 slots): Candle (S), Chalk (S), Hammer (S),
  Iron spikes x4 (S), Wolfsbane (S), Caltrops (M), Iron rations (M), 10'
  pole (L).
\item
  \textbf{The Delver} (6 slots): Candle x3 (S), Hammer (S), Iron rations
  (M), Pickaxe (L) or Shovel (L) or Sledgehammer (L), Hempen rope (L).
\item
  \textbf{The Scholar} (6 slots, 10sp cost): Chalk (S), Vial x3 (S),
  Iron rations (M), Oil flask (M), Lantern (M), Mapping kit (M, 10sp),
  Scrollcase (M).
\item
  \textbf{The Torchbearer} (6 slots): Iron rations (M), Oil flask (M),
  Torches x4 (M).
\end{itemize}

\subsection{Animals \& Mounts}\label{animals-mounts}

\begin{longtable}[]{@{}
  >{\raggedright\arraybackslash}p{(\linewidth - 10\tabcolsep) * \real{0.2097}}
  >{\centering\arraybackslash}p{(\linewidth - 10\tabcolsep) * \real{0.0806}}
  >{\centering\arraybackslash}p{(\linewidth - 10\tabcolsep) * \real{0.0645}}
  >{\centering\arraybackslash}p{(\linewidth - 10\tabcolsep) * \real{0.1935}}
  >{\centering\arraybackslash}p{(\linewidth - 10\tabcolsep) * \real{0.2097}}
  >{\centering\arraybackslash}p{(\linewidth - 10\tabcolsep) * \real{0.2419}}@{}}
\toprule\noalign{}
\begin{minipage}[b]{\linewidth}\raggedright
Animal
\end{minipage} & \begin{minipage}[b]{\linewidth}\centering
Cost
\end{minipage} & \begin{minipage}[b]{\linewidth}\centering
Size
\end{minipage} & \begin{minipage}[b]{\linewidth}\centering
Combat Speed
\end{minipage} & \begin{minipage}[b]{\linewidth}\centering
Daily Hex Pts
\end{minipage} & \begin{minipage}[b]{\linewidth}\centering
Item Slot Limit
\end{minipage} \\
\midrule\noalign{}
\endhead
\bottomrule\noalign{}
\endlastfoot
Donkey/Pony & 70sp & L & 50 & 5 & 20 \\
Dog, hunting & 35sp & M & 50 & - & - \\
Dog, war & 65sp & M & 50 & - & - \\
Hawk & 40sp & S & 80 & - & - \\
Horse, draft & 150sp & L & 50 & 5 & 25 \\
Horse, riding & 100sp & L & 80 & 6 & 20 \\
Horse, war & 300sp & L & 65 & 5 & 25 \\
Mule & 90sp & L & 65 & 5 & 30 \\
Ox & 120sp & L & 50 & 4 & 30 \\
\end{longtable}

\begin{longtable}[]{@{}lc@{}}
\toprule\noalign{}
Item & Cost \\
\midrule\noalign{}
\endhead
\bottomrule\noalign{}
\endlastfoot
Barding & x4 \\
Cart & 50sp \\
**Feed, per day & 2sp \\
Stabling, per day & 1sp \\
Wagon & 1,200sp \\
\end{longtable}

\textbf{Barding}: Animal armor. The cost is equal to four times the cost
of a normal human-sized armor set of the same type and provides the same
AC benefit. Encumbers a mount just as armor does for characters.

\textbf{Cart}: Open, road-bound, two-wheeled vehicle. Pulled by 1-2
beasts of burden. Capacity: 2x of the animals drawing it.

\textbf{Encumbrance}: A rider counts against a mount's item slot limit
at a rate of 3 slots per the rider's size level, starting at Tiny (so 9
slots for a Medium rider), plus the rider's own carried slots, if any. A
mount can carry no more than two typical riders. An animal, cart, or
wagon 1 point over its item limit gains one encumbrance level. Every 3
slots past that adds another (or every 8 points past for a cart or
wagon).

\textbf{Wagon}: Open, four-wheeled, road-bound vehicle for heavy loads.
Pulled by 4-6 beasts of burden. Capacity: 2x of the animals drawing it.

\textbf{War}: An animal not trained for combat may panic in battle. If
it's wounded, its owner must make a Hard (14+) check to keep it from
fleeing or tossing its rider.

\begin{center}\rule{0.5\linewidth}{0.5pt}\end{center}

\section{Checks \& Saves}\label{checks-saves}

\textbf{Checks} are a single roll against a target number, used to
resolve situations with interesting stakes that would either be too
tedious or difficult to describe, or involve a strong element of chance.
Even if it comes down to a die roll, players will be rewarded for
thinking the action through, and sometimes penalized for not. For
example, if a player thinks to first apply some lamp oil to the chain
and gearing responsible for raising a stuck gate, this would reduce the
difficulty.

First, the referee decides how difficult a check is:

\begin{longtable}[]{@{}cc@{}}
\toprule\noalign{}
Difficulty Level & Result Needed \\
\midrule\noalign{}
\endhead
\bottomrule\noalign{}
\endlastfoot
Moderate & 8+ \\
Daunting & 11+ \\
Hard & 14+ \\
Very Hard & 17+ \\
Heroic & 20+ \\
\end{longtable}

The player rolls 1d20 (unless success or failure would not be obvious,
in which case the referee rolls instead, in secret). If an ability score
modifier is relevant, the check will note this (e.g.~``Hard (Con x2)''
means apply double the character's Con mod). If the modified total
equals or exceeds the result needed, the check succeeds. Rolling under
indicates failure. A natural 20 is not an automatic success for a check.

\textbf{Saving throws} or \textbf{saves} represent an attempt to resist
a notable threat, such as magic, poison, or disease. A creature can
always choose to fail a saving throw. A save is a check with any
relevant ability score listed as normal (e.g.~``Save: Hard (Con)''). The
default save is Hard (14+). The exception is saves for instant-death
effects (known as ``death saves''); their default save is Daunting
(11+). Unlike a regular check, a natural 20 always saves, and a natural
1 always fails. See also \hyperref[item-saving-throws]{item saving
throws}.

\textbf{Check Bonuses}: Characters gain +1 to regular checks at each
name level.

\textbf{Save Bonuses}: Characters gain a +1 save bonus for every 2 full
Hit Dice they have.

\begin{center}\rule{0.5\linewidth}{0.5pt}\end{center}

\section{Combat}\label{combat}

\subsection{Surprise}\label{surprise}

If a creature is not in combat, the opportunity exists to surprise it.
Some creatures can naturally reliably ambush prey. Others can do so via
careful planning or magical effects. Once combat has begun, true
surprise is impossible; at best, an attacker could achieve a rear
attack. There are two major ways for characters to surprise enemies:
laying an ambush, and moving silently to backstab a target.

\subsubsection{Ambushes}\label{ambushes}

No check is required to set an ambush; either the circumstances exist
for one to be set, or it's impossible (e.g., there's no available
cover). Instead, the targets make a check to see if they notice before
it's too late.

A typical ambush requires a Hard (14+) group Perception check to detect
if the ambushers are visually concealed (camouflage, heavy cover, or
invisibility). This rises to Very Hard (17+) if most noise is drowned
out as well (e.g., high winds or other strong ambient noise)--Heroic
(20+) if the ambushers are magically silenced. Some creatures have
innate surprise abilities, which override the values above. If the party
is outdoors and has one or more members with the Fieldcraft feat for the
terrain they're in, the difficulty of their group check to notice an
ambush is one level lower than normal.

If the detection roll fails, the group is surprised (see below). If the
roll succeeds, the group realizes they are about to be ambushed; what
this means depends on the circumstances, especially the terrain and
relative placement of the ambushers and targets.

\subsubsection{Backstab}\label{backstab}

A move silently check is needed to move behind a target and launch a
surprise attack. Failure means that the attempt has been noticed, and
the target can react as appropriate. Success means the target is
surprised (see below).

\subsubsection{Effects of Surprise}\label{effects-of-surprise}

Surprise lasts one round. In that round, those that are surprised cannot
move, act, or apply Dex-based AC modifiers, and attacks on them gain a
+4 bonus. Attacks against surprised targets from behind also ignore any
shield modifiers, and raise the chance for a critical hit. Killing
surprised foes might force a Morale check.

\subsection{Reactions}\label{reactions}

Some encounters are essentially pre-determined due to the nature of the
creatures or the encounter itself, and will end up in combat no matter
what (e.g., intelligent undead and sentinels like golems will almost
always attack).

However, when any creatures are encountered, if their nature or the
circumstances don't automatically dictate their behavior, the referee
\emph{always} rolls to see how they react to the players \emph{before}
actions are taken.

If the players do not immediately attack, the referee rolls on the
Reaction Table, applying -2 to the roll if the creatures encountered are
Evil, and rerolling Hostile results if the creatures are Good:

\begin{longtable}[]{@{}cc@{}}
\toprule\noalign{}
2d6 & Behavior \\
\midrule\noalign{}
\endhead
\bottomrule\noalign{}
\endlastfoot
2 & Hostile \\
3-6 & Unfriendly (unintelligent: Hostile) \\
7-8 & Neutral/uncertain \\
9-11 & Unthreatening \\
12 & Actively helpful (unintelligent: Unthreatening) \\
\end{longtable}

\subsection{Attacking}\label{attacking}

To attack, the attacker rolls 1d20 and adds their attack bonus and all
applicable modifiers. If the result equals or exceed's the target's
Armor Class (AC), the attack hits. The most common attack modifiers are:

\begin{itemize}
\tightlist
\item
  Attacker is striking from the rear: +2
\item
  Target is surprised: +4 (replaces above)
\item
  Attacker declared an offensive stance: +2
\item
  Attacker cannot see target: -4
\item
  Attacker is fatigued: -2 (light) or -4 (heavy)
\item
  Attacker is on a mount, target is upright: +2
\item
  Attacker is using improvised weapon: -2
\item
  Target is prone, attacker is in melee: +4
\end{itemize}

For PCs, a natural 20 is always a hit. The attack is also a critical
hit, unless the attacker could only hit by rolling a natural 20. Melee
and short-range missile attacks against sleeping, paralyzed, willing,
and similar targets always hit and deal maximum damage. Such targets
include PCs at 0hp.

\paragraph{Dealing Damage}\label{dealing-damage}

An attacker's Strength modifier is applied to melee and thrown weapon
attack damage. Modifiers cannot drop a successful attack below 1 point
of damage. Ar successful charge or set vs.~charge adds two weapon dice
to the attack. Attacks against helpless targets automatically deal
maximum damage.

An \textbf{unarmed strike} deals 1d2 damage (a critical hit does 4
damage), plus the attacker's Strength modifier. Weapon dice do not
apply.

If a PC rolls a natural 20 on an attack, a \textbf{critical hit} is
scored: the attack deals its maximum possible damage. However, you
cannot score a critical hit if you can only hit by rolling a 20. The
critical hit range is increased by 4 for all attacks from behind against
a surprised target.

\subsection{Combat Phases}\label{combat-phases}

Each combat round is 10 seconds and has seven steps, taken in the
following order:

\begin{itemize}
\tightlist
\item
  \textbf{Declarations}: All spellcasting and combat stances for the
  round must be declared \emph{before} anything else in the round
  happens.
\item
  \textbf{Missile}: Combatants with readied missile weapons can fire.
  Players who chose the Dash combat stance fire first, then all
  remaining missile attacks from all sides are resolved simultaneously.
\item
  \textbf{Initiative}: 1d12 is rolled by each side involved, with the
  side that rolled the highest winning (re-roll ties). Only determines
  movement order, not action order.
\item
  \textbf{Movement}: Everyone on the side that won initiative for the
  round can move, followed by any movement by the losing side.
\item
  \textbf{Melee}: Combatants that did not fire a missile weapon can
  attack with melee weapons (or take a miscellaneous action). Players
  who chose the Dash combat stance act first, and then all remaining
  actions from all sides are resolved simultaneously.
\item
  \textbf{Magic}: Spells are canceled if desired, and then all spells
  declared and not disrupted are cast.
\item
  \textbf{Morale}: The referee may need to make morale checks for
  opponents, retainers, and other NPCs.
\end{itemize}

\subsubsection{Declaration Phase}\label{declaration-phase}

Any character not surprised or casting a spell must pick a stance. A
character casting a spell must pick the exact spell at this time, and
cannot do anything else that round (i.e., no moves, attacks, actions,
etc.).

\textbf{Combat Stances}

\begin{itemize}
\tightlist
\item
  \textbf{Offensive}: +2 attack bonus
\item
  \textbf{Defensive}: +2 Armor Class
\item
  \textbf{Dash}: Your action occurs before those of the enemy
  (exceptions: charging a set opponent; attacking a pike/spear wall;
  grapple attempts)
\item
  \textbf{Guard}: Pick a 5' square. If you remain within 10' of it and
  can move freely, the first two enemies you are in the way of that try
  to move up to or make a ranged attack against anything in that square
  must engage you instead. Enemies already within 5' of the square or
  whom you are not in the way of are unaffected. Additionally, if an
  enemy can't be locked in melee, you cannot block its move. The effects
  of a stance last only for that round. Monsters and NPCs do not use
  stances.
\end{itemize}

\subsubsection{Missile Phase}\label{missile-phase}

Missile attacks can be made if you have a ranged weapon readied at the
start of the round, are not locked in melee, and the target is not
blocked (by one or more creatures along direct line of sight, unless the
target is one or more sizes larger than any blockers).

Missile attacks use all relevant modifiers above, except the bonus for
being on a mount or the target being prone. Missile-specific modifiers
include:

\begin{itemize}
\tightlist
\item
  Target is prone, attacker is at range: -2
\item
  Firing from a moving or unsteady position: -4
\item
  Low visibility (gloom, smoke, fog, etc.): -2
\item
  Target has cover: -2 (half cover) or -4 (heavy cover)
\item
  Target is at medium range: -4
\item
  Target is at long range: -8
\end{itemize}

A miss is generally presumed to hit no other target. Other
considerations:

\begin{itemize}
\tightlist
\item
  \textbf{Crossbows} can be fired while kneeling or prone.
\item
  \textbf{Firing into Melee}: To pick a target that is in melee, the
  target must be one or more sizes larger than all those with which it
  is in melee (unless the attacker is a warrior with the marksman feat).
  Otherwise, the attacker randomly rolls to see who in that melee
  (friend or foe) they roll their attack against. A target's melee
  opponents do not provide it cover.
\item
  \textbf{Holding Fire}: Instead of firing in the Missile Phase, an
  attacker can choose to wait until the end of any other Phase. This
  allows the use of thrown weapons ``on the run'' at the end of the
  Movement Phase at short range, just before being locked in melee.
\item
  \textbf{Magic Devices}: Ranged effects and spells from magic items are
  treated as missile attacks, except for line of sight.
\item
  \textbf{Poor Conditions}: If in very windy conditions, or if the
  target is above the attacker, long-range missile attacks cannot be
  made. If both, medium-range missile attacks also cannot be made.
\end{itemize}

\subsubsection{Movement Phase}\label{movement-phase}

In an encounter, a creature can move once per round up to its combat
speed (for unencumbered PCs, this is 40'). One can move through
creatures that allow it and aren't in a tight formation. Modifications
include:

\begin{itemize}
\tightlist
\item
  \textbf{Backwards Movement} applies a x2 movement cost penalty
  (e.g.~5' of clear terrain costs 10' of movement).
\item
  \textbf{Charging}: An attacker with a lance, pike, large spear, or
  similar large pole weapon and no more than lightly encumbered can
  charge. So too can creatures with large horns/tusks. A charge must be
  in a straight line from at least 30' away, in non-difficult terrain,
  and not uphill. The attacker's first strike on the round it makes the
  charge adds two weapon dice if it hits.
\item
  \textbf{Difficult Terrain} applies a x2 movement cost penalty (or x3
  if moving backward).
\item
  \textbf{Encumbrance}: Being encumbered reduces your combat speed.
\item
  \textbf{Holding Movement}: A combatant on the side that won initiative
  can choose to move after the enemy in the round.
\item
  \textbf{Jumping} ends your move that round.
\item
  \textbf{Prone}: To stand up, a prone creature must use all their
  movement for the round.
\item
  \textbf{Set vs.~Charge}: If armed with a pike, spiked polearm, or
  large spear, a combatant not prone or locked in melee can choose not
  to move that round and instead hold their weapon firm, braced against
  the ground. A set combatant attacks first against the first charge
  made against them (even if the attacker has the Dash stance), and adds
  two weapon dice if it hits.
\end{itemize}

\paragraph{Locked in Melee}\label{locked-in-melee}

A creature within 5 feet of one or more enemies (three or more if it has
the Whirlwind feat) is \textbf{locked} in melee with those enemies.
Locked combatants cannot use missile weapons or leave thier location
unless they make a fighting withdrawal or flee.\\
\textbf{Blind} creatures can only lock opponents in melee if surrounding
them.\\
\textbf{Flight and Teleportation} allow one to leave melee without
penalty, even if surrounded.\\
\textbf{Invisible} creatures are not locked unless their melee opponents
can see them or otherwise fully ignore invisibility, or are surrounding
them.\\
\textbf{Prone} creatures cannot lock opponents in melee.\\
\textbf{Size}: A creature more than two sizes large than any of their
melee opponents is not locked, even if surrounded.

\paragraph{Fighting withdrawal}\label{fighting-withdrawal}

A creature locked in melee can move out of that melee if they didn't
make a missile attack that round, but all their movement for the round
must be backwards (applying the standard x2 movement cost penalty). They
can take no actions that round.

\paragraph{Fleeing}\label{fleeing}

A fleeing creature can move as normal, but must leave the melee they're
in. However, any enemy with which they were in melee prior to fleeing
may first choose to take its attack(s) for the round against the fleeing
creature, instead of in the Melee Phase. Such attacks gain the +2 rear
attack bonus, while the fleeing creature loses all shield and stance AC
bonuses.

\paragraph{Running}\label{running}

If a combatant did not act in the Missile Phase and commits to taking no
actions this round, they may \textbf{run}. This raises their movement
this round by 50\% (rounded down to the nearest 5'), or by 100\% with
the Running skill.

\subsubsection{Melee Phase}\label{melee-phase}

In the Melee Phase, a combatant can take one action: a single attack
against an opponent with which they are in melee, or a single non-attack
action. All actions in this phase are simultaneous, not affecting any
other action that phase, unless specifically stated otherwise.
Considerations include:

\begin{itemize}
\tightlist
\item
  \textbf{Dash}: Choosing this combat stance allows a PC to act first
  that round, instead of simultaneously.
\item
  \textbf{Grappling:} The attackers (or attacker) make individual to-hit
  rolls. The grapple is resolved among those who hit: Each combatant
  rolls their total hit dice. If the sum of the attackers' dice is
  higher than the defender's total, the defender is grappled. If the
  sums are exactly equal, everyone is struggling, and none of them can
  attack with a weapon. If the defender wins, the attackers are beaten
  back and stunned for a number of rounds equal to the number of points
  by which the defender beat them. Breaking free of a grapple requires
  another HD contest; all currently grappling together are counted as
  automatically having hit.
\item
  \textbf{Polearms} (spears, pikes, etc.) in the second rank of a battle
  formation can attack by reaching through the first rank.
\item
  \textbf{Spacing \& the Second Rank:} Only daggers, shortswords,
  spears, and polearms can be used three-abreast in a 10' area. All
  other one-handed weapons require five feet of room (two-abreast in a
  10' area), and non-thrusting two-handed weapons require a full 10'
  space to wield.
\item
  \textbf{Two-weapon fighting}: A combatant can use two melee weapons at
  once, which grants a +1 attack bonus (but no additional attacks).
\end{itemize}

\subsubsection{Magic Phase}\label{magic-phase}

If a caster wants to cancel a spell they're casting that hasn't been
disrupted by having been attacked or jostled, they must do so before any
spells are resolved.

After spells are canceled, spells still being cast are revealed and cast
in the order of their \textbf{casting time}, which is equal to their
spell level, unless stated otherwise. Spells with the same casting time
are cast simultaneously, unless stated otherwise. For each instance of
the Quickcast feat that they have, a mage reduces their spells' casting
times by 1. Other feats raise a spell's level, raising its casting time
to match. Spells cast from scrolls add 2 to their casting time.

\subsubsection{Morale Phase}\label{morale-phase}

Monsters and NPCs (but not PCs) have a \textbf{Morale threshold} between
1 and 20, representing how likely they are to fight or flee; the lower
the threshold, the better.

Creatures with a Morale of 20 only fight if cornered and always flee if
able, while those with a Morale of 1 never retreat unwillingly, but can
be convinced to (if intelligent), and are still susceptible to magical
fear. A score of -- means the creature ignores the Morale rules
altogether, just like PCs.

The referee makes a Morale check--rolling 1d20, applying any situational
modifiers (light fatigue gives -2, heavy fatigue gives -4, etc.)--at the
end of a round for each of the following that occured that round:

\begin{itemize}
\tightlist
\item
  Half or more of the side present at the battle's start has been
  incapacitated or killed.
\item
  The last of the side (or its only member) was reduced to 1/4 or less
  of its full Hit Point total.
\item
  A group with a Morale of 12 or higher that has any of its members
  killed during the surprise round.
\item
  Any other unusually trying circumstance, at the referee's discretion.
  If the roll is lower than the Morale threshold, the check fails: the
  creature or side attempts to flee.
\end{itemize}

\subsection{Escaping an Encounter}\label{escaping-an-encounter}

If a side has none of its members locked in melee after the movement of
that side for the round has been resolved (whether or not the other side
has moved yet), that side can choose to try to escape the encounter.

If the enemy follows, \textbf{pursuit checks} are made. The referee
rolls 1d12 for each group of enemies with a different combat speed,
while the players roll 1d12 for their entire party. Apply any relevant
modifiers. If only some of the PCs have modifiers, only those PCs apply
them, giving them a result different from their base party result.

\begin{longtable}[]{@{}
  >{\centering\arraybackslash}p{(\linewidth - 2\tabcolsep) * \real{0.5362}}
  >{\centering\arraybackslash}p{(\linewidth - 2\tabcolsep) * \real{0.4638}}@{}}
\toprule\noalign{}
\begin{minipage}[b]{\linewidth}\centering
Fleeing Side
\end{minipage} & \begin{minipage}[b]{\linewidth}\centering
Modifier
\end{minipage} \\
\midrule\noalign{}
\endhead
\bottomrule\noalign{}
\endlastfoot
Is faster & +1 per 5' of combat speed faster \\
Is invisible & Auto escape \\
Drops caltrops or flaming oil & +4 \\
Drops desirable items (food/treasure) & +2 or +4 \\
Is lightly fatigued & -2 \\
\end{longtable}

\begin{longtable}[]{@{}cc@{}}
\toprule\noalign{}
Pursuing Side & Modifier \\
\midrule\noalign{}
\endhead
\bottomrule\noalign{}
\endlastfoot
Is faster & +1 per 5' of combat speed faster \\
Has senses hindered & -4 \\
Is lightly fatigued & -2 \\
\end{longtable}

\textbf{Faster Side} assumes that side can use their full speed, which
isn't a given. \textbf{Hindered Senses} usually involves difficulties
using one's primary tracking sense. Hunting dogs compensate for some of
these instances.

Each individual in a pursuit has one \textbf{pursuit action} each
pursuit round. However, neither mapping nor spellcasting can occur.

\begin{itemize}
\tightlist
\item
  \textbf{Dropping Items}: Any pursued as their action can drop items to
  apply a bonus to their side's pursuit check. Dropping desirable items
  like food or treasure can be effective, but this depends on how the
  pursuers view what is dropped, how much is dropped, and how many
  pursuers there are compared to what is dropped; the referee must
  arbitrate this. If enough food is dropped \& desired, iron rations
  (+2) are always less effective than fresh rations (+4).
\item
  \textbf{Missile Attacks}: After each pursuit check, if the pursuit is
  not over, then missile attacks can be mde by those who have not used
  their action. Assume a range of 20'.
\end{itemize}

\subsubsection{Ending Pursuit}\label{ending-pursuit}

A side automatically wins if they are either six points of modifiers or
more ahead of all their opponents, or score a higher result than the
other side twice in a row.

If the pursuers catch the fleeing side, a new combat round begins in the
Melee Phase, with the two sides locked in melee, positioning and so on
to be determined by the referee.

\subsection{Death}\label{death}

When a player character's hit points reach 0, the character is
unconscious and must make a Daunting (11+) Con saving throw. On a
failure, the character dies. On a success, the character will die in 1d4
rounds, rolled in secret by the referee, unless magically healed or
aided by another player character. Each PC may attempt to aid once,
requiring a successful Very Hard (17+) check.

Even after returning to 1 or more hp, the character will remain in a
coma for 1d6 turns and must rest for a minimum of one week before
resuming any sort of strenuous activity, mental or physical.

Characters who are slain may be raised from the dead if a Mage of
sufficient level is available to perform the casting. Each time a
character is brought back from the dead, their CON score is reduced by
one point.

\begin{center}\rule{0.5\linewidth}{0.5pt}\end{center}

\section{General Adventuring}\label{general-adventuring}

\subsection{Climbing}\label{climbing}

No roll is required for simple climbs, like a basic rope or tree. A more
difficult climb might require a Moderate (Dex) check. A climber must
make one check before every 40' section to be climbed or portion
thereof. A failed check indicates a fall from halfway up that section.

Base climbing speed is 5' per round. Add 5' if the climber has the
Climbing skill or is using rope (not cumulative).

The following each raise the difficulty by one level: high winds,
extreme cold, smooth surfaces, slippery surfaces, each encumbrance
level. The aid of appropriate equipment--a rope and grapnel, or pitons
and a hammer--will lower the difficulty.

\subsection{Doors}\label{doors}

Along with puddles, the bane of adventurers' existence.

\subsubsection{Listening at Doors}\label{listening-at-doors}

While exceptions can occur, it's assumed that player characters cannot
hear through the typical dungeon door. If the situation is such that
noise would carry through the door, the referee will notify players that
they hear something without the need of the players to ask.

\subsubsection{Spotting Secret Doors}\label{spotting-secret-doors}

Requires a Perception check, with the difficulty based on how
well-concealed the door is. During cautious exploration, this is a group
check made automatically by the referee in secret for each secret door
within 10'.

A character can also actively search a 10' square area for secret doors.
This takes one turn; more than one person cannot search the same area at
the same time.

Spotting the average secret door is Very Hard (17+) during cautious
exploration, and Daunting (11+) via an active search. Finding one does
not necessarily reveal how it opens. Describing narratively how one is
searching might bypass the Perception check if the searching technique
would logically reveal the door.

\subsubsection{Stuck Doors}\label{stuck-doors}

Upon discovering that a door is stuck, a character may attempt a
Strength check to pull the door open as swiftly and quietly as possible.
Only the strongest character in the party may attempt (representing the
party's best efforts). On a failure, the party may decide to work with
crowbars or other tools to wrench the door open anyway. This takes a
turn and will trigger a wandering monster check.

Locked doors need keys, thieves' tools and someone trained in
lockpicking, or a battering ram (or axes). Any failed attempt to open a
stuck or locked door will prevent surprise on any creature on the other
side of the door.

You should also know: Doors tend to close on their own. Iron spikes are
invaluable for keeping doors open or closed (it takes a round and some
noise to hammer a spike in).

\subsection{Dungeon Random Encounters}\label{dungeon-random-encounters}

Random encounters occur in almost all dungeon areas. These are typically
monsters inclined to hostility (you're in their home) and with no
treasure (it's back in their bedroom). To make a check, the referee
rolls 1d12 at the end of the appropriate turn: on a 2 or less, an
encounter occurs.

If an area doesn't have a preset encounter rate, how often the referee
checks to see if a random encounter occurs varies, based on the nature
of the area being explored:

\begin{itemize}
\tightlist
\item
  Organized defenders, alert sounded: every turn
\item
  Organized defenders, no alert sounded: every 2 turns
\item
  No organized defenders: every 3 turns
\end{itemize}

Even if players hide somewhere (e.g.~a disused storeroom), the encounter
roll is still made. Only if the hiding place is exceptional are
encounter checks avoided.

\textbf{Noisy}: If the party makes an unusual amount of noise while
exploring (yelling, spiking a door; combats do not count), their next
encounter check is made at -4, and the party cannot surprise creatures
so encountered except via unusual means.

\subsection{Equipment Wear \& Tear}\label{equipment-wear-tear}

Equipment can be worn down! If you are hit by an attack where the
attacker rolled a 20, your armor gains a \textbf{notch} of damage
(usually denoted by an x on your character sheet next to the weapon).
Each notch reduces your AC total by 1. Weapons gain a notch when you
roll a 1 on an attack with them. Each weapon notch reduces by one step
the damage die you roll with that weapon (i.e.~d8 \textgreater{} d6
\textgreater{} d4 \textgreater{} 1).

Items can be repaired by an appropriate craftsman, costing 10\% of the
item's original price per notch. Armor and weapons can take no more than
10 notches of damage before they're destroyed.

Magic weapons and armor take ½ notches instead of full notches.

\subsection{Experience \& Leveling}\label{experience-leveling}

You gain experience primarily through recovering treasure and then
\textbf{spending it}. 1sp spent = 1XP. Money spent on purchasing or
repairing equipment, spellcasting services, and on monthly upkeep costs
does not count toward XP. Money spent on acquiring retainers and
personally conducted magical research does count towards XP. In general,
you can flavor this spending as martial training, carousing,
philanthropy, greasing palms, investing in local faction relations, or
any number of other in-world expenditures, limited only by your
imagination.

XP from magical research is earned when the research is completed.
Failing at magical research does earn XP.

Each new dungeon room or hex \textbf{explored and mapped} grants 10 XP
per character. Each hex or room past the fifth in a given delve or
adventure begins to grant cumulative XP (i.e.~room \#6=20XP, room
\#7=30XP, etc.). The referee can optionally multiply this per-room
number by the dungeon's level or hex's distance from civilization to
account for the party's growth \& added difficulty. Once the party
returns to safety to heal and restock, the counter resets. A room or hex
can only ever be ``explored'' for XP in this way once.

The table below indicates what a character's XP total needs to reach to
advance to each level:

\begin{longtable}[]{@{}cccc@{}}
\toprule\noalign{}
Level & Total XP & Warrior Hit Dice & Mage Hit Dice \\
\midrule\noalign{}
\endhead
\bottomrule\noalign{}
\endlastfoot
1 & 0 & 1d8 & 1d6 \\
2 & 2,000 & 2d8 & 2d6 \\
3 & 4,000 & 3d8 & 3d6 \\
4 & 8,000 & 4d8 & 4d6 \\
5 & 16,000 & 5d8 & 5d6 \\
6 & 32,000 & 6d8 & 6d6 \\
7 & 64,000 & 7d8 & 7d6 \\
8 & 120,000 & 8d8 & 8d6 \\
9 & 240,000 & 9d8 & 9d6 \\
10 & 360,000 & 9d8+2 & 9d6+2 \\
\end{longtable}

Each level beyond 10 requires an additional 120,000XP and adds only a
flat 2HP per level, with no Con modifier being applied.

\subsection{Falling}\label{falling}

Damage from falling is determined as follows: Falls of less than 5 ft do
no damage in game terms, falls of up to 10 ft cause 1d6 damage, falls of
up to 20 ft cause 3d6 damage, falls of up to 30 ft cause 6d6, 40 ft is
10d6, 50 ft is 15d6, and falls of over 50 ft cause 20d6 points of
damage.

\subsection{Fatigue}\label{fatigue}

Fatigue represents a serious depletion of body, mind, or spirit.

\begin{longtable}[]{@{}cc@{}}
\toprule\noalign{}
Fatigue Level & Effect \\
\midrule\noalign{}
\endhead
\bottomrule\noalign{}
\endlastfoot
Light & -2 to attacks, checks, and Morale checks \\
Heavy & Raise above to -4, halve all movement \\
Exhaustion & Halve HP, halve movement again \\
\end{longtable}

Possession four or more levels of fatigue results in death. Fatigue
levels from different sources stack.

A character exposed to any of the causes of fatigue below gains one
fatigue level at the increments listed next to each cause (e.g.~every
full day without water applies one level).

\begin{longtable}[]{@{}cc@{}}
\toprule\noalign{}
Cause of Fatigue & Fatigue Level Increments \\
\midrule\noalign{}
\endhead
\bottomrule\noalign{}
\endlastfoot
Unable to breathe & 30 seconds (3 rounds) \\
Lack of water & Days 1/2/3/4 \\
Lack of sleep & Days 2/4/8/--* \\
Lack of food & Days 2/10/20/30 \\
\end{longtable}

*A lack of sleep can only apply up to three fatigue levels, i.e.~it
cannot kill you alone.

Per positive Con modifier point, a character can ignore 30 seconds
without air, or one day of a lack of water, sleep, and/or food.

If total exposure to a cause of fatigue is avoided (e.g.~a bit of water
drank, a fitful nap here or there), the current fatigue level increment
is doubled.

\subsection{Healing}\label{healing}

Natural healing rates are dependent on the
\hyperref[upkeep-lifestyle]{lifestyle} you lead. A character leading a
Comfortable lifestyle will recover 1 HP by resting overnight in a safe
and comfortable location, and HP equal to their level per day of
uninterrupted rest. A Poor lifestyle reduces your healing rate from
daily to every two days, and a Wealthy lifestyle doubles your healing
rate. 30 days of rest will return any character to full HP.

\subsection{Item Saving Throws}\label{item-saving-throws}

Generally, only creatures make saving throws. However, some items are
especially fragile, and some rare effects (like area-effect attacks or
puddings/oozes) specifically target objects. A failed item save results
in the item's destruction, while a successful save results in no damage
or effect.

If a creature must make a saving throw and it passes, no item carried by
that creature needs to make a save unless the effect specifies
otherwise.

\textbf{Fragile items} can be almost anything (not including weapons,
armor, or typical adventuring gear), but most notably include potions,
scrolls, and wands that have been \textbf{readied} (kept close at hand,
so that they're usable in combat without requiring an action to draw).
The price of having such useful items at hand for immediate use is that
they're vulnerable to destruction. For example, a potion in a backpack
is not available for drinking in combat, but at the same time, if its
owner is hit by a fireball, the potion is safe. Fragile items are the
exception rather than the rule, and tend to be consumable, creating a
risk to their use.

\begin{itemize}
\tightlist
\item
  \textbf{Area-effect Attacks}: Affect any item that's both fragile and
  readied. Daunting (11+) save.
\item
  \textbf{Bashing Containers}: Potions in a container require a Hard
  (14+) save, while scrolls, wands, and gemstones require a Moderate
  (8+) save.
\item
  \textbf{Disintegration}: Affects all readied items, fragile or not.
  Very Hard (17+) save, Hard (14+) if the item is magical.
\item
  \textbf{Falls}: Affects carried potions (readied or not), requires a
  fall of at least 20' onto a hard surface. Hard (14+) save, -1 penalty
  for each additional full 10' fallen over 20'.
\item
  \textbf{Magic Items}: Any that provide magic attack, save, or AC
  bonuses apply this modifier to any save such items are forced to make.
\item
  \textbf{Shields}: If used to provide a save bonus against a breath
  weapon attack, and the shield's wielder fails their save, the shield
  must save as well. The save is Moderate (8+).
\item
  \textbf{Water}: Affects paper and papyrus, Heroic (20+) save. Standard
  spellbooks (with vellum pages and magical inks) and scrolls in scroll
  cases always pass their save.
\item
  \textbf{Other Cases}: Cases not covered above are a Hard (14+) save.
\end{itemize}

\subsection{Jumping}\label{jumping}

A long jump lets you safely clear 10 feet (20, with the Jumping skill),
if you run at least 10 feet first (if not, halve the result). With a
10-foot run, a vertical high jump gives you a reach with your hands of
about 5 feet (a standing high jump gives you a reach of about 3 feet).
If encumbered, apply the combat speed penalty to your jump distance. In
combat, a jump ends your move.

\subsection{Lifting}\label{lifting}

The encumbrance rules are the standard way to measure how much a PC can
carry around (see p.~30). However, to quickly calculate the maximum a
character can lift or push, use their Strength × 15 in pounds.

\subsection{Light \& Vision}\label{light-vision}

A party needs one light source for approximately every three members of
the party. For every missing light source, PCs incur a -1 penalty to
attack, up to -4. Note that light sources can be seen from much further
away than the illumination they shed for those holding them. Approaching
light will warn intelligent creatures of the approach of
surface-dwellers, perhaps giving them a chance to prepare; creatures
around a corner can see a light source whose radius projects around that
corner, while two corners between prevent its detection.

\subsection{Lockpicking}\label{lockpicking}

Characters with the proper tools and either the Lockpicking feat or
skill can attempt to pick locks. This is a Dex check, the difficulty
scaling with the locks' complexity. If a character's lockpicking ability
is from a feat rather than a skill, the difficulty should be a level
lower.

A lockpicking attempt requires 1 turn. If a character fails, they can
try again, but after two failed attempts, the character cannot try that
lock again until they gain a level.

\subsubsection{Trapped Locks}\label{trapped-locks}

Noticing a trap on a lock requires a successful Perception check (having
the Lockpicking feat reduces the difficulty to notice such traps by one
level). Only an active search for traps can reveal such traps without
triggering them. In the case of multiple traps on a single lock, the
referee will roll a separate Per check for each trap.

A successful lockpicking attempt disarms all detected traps in addition
to opening the lock, but undetected traps will be set off automatically
before any lockpicking roll is made, unless specified otherwise.

\subsection{Magic \& Spells}\label{magic-spells}

\subsubsection{Creating Potions}\label{creating-potions}

Potions can be crafted by any mage of 5th level or above, with the aid
of an alchemist (or with the Skill), and only one potion may be made at
any one time.

Potion brewing requires a stocked laboratory of at least 1,000sp in
value. A mage can only brew potions they have drank or own the recipe
for, and costs silver per dose to brew; the referee will have these
costs.

\subsubsection{Gaining New Spells}\label{gaining-new-spells}

Casters learn one random spell each time they gain a level from a random
school they can access (plus another spell if a specialist). If the
caster has just gained access to a new spell level, the spell(s) are
from that level. They can also gain access to new spells via the
following methods:

\begin{itemize}
\tightlist
\item
  \textbf{Binding}: Anyone that can read Mithric can read a spellbook or
  scroll to see what it is, but you cannot prepare and thus cast a spell
  until you have bound it to you. Only spells from schools the caster
  can access can be bound. Binding a spell takes 1 week + 100sp per
  level of the spell, and a successful Moderate (Arc x2) check, rolled
  at the end of the binding period; apply +2 if a specialist is binding
  a spell from their specialist school. Failure means the spell is
  permanently erased from the source scroll or spellbook. Success means
  that the spell has been copied to your own spellbook, and that a
  permanent bond between the caster and spell has been created: you can
  always write the spell into a spellbook or scroll, even if you don't
  have another written copy on hand to reference or have the spell
  prepared for casting.
\item
  \textbf{Research}: Casters can research new spells. These can be
  variants of spells the caster already knows--a Hard (Arc)
  check--spells the caster has witnessed being cast--a Very Hard (Arc)
  check--or spells the caster has only heard of--a Heroic (Arc) check.
  Only spells from schools the caster can access can be researched, and
  any given spell can only be attempted once per level. Researching a
  new spell requires, on average and per level of the spell, 1d4+1 x
  50sp in materials and 1 week.
\end{itemize}

\subsubsection{Mage Spell Progression
Table}\label{mage-spell-progression-table}

This shows the number of spells of a given spell level that a mage can
prepare per day (also known as their spell slots). If the mage is a
specialist, they add one to each level's spell slot total, but the extra
spell must be from the school specialized in. If a spell is
level-adjusted through feats, that spell is treated in all ways as a
spell of the level to which it has been modified.

\begin{longtable}[]{@{}ccccccc@{}}
\toprule\noalign{}
Mage Level & 1 & 2 & 3 & 4 & 5 & 6 \\
\midrule\noalign{}
\endhead
\bottomrule\noalign{}
\endlastfoot
1 & 2 & & & & & \\
2 & 3 & & & & & \\
3 & 3 & 1 & & & & \\
4 & 3 & 2 & & & & \\
5 & 4 & 2 & & & & \\
6 & 4 & 2 & 1 & & & \\
7 & 4 & 3 & 2 & & & \\
8 & 5 & 3 & 2 & & & \\
9 & 5 & 3 & 2 & 1 & & \\
10 & 5 & 3 & 2 & 2 & & \\
11 & 5 & 4 & 3 & 2 & & \\
12 & 5 & 4 & 3 & 2 & 1 & \\
13 & 6 & 4 & 3 & 2 & 1 & \\
14 & 6 & 4 & 3 & 3 & 2 & \\
15 & 6 & 5 & 4 & 3 & 2 & 1 \\
16 & 6 & 5 & 4 & 3 & 2 & 1 \\
17 & 6 & 5 & 4 & 3 & 3 & 1 \\
18 & 6 & 5 & 4 & 4 & 3 & 2 \\
19 & 6 & 6 & 5 & 4 & 3 & 2 \\
20 & 6 & 6 & 5 & 4 & 3 & 2 \\
\end{longtable}

\subsubsection{Preparing \& Casting}\label{preparing-casting}

\textbf{Preparing Spells:} Spell preparation can occur no more than once
every 24 hours, requires at least 4 hours of uninterrupted rest
immediately prior, and no fatigue levels from exhaustion or lack of
sleep. One interrupted hour of study is then needed to prepare all
spells chosen, no matter how many, but can prepare no more than two uses
of the same spell at any one time (altering a spell with a feat does not
make it a ``different'' spell). Spells, once cast, are lost from the
casting character's memory and cannot be reused until the caster
prepares them again.

\textbf{Spell Disruption}: If, between declaring a spell and casting it,
the caster is hit by an attack or the like (even if no damage is dealt),
or fails a saving throw, the spell is disrupted: it fails and its spell
slot is emptied with no other effect. Note that only spells being cast
can be disrupted. Spell effects from a rod, staff, wand, etc. and
spell-like innate creature abilities are immune to disruption.

\subsubsection{The Schools of Magic}\label{the-schools-of-magic}

All spells belong to one of eight schools of magic. A caster can only
learn spells from schools to which they have access. A caster gains
access to a new school of their choice at every name level, learning
random spells belonging to that school, one at each level they can cast.
The schools are:

\begin{itemize}
\tightlist
\item
  \textbf{Abjuration}: Spells protective in nature. These frequently
  ward against damage or hostile effects, like gas, poison, possession,
  etc.
\item
  \textbf{Conjuration}: Spells that summon creatures or objects, from a
  simple fog to extraplanar entities that serve your every whim.
\item
  \textbf{Divination}: Spells that reveal information, from hidden traps
  and chambers to items and fell secrets.
\item
  \textbf{Enchantment}: Spells that affect the minds of others and bend
  life to your will--men, monsters, and even plants--making them angry
  or ambivalent, docile or dependent.
\item
  \textbf{Evocation}: Spells that shape raw magic itself. Most purely
  offensive spells (like \emph{Magic Missile}, \emph{Fireball}, etc.)
  belong to this school, making it nearly mandatory for any aspiring
  battle wizard.
\item
  \textbf{Illusion}: Spells that deceive; imaginary sounds, smells,
  objects, creatures, and even entire environments are possible.
\item
  \textbf{Necromancy}: Spells manipulating the energies of life and
  death, covering both healing the living and interactions with the dead
  (and undead).
\item
  \textbf{Transmutation}: Spells that alter the properties of a
  creature, object, or environment. One can fly, breathe underwater, or
  gain great strength, alter the size of a creature or object, or
  transmute one type of material into another. Sometimes called
  ``Alteration.''
\end{itemize}

\subsubsection{Scrolls}\label{scrolls}

Spells can be bound to scrolls; each holds one spell. Holding it with
both hands and reading from it aloud casts its spell, disintegrating the
scroll.

A spell on a scroll is not a prepared spell and can be cast even if the
caster does not own the spell or isn't able to cast spells at that
spell's level. The caster must still have access to the school to which
the spell belongs, however. Spells on scrolls can't be modified through
feats, though a caster can scribe a modified spell, and spells can't be
prepared from scrolls.

For the purposes of range and so on, a scroll spell is treated as if the
reader is casting it normally or is the minimum level required to cast
it normally, whichever is higher.

Scrolls may be scribed by anyone who has the ability to both read
scrolls and to cast the spell being scribed. It costs 500sp and one week
per level of the spell, which can be broken up into multiple sessions.
This is a Moderate (Arc x2) check, rolled in secret by the referee at
the end of the week. On a roll of a 1, the referee rerolls: a result of
1-10 means that a cursed scroll has been created.

\textbf{Creating Magic Items:} Potions are created by alchemists, with
the more powerful potions usually requiring the help of a Magic-User. A
Magic-User seeking to create potions must employ an alchemist. A
Magic-User must be 9th level to create potions on their own, and 11th
level to create other magic items.

\subsubsection{Spell List}\label{spell-list}

Spells are as in
\href{https://osrsimulacrum.blogspot.com/2021/06/simulacrum-beta-release.html}{\emph{Simulacrum}}.

\subsection{Mapping}\label{mapping}

Maps are usually best made simply: boxes and lines are sufficient to
keep you from getting lost. The players' map represents an actual
in-game object. If the players at the table are making a map, then a
character must also be making one and have the tools to do so. This has
several corollaries: the party must have light (they can only map what
they see) and mapping supplies (something to write with and something to
write on), and they must be moving at a cautious exploration rate.
Perhaps most importantly, a map being actively made is a fragile item
(see \hyperref[item-saving-throws]{item saving throws})--if something
happens to the map in-game, it happens to the players' map as well! If
the party wants backup copies, the players must actually draw them. If
the entire party dies in the dungeon, the only way their maps will
survive is if copies were left on the surface.

\subsection{Moving Silently}\label{moving-silently}

Anyone can attempt to sneak. A check is called for when attempting
movement that normally attracts attention, such as slipping past a
guard, or maneuvering behind a target for a surprise attack.

Moving silently requires a successful Dex check, the difficulty of which
is set by the referee, and which is rolled in secret by the referee.
While a failed move silently check will alert the enemy in some fashion,
it does not necessarily give away your position, let alone imply a
bumbling, noisy disaster.

A move silently attempt reduces your movement to 1/4 (normally, to 10'
per round). Every additional 10' per round added to that movement rate
raises the difficulty by one level.

\begin{itemize}
\tightlist
\item
  \textbf{Alert Enemies}: The attention of most intelligent undead
  (e.g., skeletons, zombies) and constructs (e.g.~golems) never wavers.
  As such, the difficulty of move silently checks against them is one
  difficulty higher.
\item
  \textbf{Armor}: If wearing non-magical medium or heavy armor, increase
  the difficulty by one level.
\item
  \textbf{Group Checks}: If a group attempts to sneak together, this is
  a group check, with the difficulty based on the group's least stealthy
  member.
\item
  \textbf{Surprise}: Moving silently can be used to set up surprise
  attacks.
\end{itemize}

\subsection{NPC Spellcasting}\label{npc-spellcasting}

Non-player characters may be hired to cast spells or perform other
services. As a general guideline, spells cost \emph{roughly} the
following, and will be subject to a host of in-the-fiction
considerations:

\begin{longtable}[]{@{}cc@{}}
\toprule\noalign{}
Spell Level & Cost per Casting \\
\midrule\noalign{}
\endhead
\bottomrule\noalign{}
\endlastfoot
1st & 50sp \\
2nd & 100sp \\
3rd & 250sp \\
4th & 500sp \\
5th & 1000sp \\
6th & 2500sp+ \\
\end{longtable}

Paying for spellcasting services does not count towards XP.

Sages or alchemists not employed by a PC will often charge around 100sp
to identify a potion (which takes an hour or so), or 200+sp to identify
a magical item (which can take upwards of a week).

\subsection{Perception Checks}\label{perception-checks}

Perception checks are always made in secret by the referee. There's no
requirement for players to constantly state that they're doing basic
investigative tasks that are repetitive and/or obvious, like ``I'm
looking at the floor'' or ``I'm listening for noises.'' The slow pace of
cautious exploration accounts for these.

However, sometimes a scenario involves something unusually subtle or a
creature has the abilities or has taken the effort to evade typical
scrutiny, and calling out a specific action might be required.

Overall, checks aren't made just to use one's eyes or ears or otherwise
notice the obvious. A Perception check only occurs when the rules call
for one (e.g.~looking for traps or secret doors), or if the referee
decides one is needed.

For a group Perception check, the referee applies the group's average
Per modifier, if any, rounding normally. For checks by individuals with
a positive Per modifier, \emph{twice} the searcher's bonus is applied.

\subsection{Readied Items}\label{readied-items}

Small, easily accessible items (e.g., sheathed daggers, wands, or
potions) on your character can be declared as \textbf{readied}. A scroll
in a scroll case, or something buried in your backpack, could not be
readied. A readied item takes no action to draw. For example, you could
take out a readied wand and fire it in the same Missile Phase.

The downside is that readied items are vulnerable to destruction. If a
creature fails certain types of saving throws (most notably, against an
area-effect attack), most readied items then have to make saves of their
own to avoid destruction. See \hyperref[item-saving-throws]{item saving
throws} for more details.

\subsection{Retainers}\label{retainers}

Retainers cost 50sp on initial hire (+50sp to request a class), plus ½
share of treasure. They will be level 1d3, but cannot be higher than the
hiring PC's level. Retainers' starting Morale typically ranges from 18
(the craven) to 6 (the elite), and can be randomly determined at hiring
with a 2d6+6 roll.

Loyalty is checked with a Morale roll after each adventure, if the
retainer is reduced to 1/4 or less of its full HP total, or if their
loyalty is severely tested. If a success is rolled, the retainer's
Morale threshold decreases by one to a minimum of 6. On a failure, the
retainer departs.

Porters/torchbearers come with no equipment. Equipment purchased by the
PC is kept by the PC. Retainers come with the same basic starting
equipment as PCs, but no starting silver. Stats, specializations \&
feats are rolled randomly, and no skills are included (but can be chosen
if a player ends up playing the retainer as a replacement for a lost PC,
for example).

Starting armor is based on class. Mages roll 1d6-1; Warriors roll 1d6+1:

\begin{itemize}
\tightlist
\item
  0-1=No armor (AC 9)
\item
  2-3=Leather (AC 12)
\item
  4-5=Ring (AC 13)
\item
  6=Scale (AC 14)
\item
  7=Chainmail (AC 15)
\end{itemize}

Choose a weapon option:

\begin{itemize}
\tightlist
\item
  Option A: One medium melee weapon \& a shield
\item
  Option B: One large weapon
\item
  Option C: Two medium one-handed weapons (+1 to hit)
\item
  Option D: One medium melee weapon \& one ranged weapon (plus ammo)
\end{itemize}

If appropriate, the referee can choose to roll on the Offer Reaction
Table below--applying modifiers based on party reputation, the mission
description, or lavish or miserly rates of pay if desired--to decide the
potential retainer's reaction:

\begin{longtable}[]{@{}cc@{}}
\toprule\noalign{}
2d6 & Reaction to Offer \\
\midrule\noalign{}
\endhead
\bottomrule\noalign{}
\endlastfoot
2 & Hostilely declines* \\
3-5 & Declines \\
6-11 & Accepts \\
12 & Eagerly accepts** \\
\end{longtable}

* The reaction is so bad that the NPC spreads negative rumors about the
PC and/or party, resulting in a -2 on further hiring rolls on this table
if the PC and/or party attempt further recruitment in this area.\\
** Permanent -1 bonus to the retainer's Morale threshold.

\subsection{Strongholds \& Domains}\label{strongholds-domains}

Strongholds and domains are as in OSE Advanced, with clarifications and
detailed procedures as in
\href{https://alexmooney.github.io/ACKS_SRD/Chapter07.html\#strongholds-and-domains}{ACKS}.

\subsection{Swimming}\label{swimming}

All characters can swim, barring an unusual background. Assuming no
current, a land-dweller swims at half their combat speed, and can do so
for hours equal to 1/4 their Con score (3/4 speed and 1/2 Con with the
Swimming skill); round down. If lightly encumbered, halve the amount of
time a character can swim. Higher encumbrance (or wearing medium or
heavy armour) causes one to sink and begin to drown if in deep enough
water (see \hyperref[fatigue]{Fatigue}).

\subsection{Trap Detection}\label{trap-detection}

Spotting a trap is a Perception check. A character can actively search a
10' square for traps. This takes 1 turn; only one person can search a
given square at any one time. Subtle but ultimately visible traps and
triggers (such as tripwires) should be automatically detected by this; a
failed search usually does not trigger traps.

How well a trap is concealed determines the difficulty. The average
concealed trap is Very Hard (17+) to spot during cautious exploration,
or Daunting (11+) if actively being searched for and not visible.

\textbf{10-foot Poles}: Prodding ahead with one or more of these and
using cautious exploration gives a 2-in-6 chance of triggering area
traps (safely, unless the trap specifies otherwise).

\textbf{Area Traps}: Traps that are triggered by moving into a map
square (e.g.~pit traps, but not trapped doors) are area traps. During
cautious exploration, a separate group check is automatically made for
each area trap trigger within 10 feet.

\subsection{Upkeep \& Lifestyle}\label{upkeep-lifestyle}

At the end of every in-game month, player characters must pay upkeep
equivalent to a percentage of their current XP total in sp, which does
not count towards XP (for example, a 1st-level character with 1,500 XP
would pay 15sp at the end of a month, which would set their lifestyle to
``Comfortable'' for the following month). This is an abstraction of
buying food, shelter, entertainment, and generally living the lifestyle
of an adventurer.

There are three lifestyles: Poor, Comfortable, and Wealthy. A Poor
lifestyle gives you threadbare clothing, simple food and lodgings, and a
livable though uncomfortable experience. Comfortable gives you good
food, well-kept lodging, respectable conditions and your basic needs
readily met. A Wealthy lifestyle is the life of luxury: a spacious home,
sumptuous table fare, skilled and fashionable tailors, servants, and
other attendants to meet your every need.

Your lifestyle choice can have in-game consequences. Living rough might
help you avoid notice, but you're unlikely to make important connections
doing so. Maintaining a luxurious lifestyle will let you access and
befriend the rich and powerful, but you run the risk of attracting
thieves, deceit, intrigue, and treachery.

Also, your healing rates during downtime are affected by your lifestyle
(see \hyperref[healing]{healing}).

\begin{longtable}[]{@{}cc@{}}
\toprule\noalign{}
Lifestyle & \% of XP in sp \\
\midrule\noalign{}
\endhead
\bottomrule\noalign{}
\endlastfoot
Poor & 0.5\% \\
Comfortable & 1\% \\
Wealthy & 2\% \\
\end{longtable}

\begin{center}\rule{0.5\linewidth}{0.5pt}\end{center}

\section{Wilderness Exploration}\label{wilderness-exploration}

In a campaign where the journey is as important as the destination (or
where there isn't even necessarily a main destination \emph{per se}),
overland movement comes into play.

Travel overland is measured in 6-mile hexes. \textbf{Hex points} are
spent to travel through hexes, and all player characters have a base hex
point allowance of \textbf{four per day}, which can be modified by
mounts, encumbrance, and fatigue (applied before encumbrance). A party
travels at the rate of its slowest member.

The base cost in hex points to enter a hex depends on its terrain type.
Unlisted terrain features like great rivers, canyons, etc. can further
raise the cost.

\begin{longtable}[]{@{}cc@{}}
\toprule\noalign{}
Terrain Type & Entry Cost \\
\midrule\noalign{}
\endhead
\bottomrule\noalign{}
\endlastfoot
Plains, steppe, farmland & 1 \\
Hills, woods, desert, rough & 2 \\
Mountains, jungle, swamps & 4 \\
\end{longtable}

Terrain factors can affect the base entry cost of a hex:

\begin{longtable}[]{@{}cc@{}}
\toprule\noalign{}
Terrain Modifiers & Cost Modifier \\
\midrule\noalign{}
\endhead
\bottomrule\noalign{}
\endlastfoot
Heavy rain/deep snow/thick fog & +1 \\
Temperature extremes & +1 \\
Good roads or excellent trails & -1 \\
\end{longtable}

The good roads bonus can apply once per day; at least two hexes of good
roads must be covered that day.

A party may choose one of two optional march types each day, reflecting
its priorities: either caution or speed.

\begin{itemize}
\tightlist
\item
  \textbf{Cautious March}: Reduce the party's hex allowance by one, and
  apply a +1 bonus to all rolls for overland random encounters, and the
  party has a better chance of tactically favorable encounters.
\item
  \textbf{Forced March}: Add one to the party's hex allowance, but gain
  one fatigue level at day's end. Can be kept up until a marcher is
  heavily fatigued. A day of full rest is required to remove one level
  of forced march fatigue from most creatures.
\end{itemize}

\subsection{Entering a Hex}\label{entering-a-hex}

If a party wants to enter a hex, but lacks some of the points needed,
they points they do have are spent towards entering it, but the party
ends the day in their current hex. The party only enters the new hex
once its full hex point cost is paid.

\subsubsection{Navigation and Getting
Lost}\label{navigation-and-getting-lost}

If the party has never been to a hex they've just entered, the party
makes a \textbf{navigation check} by rolling 1d12. Subtract the hex's
base Entry Cost, and another -6 if there's thick fog, a blizzard, a
sandstorm, or the like in the hex. Remove up to 2 points of these
penalties if the party has any members with the Fieldcraft feat for that
hex's terrain type. This check is skipped if the party has a respectable
map or knowledgeable guide, or if there is a road, trail, coastline, or
river in the current hex they can logically follow to their next hex.
The party is assumed to map any hex passed through.

On a result of 1 or less, the check fails \& the party is lost. To leave
the hex, the party must spend hex points equal to the amount they spent
to enter it originally, and then make a new navigation check. Failure
means that the party is still lost.

\subsubsection{Searching}\label{searching}

Entering a hex allows a party to determine its features. There are two
types: \textbf{overt} and \textbf{hidden}.

Overt features require no special effort to find. Hidden features may or
may not exist, but can only be found if the party searches the hex,
moving off the beaten path to seek out points of interest there. A hex
may have both types.

Searching a hex requires spending the same number of hex points that it
cost to enter the hex, and, like entering a hex, is only complete when
the full hex point cost is paid. The party then makes another navigation
check; failure means the party is lost, which is resolved normally.
Success reveals one hidden feature in the hex, if any.

\subsection{Camping in the Wilds}\label{camping-in-the-wilds}

\subsubsection{Building a Fire}\label{building-a-fire}

Given a means of producing flame (e.g.~a tinderbox, magic) and a stash
of wood (either gathered from the forest or carried in packs), a
character may attempt to build a fire.

\textbf{Good conditions:} In favourable conditions, with decent wood and
a relatively dry campsite, fire-building automatically succeeds.

\textbf{Bad conditions:} In more troublesome circumstances, getting a
fire going requires a Moderate (8+) check, though ``Dwarves can make a
fire almost anywhere out of almost anything, wind or no wind'' and will
automatically succeed. However, the referee may reduce the chance of
success to account for extreme cold or damp.

\subsubsection{Fetching Water}\label{fetching-water}

Finding water to drink is assumed to have happened naturally while
traveling, except in an exceptionally dry environment, when it is only
found on a 2-in-6 chance per hex.

\subsubsection{Cooking}\label{cooking}

Given a fire, cooking pots, and ingredients (e.g.~foraged food, standard
rations, hunted game), someone may cook a meal, a Moderate (8+) task.
Someone particularly skilled in Cooking succeeds automatically.

\textbf{If the check succeeds:} An especially tasty dish is produced.
Those who eat the meal lower the difficulty of the Con check required to
rest (see below), due to their hearty supper.

\textbf{If the check fails:} A palatable but not exemplary dish is
produced. A natural 20 denotes a ruined meal (burned, spilled, etc.)
that is utterly inedible. No effect on the Con check.

\subsubsection{Rest Checks}\label{rest-checks}

When camping in the wild, characters' ability to get a good night's rest
is determined by their equipment (whether they have a bedroll and/or
tent), their warmth (whether they have a fire burning), and the season.
Non-ideal circumstances require PCs to make a Con check, with the
difficulty listed below.

\begin{longtable}[]{@{}
  >{\raggedright\arraybackslash}p{(\linewidth - 10\tabcolsep) * \real{0.0879}}
  >{\raggedright\arraybackslash}p{(\linewidth - 10\tabcolsep) * \real{0.1648}}
  >{\raggedright\arraybackslash}p{(\linewidth - 10\tabcolsep) * \real{0.1868}}
  >{\raggedright\arraybackslash}p{(\linewidth - 10\tabcolsep) * \real{0.1868}}
  >{\raggedright\arraybackslash}p{(\linewidth - 10\tabcolsep) * \real{0.1868}}
  >{\raggedright\arraybackslash}p{(\linewidth - 10\tabcolsep) * \real{0.1868}}@{}}
\toprule\noalign{}
\begin{minipage}[b]{\linewidth}\raggedright
Fire
\end{minipage} & \begin{minipage}[b]{\linewidth}\raggedright
Bed
\end{minipage} & \begin{minipage}[b]{\linewidth}\raggedright
Winter
\end{minipage} & \begin{minipage}[b]{\linewidth}\raggedright
Spring
\end{minipage} & \begin{minipage}[b]{\linewidth}\raggedright
Summer
\end{minipage} & \begin{minipage}[b]{\linewidth}\raggedright
Autumn
\end{minipage} \\
\midrule\noalign{}
\endhead
\bottomrule\noalign{}
\endlastfoot
No fire & No bedding & Automatic failure & Hard (14+) & Daunting (11+) &
Hard (14+) \\
No fire & Bedroll or tent & Automatic failure & Daunting (11+) & Good
night's rest & Daunting (11+) \\
No fire & Bedroll \& tent & Hard (14+) & Daunting (11+) & Good night's
rest & Daunting (11+) \\
Campfire & No bedding & Automatic failure & Hard (14+) & Daunting (11+)
& Hard (14+) \\
Campfire & Bedroll or tent & Hard (14+) & Good night's rest & Good
night's rest & Good night's rest \\
Campfire & Bedroll \& tent & Daunting (11+) & Good night's rest & Good
night's rest & Good night's rest \\
\end{longtable}

\textbf{If the check succeeds:} The character gets a good night's sleep
and regains 1hp overnight.

\textbf{If the check fails:} The character fails to get a good night's
sleep and suffers one level of fatigue due to lack of sleep.

\subsection{Hunting}\label{hunting}

In the wilderness, the party can stop and hunt for food. Hunting must be
the sole activity that day (i.e.~no resting or travelling), and
generates a random encounter check for that hex. So long as the hex has
game, hunting always feeds the party for the day, even if the hunting
check fails; if successful, it also provides 1 additional standard
ration for every point over the result needed.

Hunting is a Daunting (11+) check, though the Fieldcraft feat lowers the
difficulty in favored terrain to Moderate (8+). However, some hexes may
be more or less difficult, others might require a character with
Fieldcraft and the matching favored terrain to hunt in at all, while
still others are too barren for any hunter.

\subsection{Weather}\label{weather}

At the start of each day, the referee rolls 1d8 (the travel die) and 1d6
(the combat die).

On a roll of 1 on the travel die, heavy rain, snow, a sandstorm, or
other similar natural environmental hindrance appropriate to the terrain
occurs: apply the standard +1 Hex Terrain Modifier from such for all
hexes entered that day. If rain is falling, the good roads hex terrain
bonus does not apply that day to dirt roads (most of them), as they are
rapidly reduced to mud.

On a roll of 1 on the combat die, strong winds occur: apply the poor
conditions modifier to outdoor missile attacks that day, and \emph{Fog},
\emph{Stinking Cloud}, \emph{Cloudkill} and any similar effects do not
function outdoors.

\subsubsection{Storms}\label{storms}

If both rolls have ``1'' results, the weather is a severe storm. In
addition to the above effects, apply another +1 Hex Terrain Modifier to
all hexes entered that day (+2 total). In winter, this is extreme
snowfall, enough to make further movement impossible that day without
proper gear or magical aid. A storm's effects may linger for a while
after the storm: a +1 Hex Terrain Modifier to all hexes the next day.

\subsection{Wilderness Random
Encounters}\label{wilderness-random-encounters}

One random encounter check (aka ``wandering monster check'') is made
each time a navigation check is made. One check is also made if the
party enters a wilderness hex without need of a navigation check
(i.e.~they have a map), or if the party isn't lost but does not leave
the wilderness hex they are in that day.

To make a check, the referee rolls 1d12, subtracts the hex's base entry
cost, and applies all relevant modifiers to the roll:

\begin{longtable}[]{@{}
  >{\centering\arraybackslash}p{(\linewidth - 2\tabcolsep) * \real{0.8889}}
  >{\centering\arraybackslash}p{(\linewidth - 2\tabcolsep) * \real{0.1111}}@{}}
\toprule\noalign{}
\begin{minipage}[b]{\linewidth}\centering
Situation
\end{minipage} & \begin{minipage}[b]{\linewidth}\centering
Modifier
\end{minipage} \\
\midrule\noalign{}
\endhead
\bottomrule\noalign{}
\endlastfoot
Hex is unusually dangerous & -1 or -2 \\
Hex is safe (e.g.~patrolled) & +1 \\
Party has 1+ members with Fieldcraft for that hex's terrain type & +1 \\
Players are moving at a cautious march & +1 \\
\end{longtable}

On a result of 1 or less, an encounter occurs.

If the party spends the night in the wilderness, make an additional
random encounter check. Subtract 1 if the party keeps a fire big enough
to see by going through the night (although fire scares off some of the
creatures that notice it, like bears or wolves). If an encounter occurs,
the referee randomly rolls to see on whose watch (if any) the encounter
takes place.

After determining the type of creature(s) encountered, the referee rolls
on the following table, applying a +2 bonus if the party is moving at a
cautious march:

\begin{longtable}[]{@{}
  >{\centering\arraybackslash}p{(\linewidth - 2\tabcolsep) * \real{0.0494}}
  >{\centering\arraybackslash}p{(\linewidth - 2\tabcolsep) * \real{0.9506}}@{}}
\toprule\noalign{}
\begin{minipage}[b]{\linewidth}\centering
1d12
\end{minipage} & \begin{minipage}[b]{\linewidth}\centering
Encounter Type
\end{minipage} \\
\midrule\noalign{}
\endhead
\bottomrule\noalign{}
\endlastfoot
1 & \emph{Ambushed}: The creatures attempt to ambush the players. \\
2-4 & \emph{Stumbled Upon}: Halve the encounter distance. \\
5-8 & \emph{Normal} \\
9-11 & \emph{Brief Warning}: The players have 1 round to act before the
encounter begins. \\
12 & \emph{Sign}: Party may escape, or prepare an ambush. \\
\end{longtable}

\section{Hirelings}\label{hirelings}

\subsection{Standard Hirelings}\label{standard-hirelings}

Short-term services of simple craftsmen and laborers are relatively
easily procured, but it is harder to find individuals willing to take
service for longer than a few days, especially if considerable travel is
involved.

\begin{longtable}[]{@{}lll@{}}
\toprule\noalign{}
Hireling & Daily Rate & Monthly Rate \\
\midrule\noalign{}
\endhead
\bottomrule\noalign{}
\endlastfoot
Carpenter, Mason, Servant & 11sp & 200sp \\
Groom, Laborer, Linkboy, Pack Handler & 6sp & 120sp \\
Cook, Leatherer, Tailor & 10sp & 200sp \\
Limner & 9sp & 180sp \\
Teamster & 7sp & 140sp \\
\end{longtable}

\textbf{Carpenter}: Skilled in the working of wood, a carpenter might be
retained to construct anything from a table to a palisade. Their
expertise is also invaluable for the manufacturing of shields and
similar items.

\textbf{Cook}: Familiar with the preparation of various types of food,
and sometimes also knows a little herblore.

\textbf{Groom}: Proficient in the care of horses, an attentive groom can
usually tell a good mount from a bad; also known as an ostler or stable
hand.

\textbf{Laborer}: Essentially unskilled, laborers are suitable for only
the most menial sorts of work; this category includes bearers and
porters, each of which is able to carry up to 12 slots of items, or
twice that if a pole or other contrivance is utilized.

\textbf{Leatherer}: Capable of producing a wide range of leather goods,
such as packs, belts or riding gear; a leatherer is indispensable for
the making of scabbards, sheathes, shields and the other leather
components of arms and armor.

\textbf{Limner}: Adept in the painting of signs and the illumination of
heraldic devices, among other similar tasks.

\textbf{Linkboy}: Usually hired to bear a lantern or torch, a linkboy is
typically (but not always) a youth.

\textbf{Mason}: Expert in the working of stone or plaster, masons are
essential for the construction of many significant buildings and
fortifications.

\textbf{Pack Handler:} Practiced in the burdening, handling and
unburdening of various pack animals.

\textbf{Servant}: Typically serving as valets, butlers, maids,
messengers or simple lackeys, servants are expected to look to the needs
of their master.

\textbf{Tailor}: Accomplished in the repair and making of clothes or
other cloth items; the services of a tailor are also required for the
production of various types of textile based armor and coverings.

\textbf{Teamster}: Experienced drivers of carts and wagons, teamsters
are usually experts at loading and unloading their vehicles, as well as
handling the animals with which they are familiar.

\subsection{Mercenaries/Men-at-Arms}\label{mercenariesmen-at-arms}

Mercenaries are as on pgs. 112-113 of \emph{OSE Advanced.} The majority
of regular men-at-arms are zero-level characters with 1d4+3 hit points.
The following additional mercenaries are available (and in some cases
required):

\textbf{Captain}: Equivalent to a 5th- to 8th-level warrior (1-4=5th,
5-7=6th, 8-9=7th, 0=8th). A captain may lead 20 men at arms and one
lieutenant per level of experience, plus any necessary sergeants; the
monthly wage demanded by a captain is equal to his level x 100sp.

\textbf{Lieutenant}: Equivalent to a 2nd- (1-7) or 3rd- (8-0) level
warrior. A lieutenant may lead ten men at arms per level of experience,
plus any necessary sergeants. A lieutenant serving under a captain
extends the number of troops the captain can effectively command and
control. The monthly wage demanded by a lieutenant is equal to his level
x 100sp.

\textbf{Sergeant}: Equivalent to a 1st-level warrior. A sergeant can
lead up to ten men independently or in service to a lieutenant or
captain. In any given company, there must be one sergeant for every five
to ten men at arms. The monthly wage required by a sergeant is ten times
that of the troop type he leads.

A player character warrior of the appropriate level may serve as a
sergeant, lieutenant or captain, as might an allied non-player character
fighter or retainer.

\subsection{Expert
Hirelings/Specialists}\label{expert-hirelingsspecialists}

Obtaining the services of very skilled craftsmen and other professional
servitors typically involves the expenditure of considerable time and
resources. While it is possible to retain such hirelings for short
periods, few will agree to a term of less than a month and most expect
to serve considerably longer.

\begin{longtable}[]{@{}ll@{}}
\toprule\noalign{}
Specialist & Monthly Wage \\
\midrule\noalign{}
\endhead
\bottomrule\noalign{}
\endlastfoot
Alchemist & 1,000sp \\
Animal Trainer & 500sp \\
Armorer & 500sp* \\
Blacksmith & 220sp \\
Engineer (Architect) & 500sp* \\
Engineer (Artillerist) & 300sp \\
Engineer (Miner or Sapper) & 300sp \\
Jeweler/Gemcutter & 200sp* \\
Sage & Special \\
Scribe & 200sp \\
Ship Crew & 120sp \\
Ship Captain & 500sp \\
Spy & Special \\
Steward/Castellan & Special \\
Weaponsmith & 300sp* \\
\end{longtable}

\emph{*Cost does not include all remuneration or special fees.}

\textbf{Alchemist}: Identify potions and substances. Based on a sample
or recipe, an alchemist can produce a potion at twice the normal speed
and for half the normal cost (see \emph{OSE Advanced, Magical Research,
p126}). An alchemist may also research new potions, but this takes twice
as long and costs twice as much as normal.

\textbf{Animal Trainer}: Specialized trainers are required for exotic
animals or larger numbers of normal animals. A trainer can have up to
six animals under their care at a time. It will take a minimum of one
uninterrupted month to teach an animal the first new behavior or trick.
After this first month, an animal has become accustomed to the trainer
and can be taught additional behaviors at twice the rate (two weeks per
behavior).

\textbf{Armorer}: Required for the production and maintenance of armor
and shields; for every 60 men at arms or barded warhorses present, there
must be at least one armorer available. Each must be provided with a
workroom, forge, and assistants at an additional cost
(\textasciitilde400sp). An armorer can use spare time (prorated based on
number of supported troops) to make additional armor, helmets, or
shields at 25\% of their usual cost. Per month, an armorer can make
three shields or one suit of armor.

\textbf{Blacksmith:} Essential for the basic maintenance of a stronghold
and any resident soldiery; for every blacksmith retained the needs of up
to one hundred and twenty men or horses can be met, but there must be at
least one in every stronghold and a workroom and forge must be provided
for each (\textasciitilde400sp). Besides the usual duties (horseshoes,
nails, hinges, etc.) a hired smith can turn out some basic weaponry each
month: 30 arrowheads or quarrel tips, or 10 spear heads, or 5
morningstars, or 2 flails or polearm heads.

\textbf{Engineer (Architect)}: Necessary for the successful construction
of any but the most simple of surface structures. An architect requires
payment by the month, even for short projects, and expects to receive an
additional sum equal to 10\% of the total building costs. Unless the
construction site was approved by an architect, there is a 75\% chance
that any structure will collapse in 1d100 months.

\textbf{Engineer (Artillerist)}: Mandatory for the construction and
correct operation of siege weapons, such as the trebuchet or ballista.
No such engines can be made or properly used without the services of
such an individual. If employment is for short term only, say a few
months or less, then rates of pay and costs will be increased from 10\%
to 60\%.

\textbf{Engineer (Miner or Sapper)}: Indispensable for the overseeing of
any mining operations, underground construction, or siege and counter
siege works that involve trenches, fortifications, assault towers and
other similar siege devices.

\textbf{Jeweler/Gemcutter}: Able to speedily and accurately appraise the
value of most gems, jewelry and other precious objects, a jeweler is
also capable of repairing, enhancing or newly creating ornamented items
and jewelry. The total value of the materials can be increased by from
10\% to 40\%, depending on the skill of the jeweler. Likewise, a
gemcutter might well increase the value of a rough or poorly cut stone
(those under 5,000sp base value), or the stone might be ruined in the
process. Note that jeweler/gemcutters cannot be held responsible for
damage. Dwarven jeweler/gemcutters add 20\% to skill level determination
rolls, but cost twice as much to employ.

\begin{longtable}[]{@{}
  >{\raggedright\arraybackslash}p{(\linewidth - 2\tabcolsep) * \real{0.2794}}
  >{\raggedright\arraybackslash}p{(\linewidth - 2\tabcolsep) * \real{0.7206}}@{}}
\toprule\noalign{}
\begin{minipage}[b]{\linewidth}\raggedright
Jeweler Skill Level
\end{minipage} & \begin{minipage}[b]{\linewidth}\raggedright
\end{minipage} \\
\midrule\noalign{}
\endhead
\bottomrule\noalign{}
\endlastfoot
01-20 & Fair---10\% increase 90\% likely \\
21-50 & Good---20\% increase 50\% likely, +10\% otherwise \\
51-75 & Superior---30\% increase 60\% likely, +10\% otherwise \\
76-90 & Excellent---40\% increase 70\% likely, +10\% otherwise \\
91-00 & Masterful---40\% increase 60\% likely, +20\% otherwise \\
\end{longtable}

\begin{longtable}[]{@{}
  >{\raggedright\arraybackslash}p{(\linewidth - 2\tabcolsep) * \real{0.2958}}
  >{\raggedright\arraybackslash}p{(\linewidth - 2\tabcolsep) * \real{0.7042}}@{}}
\toprule\noalign{}
\begin{minipage}[b]{\linewidth}\raggedright
Gemcutter Skill Level
\end{minipage} & \begin{minipage}[b]{\linewidth}\raggedright
\end{minipage} \\
\midrule\noalign{}
\endhead
\bottomrule\noalign{}
\endlastfoot
01-30 & Shaky---d12, one roll, 1 improves, 10-12 ruins stone \\
31-60 & fair---d12, one roll, 1-2 improves, 12 ruins \\
61-90 & good---d12, one roll, 1-3 improves, 12 ruins \\
91-00 & Superb---d20,1-5 improves, 20 ruins stone \\
\end{longtable}

\emph{Note: Giving a gem to a jeweler/gemcutter to improve counts as
``spending'' it for XP purposes; you gain XP equal to the gem's original
value. Note any increase in value as ``profit'' in your inventory (e.g.,
``ring {[}200 sp profit{]}'') and log it as mercantile income for the
month if/when sold.}

\textbf{Sage}: A person with a degree of knowledge on just about
everything, a lot of knowledge in a few specific fields, and
authoritative knowledge in his or her special fields of study. Each sage
specializes in one or more minor fields of study, and a handful of
special categories within a major field of study. Only fighters,
paladins, rangers, and thieves are able to hire a sage, though anyone
can consult one; a sage will only accept service on a permanent,
lifetime basis. As a sage will bring nothing save thinking ability and
knowledge, an offer of employment must consider the following:

\begin{longtable}[]{@{}ll@{}}
\toprule\noalign{}
\endhead
\bottomrule\noalign{}
\endlastfoot
Support \& Salary per Month & 200 to 1200sp \\
Research Grants per Month & 200 to 1200sp \\
Initial Material Expenditure & 20,000sp minimum* \\
\end{longtable}

\emph{*A 20,000sp expenditure will allow the sage to operate at 50\% of
normal efficiency, and for each additional 1,000sp thereafter, the sage
will add 1\% to efficiency until 90\% is reached (upon expenditure of
60,000sp). After 90\%, to achieve 100\% efficiency the cost per 1\% is
4,000sp (for the obviously erudite and rare tomes, special supplies and
equipment, etc. - assuming such are available, of course). All told,
expenditures must be 100,000sp for 100\% sage efficiency in specific and
exacting question areas.}

\textbf{Scribe}: Practiced in the art of writing, a typical scribe is
expected to keep records, write letters and copy documents. Others may
possess additional skills, such as cartography, counterfeiting,
cryptography, illuminating or the ability to write, read or otherwise
comprehend more than one language. Such accomplished individuals might
command up to ten times the standard wage.

\textbf{Ship Crew}: Skilled workers who can handle a ship. Sailors can
fight to defend their ship, typically being equipped with a sword,
shield, and leather armor.

\textbf{Ship's Captain}: A captain is required for any large ship, is
skilled like a sailor, and has an intimate knowledge of the particular
coasts they frequent.

\textbf{Spy}: Recruited to secretly watch the actions of others and
gather information, fees may vary wildly, from perhaps a mere hundred
silver pieces to many thousands, depending on the individual and the
difficulty of what is asked.

\textbf{Steward/Castellan}: Responsible for the administration of a
stronghold in the absence or inability of a player character, a steward
holds a position of great prestige and trust. Whilst serving within the
stronghold, a steward is capable of leading forty men at arms and two
lieutenants for every level of experience he possesses, as well as the
necessary number of sergeants. The monthly wage due to a steward is
equal to his level x 100sp. A retainer of an appropriate class and level
could be appointed as steward.

\textbf{Weaponsmith}: Required for the production and maintenance of
weaponry; for every sixty men at arms present, there must be at least
one weaponsmith available. Each must be provided with a workroom, forge,
and assistants at an additional cost (\textasciitilde400sp). A
weaponsmith can use spare time (prorated based on number of supported
troops) to make additional weapons at a rate of five weapons per month
at 25\% of their usual cost.

\end{multicols}

\end{document}
